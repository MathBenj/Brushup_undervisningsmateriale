\section{Brøker}
\begin{frame}{Brøker}
\begin{itemize}
		\setlength\itemsep{1em}
	\item<1-> \emph{Brøker} er tal på formen
	\begin{align*}
	\frac{a}{b},
	\end{align*}
	hvor $a$ og $b$ er reelle tal og $b\neq 0$.
	\item<2->  Vi kalder $a$ for brøkens \emph{tæller} og $b$ for brøkens \emph{nævner}.
	\item<3-> En brøk $\frac{a}{b}$ skal forstås som $a$ divideret med $b$.
	\item<4-> Vi vil kun tænke på $\frac{a}{b}$ som decimaltal hvis $b$ går op i $a$.
	
	Eksempelvis er $\frac{1}{3}\neq 0.33$.
\end{itemize}

\end{frame}
\begin{frame}{Brøker}{Regneregler}
\begin{itemize}
		\setlength\itemsep{1em}
	\item<1-> For brøker har vi følgende regneregler
	\begin{align*}
	\onslide<1->{\frac{a}{c}\pm\frac{b}{c}=\frac{a\pm b}{c},}&&\onslide<2->{\frac{a}{b}\frac{c}{d}=\frac{ac}{bd},}&&\onslide<3->{\frac{\frac{a}{b}}{\frac{c}{d}}=\frac{ad}{bc},}\\
	\onslide<4->{a\frac{b}{c}=\frac{ab}{c},}&& \onslide<5->{\frac{\frac{a}{b}}{c}=\frac{a}{bc},}&&\onslide<6->{\frac{a}{\frac{b}{c}}=\frac{ac}{b}.}
	\end{align*}
	\item<7-> Eksempler: Udregn
\end{itemize}
	\begin{align*}
\onslide<7->{\frac{4}{5}-\frac{2}{5}\setbeamercovered{}\onslide<8->{=\frac{2}{5}},}&&\onslide<9->{\frac{3}{4}\cdot\frac{9}{4} \setbeamercovered{}\onslide<10->{=\frac{27}{16}},}&&\onslide<11->{\frac{\frac{1}{2}}{\frac{3}{5}}\setbeamercovered{}\onslide<12->{=\frac{5}{6}},}\\
\onslide<13->{2\cdot\frac{4}{5}\setbeamercovered{}\onslide<14->{=\frac{8}{5}},}&& \onslide<15->{\frac{\frac{4}{3}}{7}\setbeamercovered{}\onslide<16->{=\frac{4}{21}},}&&\onslide<17->{\frac{5}{\frac{2}{3}}\setbeamercovered{}\onslide<18->{=\frac{15}{2}}.}
\end{align*}
\end{frame}

\begin{frame}{Brøker}{Forkorte/Forlænge}
\begin{itemize}
		\setlength\itemsep{1em}
	\item<1-> Man kan gange (dividere) en brøks tæller og nævner med samme tal (bortset fra 0) uden at ændre værdien af brøken:
	\begin{align*}
	\frac{a}{b}=\frac{a}{b}\cdot1=\frac{a}{b}\cdot \frac{c}{c}=\frac{ac}{bc}.
	\end{align*}
	\item<2-> Dette kaldes at forlænge (forkorte) en brøk. 
	\item<3-> Vi vil altid forkorte et svar på en opgave så meget som muligt.
	\item<4-> Eksempler: Udregn
	\begin{align*}
	\frac{1}{2}+\frac{3}{4}\setbeamercovered{}\onslide<5->{=\frac{2}{4}+\frac{3}{4}\onslide<6->{=\frac{5}{4}}},&& \onslide<7->{\frac{6}{8}\cdot\frac{1}{4}\setbeamercovered{}\onslide<8->{=\frac{3}{4}\cdot \frac{1}{4}\onslide<9->{=\frac{3}{16}}}}
	\end{align*}
	\item<10-> Eksempel: Reducer udtrykket
	\begin{align*}
	\frac{ab+a^2b}{a(1+a)}\setbeamercovered{}\onslide<11->{=\frac{a(1+a)b}{a(1+a)}\onslide<12->{=b}}.
	\end{align*}
\end{itemize}
\end{frame}



\section{Potenser}
\begin{frame}{Potenser}
\begin{itemize}
		\setlength\itemsep{1em}
	\item<1-> Hvis vi ganger et tal $x$ med sig selv $n> 0$ gange kaldes det resulterende tal for $x^n$. Altså
	\begin{align*}
	x^n=\underbrace{x \cdot x \cdot \dots \cdot x}_{\textup{n gange}}.
	\end{align*}
	\item<2-> $x$ kaldes grundtallet og $n$ kaldes eksponenten.
	\item<3-> Hvis $n<0$ så er
	\begin{align*}
	x^n=\frac{1}{x^{-n}}=\frac{1}{\underbrace{x \cdot x \cdot \dots \cdot x}_{\textup{n gange}}}.
	\end{align*}
	\item<4-> Specielt gælder at $x^0=1$ og at $0^0$ ikke defineres.
	\item<5-> Eksempler: Udregn $3^4$, $2^{-3}$ og $0^8$.
\end{itemize}
\end{frame}

\begin{frame}{Potenser}{Regneregler}
\begin{itemize}
		\setlength\itemsep{1em}
	\item<1-> For potenser har vi følgende regneregler
	\begin{align*}
	\onslide<1->{x^ax^b=x^{a+b},}&& \onslide<2->{\frac{x^a}{x^b}=x^{a-b},}&&\onslide<3->{(xy)^a=x^ay^a,}\\
	\onslide<4->{\Big(\frac{x}{y}\Big)^a=\frac{x^a}{y^a},}&&\onslide<5->{(x^a)^b=x^{ab},}&& \onslide<6->{x^{-a}=\frac{1}{x^a}.}	\end{align*}
	\item<7-> Bemærk at vi ikke har præsenteret nogle regneregler for potenser på formen $(x+y)^a$
	\item<8-> Eksempler: Udregn følgende
\end{itemize}
\begin{align*}
\onslide<8->{\frac{(2\cdot3)^2}{2^3}\setbeamercovered{}\onslide<9->{=\frac{2^2 3^2}{2^3}\onslide<10->{=\frac{9}{2}}},}&& \onslide<11->{\Big(\frac{2^3}{3}\Big)^{-2}\setbeamercovered{}\onslide<12->{=\frac{2^{-6}}{3^{-2}}\onslide<13->{=\frac{9}{64}}},}&& \onslide<14->{(-x)^2-x^2\setbeamercovered{}\onslide<15->{=x^2-x^2 \onslide<16->{=0}}.}
\end{align*}
\end{frame}


\section{Rødder}
\begin{frame}{Rødder}
\begin{itemize}
		\setlength\itemsep{1em}
	\item<1-> For ethvert $x\geq 0$ og ethvert positivt heltal $n$ findes der et tal $\sqrt[n]{x}\geq 0$ så
	\begin{align*}
	(\sqrt[n]{x})^n=x.
	\end{align*}
%	\item Bemærk, at $(x^{\frac{1}{n}})^n=x^{\frac{n}{n}}=x.$
%	Dermed er $x^{\frac{1}{n}}=\sqrt[n]{x}$.% hvilket viser at rødder er potenser med eksponent $\frac{1}{n}$.	
	\item<2-> Hvis $n$ er lige så er $(\pm \sqrt[n]{x})^n=x$. Eksempelvis er $(-2)^2=2^2$.
	\item<3-> Hvis $n$ er ulige kan man godt tage en $n$'te rod af et negativt tal. Eksempelvis er $\sqrt[3]{-8}=-2$.
%	\item Hvis $n=2$ skriver vi $\sqrt{x}$ i stedet for $\sqrt[2]{x}$.
	\item<4-> Eksempler: Udregn $\sqrt{81}$,  $\sqrt[4]{16}$ og $\sqrt[n]{x^n}$.
\end{itemize}
\end{frame}
\begin{frame}{Rødder}{Regneregler}
\begin{itemize}
		\setlength\itemsep{1em}
	\item<1-> For rødder har vi følgende regneregler
	\begin{align*}
	\onslide<1->{\sqrt[n]{x}=x^{\frac{1}{n}},}&&\onslide<2->{ \sqrt[n]{x^m}=x^{\frac{m}{n}}=(\sqrt[n]{x})^m,}&&\onslide<3->{\sqrt[n]{xy}=\sqrt[n]{x}\sqrt[n]{y},}&& \onslide<4->{\sqrt[n]{\frac{x}{y}}=\frac{\sqrt[n]{x}}{\sqrt[n]{y}}.}
	\end{align*}
	\item<5-> Bemærk, at vi har mange regneregler som er ``ens'' for rødder og potenser.
	\item<6-> Eksempler: Udregn

\end{itemize}
\begin{align*}
\onslide<6->{\sqrt[3]{5^6}\setbeamercovered{}\onslide<7->{=5^\frac{6}{3}}\onslide<8->{=5^2}\onslide<9->{=25},}&&\onslide<10->{ \sqrt{\sqrt{\sqrt{256}}}\setbeamercovered{}\onslide<11->{=\sqrt{\sqrt{16}}\onslide<12->{=\sqrt{4}\onslide<13->{=2}}},}&&\onslide<14->{\sqrt{\frac{144}{81}}\setbeamercovered{}\onslide<15->{=\frac{\sqrt{144}}{\sqrt{81}}\onslide<16->{=\frac{12}{9}\onslide<17->{=\frac{4}{3}}}},}\\
\onslide<18->{\frac{3}{\sqrt{3}}\setbeamercovered{}\onslide<19->{=\frac{3\sqrt{3}}{\sqrt{3}\sqrt{3}}\onslide<20->{=\frac{3\sqrt{3}}{3}\onslide<21->{=\sqrt{3}}}},}&& \onslide<22->{\frac{\sqrt{27}}{3}\setbeamercovered{}\onslide<23->{=\frac{\sqrt{3^2\cdot 3}}{3}\onslide<24->{=\frac{3\sqrt{3}}{3}\onslide<25->{=\sqrt{3}}}}.}
\end{align*}
\end{frame}

\section{Kvadratsætninger}
\begin{frame}{Kvadratsætninger}
	\vspace{-0.5cm}
\begin{itemize}
		\setlength\itemsep{1em}
	\item<1-> Vi har set at $(xy)^n=x^ny^n$. Vi vil nu se at udtrykket $(x+y)^n$ ikke er helt så let at håndtere. 
	\item<2-> Vi har følgende formler
	\begin{align*}
	(a+b)^2&=a^2+b^2+2ab\\
	(a-b)^2&=a^2+b^2-2ab\\
	(a+b)(a-b)&=a^2-b^2.
	\end{align*}
	\item<3-> Eksempler: Reducer
\end{itemize}
\begin{align*}
\onslide<3->{(x+y)^2+(x-y)^2-x^2-y^2\setbeamercovered{}\onslide<4->{=x^2+y^2+2xy+x^2+y^2-2xy-x^2-y^2}\onslide<5->{=x^2+y^2}}\\
\onslide<6->{ \frac{1}{a+b}+\frac{1}{a-b}\setbeamercovered{}\onslide<7->{=\frac{a-b}{(a+b)(a-b)}+\frac{a+b}{(a+b)(a-b)}\onslide<8->{=\frac{2a}{a^2-b^2}}}}\\
\onslide<9->{\frac{2x^2+2-4x}{2x^2-2}\setbeamercovered{}\onslide<10->{=\frac{2(x^2+1-2x)}{2(x^2-1)}\onslide<11->{=\frac{2(x+1)^2}{2(x+1)(x-1)}\onslide<12->{=\frac{x+1}{x-1}}}}}
\end{align*}
\end{frame}

