\newpage
\section{Math101 answers}
\begin{enumerate}
	
	\item The answers are:: $f(-1)=2$ and $f(2)=17$.
	
	\item The answers are:
	\begin{itemize}
		\item No since then $f(0)$ would be equal to both $2$ and $0$.
		\item $f_+(x) = 1+\sqrt{1-x^2}$.
		\item $f_-(x) = 1-\sqrt{1-x^2}$
	\end{itemize}	
	
	\item The answer is $(f\circ g)(x)=x$.
	
	\item The answers are:
\begin{align*}
D(f)=\R\setminus\{1\},&& D(g)=\R\setminus\{-1,1\},&& D(h)=[\frac{3}{2},\infty[.
\end{align*}
	
	
	\item  The answers are $(f\circ g)(1)=\frac{\sqrt{2}}{2}$ and $(g\circ f)(1)=\frac{1}{2}$, hence $f\circ g\neq g\circ f$
	
	\item The point of intersection is $(\frac{1}{4},\frac{7}{4})$.	
	
		\item The answers are $(f\circ g)(x)=1$ and $(g\circ f)(x)=5$.
	
		\item The answers are:
	\begin{align*}
	D(f)=\R,&& D(g)=\R\setminus\{1,3\},&& D(h)=[0,2].
	\end{align*}
	
	
	

	\item Choose $f(x)=e^x$ and $g(x)=2x^2-1$.
	
	\item The point of intersection is $(-1,1)$.
	
	\item Choose $f(x)=x^2$, $g(x)=\sin(x)$ and $h(x)=3x$.
	
	\item The answers are:
	\begin{align*}
	f(g(x))=\frac{3x^2}{(1-2x)^2},&& f(h(x))=\frac{3}{x},&& h(g(x))=\frac{1}{\sqrt{x}}+2,\\ h(f(x))=\sqrt{3}\frac{1}{x-2}+2,&&g(f(h(x)))=\frac{x}{3}.
	\end{align*} 
	
		\item No.
	
\item \label{it:fun5} In Figure~\ref{fig:fun5} is sketched a function which satisfies:
\begin{enumerate}
	\item has domain $[-1,1]$,
	\item intersects the points $(-1,0)$ and $(1,1)$,
	\item intersects the $y$-axis at $-1$.
\end{enumerate}
Note that many other functions satisfy these conditions.


\begin{figure}
	\centering
	\begin{tikzpicture}
	\begin{axis}[xmin=-1,xmax=1,ymin=-1,ymax=1,axis x line=center,
	axis y line=center, restrict y to domain =-5:5]
	\addplot[thick,blue,samples=200, domain= -1:0] {-x-1};
	\addplot[thick,blue,samples=200, domain= 0:1] {(2*x-1)};
	\end{axis}
	\end{tikzpicture}
	\caption{Exercise~\ref{it:fun5}.}
	\label{fig:fun5}
\end{figure}
	
\end{enumerate}