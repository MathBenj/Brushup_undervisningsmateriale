%%%
%%% Master document
%%% For Brush-up kursusmateriale
%%%
%%
%% Include preamble and marcos
%%
%%%
%%% Preamble
%%% 
\documentclass[12pt,oneside,a4paper,reqno]{book}  %%
%% Usual stuff
%%
\usepackage[utf8]{inputenc}
\usepackage[danish]{babel}
\usepackage[type={CC},
modifier={by-nc},
version={4.0},]{doclicense}
% \usepackage[T1]{fontenc}
\usepackage[nosf]{kpfonts}
\usepackage[t1]{sourcesanspro}
\usepackage{multicol}
\usepackage{xcolor}
\usepackage{hyperref}
\hypersetup{%
	%pdfpagelabels=true,%
	plainpages=false,%
	pdfauthor={Author(s)},%
	pdftitle={Title},%
	pdfsubject={Subject},%
	bookmarksnumbered=true,%
	pdfstartview=FitH%
}
\usepackage[a4paper,left=1in,right=1in,top=1in,bottom=1in,footskip=.25in]{geometry}
\setlength{\headheight}{26pt}
\usepackage{lscape}

%%
%% Math
%%
\usepackage{amsmath}
\usepackage{amssymb}
\usepackage{mathtools}
\usepackage{mathrsfs}
\usepackage{amsthm}
\usepackage{mdframed}
\usepackage{fancyhdr}
% \usepackage{lastpage}
\usepackage{microtype}
\usepackage{titlesec}

\titleformat{\chapter}[block]
  {\normalfont\huge\bfseries}{\thechapter.}{1em}{\Huge}
\titlespacing*{\chapter}{0pt}{-19pt}{0pt}


\fancyhead[R]{\thepage} %even page - chapter title
\fancyhead[L]{\small\nouppercase\rightmark} %uneven page - section title
\fancyfoot{}


\usepackage{tikz}
\usepackage{pgfplots}
\pgfplotsset{
	width=7cm,
	compat=1.3,
	inner axis line style={=>},
	axis lines=middle,
	scale only axis,
 	xlabel={\small $x$},
  	ylabel={\small $y$},
}
\usetikzlibrary{patterns,calc}

\usepackage{booktabs}
\usepackage{array}
\usepgfplotslibrary{fillbetween}
\colorlet{colorgray}{gray!45}


%Til formelsamling
\newcommand{\header}{
}

\makeatletter
\newcommand{\mysection}{\@startsection{section}{1}{0mm}%
                                {.2ex}%
                                {.2ex}%x
                                {\sffamily\small\bfseries}}
\newcommand{\mysubsection}{\@startsection{subsection}{1}{0mm}%
                                {.2ex}%
                                {.2ex}%x
                                {\sffamily\footnotesize
                                	\bfseries}}



\def\multi@column@out{%
   \ifnum\outputpenalty <-\@M
   \speci@ls \else
   \ifvoid\colbreak@box\else
     \mult@info\@ne{Re-adding forced
               break(s) for splitting}%
     \setbox\@cclv\vbox{%
        \unvbox\colbreak@box
        \penalty-\@Mv\unvbox\@cclv}%
   \fi
   \splittopskip\topskip
   \splitmaxdepth\maxdepth
   \dimen@\@colroom
   \divide\skip\footins\col@number
   \ifvoid\footins \else
      \leave@mult@footins
   \fi
   \let\ifshr@kingsaved\ifshr@king
   \ifvbox \@kludgeins
     \advance \dimen@ -\ht\@kludgeins
     \ifdim \wd\@kludgeins>\z@
        \shr@nkingtrue
     \fi
   \fi
   \process@cols\mult@gfirstbox{%
%%%%% START CHANGE
\ifnum\count@=\numexpr\mult@rightbox+2\relax
          \setbox\count@\vsplit\@cclv to \dimexpr \dimen@-1cm\relax
\setbox\count@\vbox to \dimen@{\vbox to 0cm{\header}\unvbox\count@\vss}%
\else
      \setbox\count@\vsplit\@cclv to \dimen@
\fi
%%%%% END CHANGE
            \set@keptmarks
            \setbox\count@
                 \vbox to\dimen@
                  {\unvbox\count@
                   \remove@discardable@items
                   \ifshr@nking\vfill\fi}%
           }%
   \setbox\mult@rightbox
       \vsplit\@cclv to\dimen@
   \set@keptmarks
   \setbox\mult@rightbox\vbox to\dimen@
          {\unvbox\mult@rightbox
           \remove@discardable@items
           \ifshr@nking\vfill\fi}%
   \let\ifshr@king\ifshr@kingsaved
   \ifvoid\@cclv \else
       \unvbox\@cclv
       \ifnum\outputpenalty=\@M
       \else
          \penalty\outputpenalty
       \fi
       \ifvoid\footins\else
         \PackageWarning{multicol}%
          {I moved some lines to
           the next page.\MessageBreak
           Footnotes on page
           \thepage\space might be wrong}%
       \fi
       \ifnum \c@tracingmulticols>\thr@@
                    \hrule\allowbreak \fi
   \fi
   \ifx\@empty\kept@firstmark
      \let\firstmark\kept@topmark
      \let\botmark\kept@topmark
   \else
      \let\firstmark\kept@firstmark
      \let\botmark\kept@botmark
   \fi
   \let\topmark\kept@topmark
   \mult@info\tw@
        {Use kept top mark:\MessageBreak
          \meaning\kept@topmark
         \MessageBreak
         Use kept first mark:\MessageBreak
          \meaning\kept@firstmark
        \MessageBreak
         Use kept bot mark:\MessageBreak
          \meaning\kept@botmark
        \MessageBreak
         Produce first mark:\MessageBreak
          \meaning\firstmark
        \MessageBreak
        Produce bot mark:\MessageBreak
          \meaning\botmark
         \@gobbletwo}%
   \setbox\@cclv\vbox{\unvbox\partial@page
                      \page@sofar}%
   \@makecol\@outputpage
     \global\let\kept@topmark\botmark
     \global\let\kept@firstmark\@empty
     \global\let\kept@botmark\@empty
     \mult@info\tw@
        {(Re)Init top mark:\MessageBreak
         \meaning\kept@topmark
         \@gobbletwo}%
   \global\@colroom\@colht
   \global \@mparbottom \z@
   \process@deferreds
   \@whilesw\if@fcolmade\fi{\@outputpage
      \global\@colroom\@colht
      \process@deferreds}%
   \mult@info\@ne
     {Colroom:\MessageBreak
      \the\@colht\space
              after float space removed
              = \the\@colroom \@gobble}%
    \set@mult@vsize \global
  \fi}

\makeatother
%%%
%%% Macros
%%%
%%
%% Theorem-like environments 
%%
%
% The theorem style follows the recommendations of AMS (see amsthm documentation page 7)
%
\theoremstyle{plain}
\newtheorem{theorem}{Theorem}[section]
\newtheorem{lemma}[theorem]{Lemma}
\newtheorem{proposition}[theorem]{Proposition}
\newtheorem{corollary}[theorem]{Corollary}
%
\theoremstyle{definition}
\newtheorem{definition}[theorem]{Definition}
\newtheorem{example}[theorem]{Example}
%
\theoremstyle{remark}
\newtheorem{remark}[theorem]{Remark}
%%
%% Notation
%%
%
% Spaces
%
\newcommand{\Cinf}{C^\infty}
\newcommand{\No}{\mathbf{N}_0}
\newcommand{\N}{\mathbf{N}}	
\newcommand{\Z}{\mathbf{Z}}
\newcommand{\R}{\mathbf{R}}
\newcommand{\C}{\mathbf{C}}
\newcommand{\SR}{\mathscr{S}(\R^d)}
\newcommand{\SpR}{\mathscr{S}'(\R^d)}
\newcommand{\sH}{\mathscr{H}}
%
% Functions/Operators
% 
\newcommand{\abs}[1]{\vert #1\vert}
\newcommand{\pair}[2]{\langle #1,#2\rangle}
\newcommand{\ip}[2]{\langle #1,#2\rangle}
\newcommand{\norm}[1]{\Vert #1\Vert}
\newcommand{\Op}{\mathrm{Op}}
\newcommand{\jn}[1]{\langle#1\rangle}


\newcommand{\boundellipse}[3]% center, xdim, ydim
{(#1) ellipse (#2 and #3)
}
%%
%% Document
%%


\author{Benjamin Buus Støttrup}
\begin{document}
\pagestyle{plain}
\begin{titlepage}
    \begin{center}
        \vspace*{1cm}
        \Huge
        \textbf{Opgaver og facit til Math101}

        \vspace{0.5cm}
        \large
        Udarbejdet til Math101 kurset på Aalborg Universitet i 2018.
        \vspace{1.5cm}

        af

        \vspace{1.5cm}
        \textbf{Benjamin Buus Støttrup}

        \vfill
        Senest opdateret \today
        \vspace{2cm}




    \end{center}
    \doclicenseThis
 \end{titlepage}



\pagenumbering{roman}
\tableofcontents
\chapter*{Introduktion}\normalsize
Dette dokument indeholder en formelsamling, opgaver og facit til anvendelse i kurset Math101. Math101 er et genopfriskningskursus der typisk afholdes i efterårssemestret for de førsteårsstuderende der i løbet af semestret har fundet ud af at deres matematikniveau kunne trænge til løft. Formålet med Math101 er at give deltagerene en genopfriskning af de matematiske regnefærdigheder, primært ved at de skal løse et stort antal opgaver. Math101 kurset blev revideret i 2018 i forbindelse med at Brush-up kurserne også fik en større revidering. Math101 er tiltænkt et tilbud til de studerende der ikke benyttede sig af Brush-up, men som alligevel har brug for mere matisk rutine. Indholdet i Math101 er derfor en kraftigt reduceret version af Brush-up. Selvom emnerne går igen er opgaverne forskellige.

Math 101 består af 9 korte lektioner omhandlende følgende emner 
\begin{enumerate}
    \item Brøker, Potenser, Rødder, Kvadratsætninger.
    \item Første- og andengradsligninger.
    \item Introduktion til funktioner, herunder første-og andengradspolynomier.
    \item Inverse funktioner, Logaritme- og eksponentialfunktioner samt trigonometriske funktioner.
    \item Introduktion til differentialregning.
    \item Produktregelen, kvotientregelen og kædereglen.
    \item Bestemte og ubestemte integraler.
    \item Delvis integration og integration ved substitution.
\end{enumerate}
En lektion består typisk af en kort gennemgang af lektionens emne ud fra slides samt regning af opgaver. Alle andre opgaver bør løses uden brug af lommeregner o.l.\ elektroniske hjælpemidler. \emph{Har man ikke tænkt sig at regne i hånden kan man lige så godt lade være med at regne opgaverne!}. Bagerst i dette dokument findes en liste med svar til opgaverne. Nogle af svarene er med uddybende forklaringer, og det er ikke alle spørgsmål som kun har en løsning.


\addcontentsline{toc}{chapter}{Introduktion}
%%
%% Include Sections
%%
\pagenumbering{arabic}
\pagestyle{fancy}
\chapter{Formelsamling}
Nedenfor findes en formelsamling der dækker de emner der gennemgås i dette dokument. Formålet med formelsamlingen er at gøre det nemmere for læseren at løse opgaverne uden at skulle bladre alt for meget i dokumentet. Formelsamlingen er meget kompakt, og er derfor også velegnet til at printe og medbringe til skriftlige eksamener o.l.\ hvor det er godt at have et kompakt opslagsværk over de mest almindelige formler.
{ \newgeometry{top=5mm,bottom=5mm,left=5mm,right=5mm}
    \fontsize{8pt}{0.5pt}\selectfont
    \setlength{\abovedisplayskip}{1pt}
    \setlength{\belowdisplayskip}{1pt}
    \pagestyle{empty}
    \setlength{\parindent}{0pt}
    \small
    \begin{multicols*}{3}
        \addtocontents{toc}{\protect\setcounter{tocdepth}{0}}
        \mysection{Brøker}
Brøker er tal på formen
\begin{align*}
\frac{a}{b},
\end{align*}
hvor $a,b$ er tal samt $b\neq 0$. $a$ er \emph{tælleren} og $b$ er \emph{nævneren}.
\mysubsection{Regneregler}
Der gælder
\begin{align*}
\frac{a}{c}\pm\frac{b}{c}&=\frac{a\pm b}{c},&&\frac{a}{b}\frac{c}{d}=\frac{ac}{bd},&&\frac{\frac{a}{b}}{\frac{c}{d}}=\frac{ad}{bc},\\
a\frac{b}{c}&=\frac{ab}{c},&&\frac{\frac{a}{b}}{c}=\frac{a}{bc},&&\frac{a}{\frac{b}{c}}=\frac{ac}{b}.
\end{align*}
\mysubsection{Forkorte/Forlænge Brøker}
Fælles faktorer kan forkortes:
\begin{align*}
\frac{a}{b}=\frac{ac}{bc}
\end{align*}

\mysection{Potenser}
Potenser er tal på formen $x^a$, 
$x$ er \emph{grundtallet} og $a$ er \emph{eksponenten}.
\mysubsection{Regneregler}
Der gælder
\begin{align*}
x^ax^b=x^{a+b},&& \frac{x^a}{x^b}=x^{a-b},&&(xy)^a=x^ay^a,\\
\Big(\frac{x}{y}\Big)^a=\frac{x^a}{y^a},&&(x^a)^b=x^{ab},&& x^{-a}=\frac{1}{x^a}.
\end{align*}

\mysection{Rødder}
Hvis $x\geq 0$ og $n\in \Z_+$ så findes et tal $\sqrt[n]{x}>0$ så
\begin{align*}
(\sqrt[n]{x})^n=x.
\end{align*}
Bemærk at $\sqrt[n]{x}=x^{\frac{1}{n}}$.
\mysubsection{Regneregler}
Der gælder
\begin{align*}
\sqrt[n]{x}=x^{\frac{1}{n}},&& \sqrt[n]{x^m}=x^{\frac{m}{n}}=(\sqrt[n]{x})^m,\\
\sqrt[n]{xy}=\sqrt[n]{x}\sqrt[n]{y},&& \sqrt[n]{\frac{x}{y}}=\frac{\sqrt[n]{x}}{\sqrt[n]{y}}.
\end{align*}

\mysection{Kvadratsætninger}
Der gælder
\begin{align*}
(a+b)^2&=a^2+b^2+2ab\\
(a-b)^2&=a^2+b^2-2ab\\
(a+b)(a-b)&=a^2-b^2.
\end{align*}
        \section{Ligninger}
Ligninger kan reduceres med følgende regler:
\begin{enumerate}
	\item Man må lægge til/trække fra med det samme tal på begge sider af et lighedstegn.
	\item Man må gange/dividere med det samme tal (undtagen 0) på begge sider af et lighedstegn.
\end{enumerate}
\subsection{Andengradsligninger}
Andengradsligninger er på formen
\begin{align}\label{eq:lig1}
ax^2+bx+c=0,
\end{align}
Løsningerne til~\eqref{eq:lig1} er
\begin{align*}
x=\frac{-b\pm \sqrt{b^2-4ac}}{2a}.
\end{align*}
\subsection{Faktorisering}
Hvis $ax^2+bx+c=0$ har rødder $r_1$ og $r_2$ så gælder.
\begin{align*}
ax^2+bx+c=a(x-r_1)(x-r_2).
\end{align*}
        \mysection{Funktioner}
En funktion $f\colon X\to Y$ tildeler alle $x\in X$ \emph{præcis ét} element $ f(x) \in Y$.

\mysubsection{Sammensatte funktioner}
Hvis $f\colon X\to Y$ og $g\colon Y\to Z$ defineres sammensætningen $g\circ f\colon X\to Z$ ved $(g\circ f)(x)=g(f(x))$. $f$ er den \emph{indre funktion}, $g$ er den \emph{ydre funktion}
\begin{center}
	\begin{tikzpicture}[scale=0.45]
		\draw \boundellipse{0,0}{0.7}{1.4};
		\draw\boundellipse{5,0}{0.7}{1.4};
		\draw\boundellipse{10,0}{0.7}{1.4};
		\node at (0.45,1.7) [label=left:$X$]{};
		\node at (5.45,1.7) [label=left:$Y$]{}; 
		\node at (10.45,1.7) [label=left:$Z$]{}; 
		\node[circle, fill,inner sep=1pt] (x) at (0.05,0) [label=below:$x$]{};
		\node[circle, fill,inner sep=1pt] (fx) at (5.05,0) [label=below:$f(x)$]{};
		\node[circle, fill,inner sep=1pt] (gfx) at (10.05,0) [label=below:$g(f(x))$]{};
		\draw[thick,->,blue] (x) to [bend right] node[pos=0.5, label=below:$f$] {} (fx) ;
		\draw[thick,->,red] (fx) to [bend right] node[pos=0.5, label=below:$g$] {} (gfx) ;
		\draw[thick,->,purple] (x) to [bend left] node[pos=0.25, label=above:$g\circ f$] {} (gfx) ;
\end{tikzpicture}%
\end{center}
\mysubsection{Inverse funktioner}
To funktioner $f\colon X\to Y$ og $g\colon Y\to X$ er hinandens \emph{inverse} hvis
\begin{align*}
f(g(y))=y,\quad \textup{og}\quad g(f(x))=x 
\end{align*}
for alle $x$ i $X$ og $y$ i $Y$.

% \begin{center}
% \begin{tikzpicture}[scale=0.7]
% \draw \boundellipse{0,0}{0.7}{1.4};
% \draw \boundellipse{5,0}{0.7}{1.4};
% \node at (0.45,1.7) [label=left:$X$]{};
% \node at (5.45,1.7) [label=left:$Y$]{}; 
% \node[circle,fill,inner sep=1pt] (x) at (0,1) [label=below:$x$] {};
% \node[circle,fill,inner sep=1pt] (gy) at (0,-1) [label=above:$g(y)$] {};
% \node[circle,fill,inner sep=1pt] (fx) at (5,1) [label=below:$f(x)$] {};
% \node[circle,fill,inner sep=1pt] (y) at (5,-1) [label=above:$y$] {};
% \draw[thick,->,blue] (x) to[bend left]node[pos =0.5, label=above:$f$] {} (fx);
% \draw[thick,->,red] (fx) to[bend left] node[pos =0.5, label=above:$g$] {} (x);
% \draw[thick,->,red] (y) to[bend left]node[pos =0.5, label=below: $g$] {} (gy);
% \draw[thick,->,blue] (gy) to[bend left]node[pos =0.5, label=below: $f$] {} (y);
% \end{tikzpicture}%
% \end{center}

\mysubsection{Polynomier}
Et førstegradspolynomium har forskrift:
\begin{equation*}
f(x)=ax+b.
\end{equation*}
Et andengradspolynomium har forskrift:
\begin{equation*}
f(x)=ax^2+bx+c.
\end{equation*}
\mysubsection{Logaritmer og eksponentialfunktioner}
\emph{Logaritmen med grundtal $a$}, $\log_a\colon ]0,\infty[\to \R$ er invers til eksponentialfunktionen $f_a(x)=a^x$ ($a>0$, $a\neq 1$). Der gælder at
\begin{equation*}
\log_a(a^x)=x\quad \textup{og} \quad a^{\log_a(y)}=y
\end{equation*}
og vi har
\begin{align*}
\ln x=\log_e x,&& \log x=\log_{10} x
\end{align*}
\mysubsection{Regneregler}
Der gælder
\begin{align*}
\log_a(xy)&=\log_a(x)+\log_a(y),\\\log_a\Big(\frac{x}{y}\Big)&=\log_a(x)-\log_a(y),\\ \log_a(x^r)&=r\log_a(x).
\end{align*}
\mysection{Trigonometriske funktioner}
De trigonometriske funktioner er defineret ud fra enhedscirklen:
\begin{center}
	\begin{tikzpicture}[scale=0.6]
	\begin{axis}[xmin=-0.05,xmax=1.1,ymin=-0.1,ymax=1.1,axis x line=center,
	axis y line=center, axis equal, xtick={0,1},ytick={0,1}]
	%PERMANENT STUFF
	\addplot[domain=0:pi/2,thick, samples=100] ({cos(deg(x))},{sin(deg(x))});
	\addplot[domain=0:sqrt(3)/2,thick] {1/sqrt(3)*x};
	\addplot[domain=0:pi/6,thick,samples=100] ({0.2*cos(deg(x))},{0.2*sin(deg(x))}) node[label=right:{\small$\theta$},pos=0.5] {};
	%cos
	\addplot[dotted,gray,thick] coordinates {(sqrt(3)/2,0) (sqrt(3)/2, 1/2)};
	\node at (axis cs: {sqrt(3)/4},-0.05) {$\cos(\theta)$};
	\addplot[thick,red,domain=0:sqrt(3)/2] {0};
	%sin
	\node at (axis cs: -0.05,1/4) {\rotatebox{90}{$\sin(\theta)$}};
	\addplot[dotted, gray,thick,domain=0:sqrt(3)/2] {1/2};	
	\addplot[thick,green]  coordinates { (0,0) (0,1/2)};
	%tan
	\addplot[thick,domain=sqrt(3)/2:1.2] {1/sqrt(3)*x};
	\node at (axis cs: 1.05,1/4) {\rotatebox{90}{$\tan(\theta)$}};
	\addplot[blue,thick] coordinates {(1,0) (1, 1/sqrt(3)};
	\end{axis}
	\end{tikzpicture}%
\end{center}
Der gælder at $ \tan(\theta)=\frac{\sin(\theta)}{\cos(\theta)}$ samt
\begin{center}
\begin{tabular}{@{} lccccc @{}}
	\toprule 
	$\theta$			& 0			&$ \frac{\pi}{6} $  		&$ \frac{\pi}{4} $ 		&$ \frac{\pi}{3}$ 			&$ \frac{\pi}{2} $		\\ \midrule
	$\sin \theta$		&0			&$ \frac{1}{2} $			&$ \frac{\sqrt{2}}{2} $	& $ \frac{\sqrt{3}}{2} $ 	& 1						\\ \midrule
	$\cos \theta$		&1			&$\frac{\sqrt{3}}{2}$		&$\frac{\sqrt{2}}{2}$	& $\frac{1}{2}$				&0	\\ \midrule
	$\tan \theta$		&0			&$\frac{1}{\sqrt{3}}$		&$1$					& $ \sqrt{3} $				&-		\\ \midrule
\end{tabular}
\end{center}





%\begin{tabular}{@{} lccc @{}}
%	\toprule 
%	$\theta$			& $\sin \theta$			& $\cos \theta$ 		& $\tan \theta$ 		\\ \midrule
%	0					&0						&1						&0						\\ \midrule
%	$ \frac{\pi}{6} $	&$\frac{1}{2}$			&$\frac{\sqrt{3}}{2}$	&$\frac{1}{\sqrt{3}}$	\\ \midrule
%	$ \frac{\pi}{4} $	&$\frac{\sqrt{2}}{2}$	&$\frac{\sqrt{2}}{2}$	&$1$					\\ \midrule
%	$ \frac{\pi}{3} $	&$\frac{\sqrt{3}}{2}$	&$\frac{1}{2}$			&$\sqrt{3}$				\\ \midrule
%	$ \frac{\pi}{2} $	&1						&0						&						\\ \bottomrule  
%\end{tabular}


















        \mysection{Differentialregning}
Den afledede af $f$ skrives som $f'=\frac{d}{dx}f=\frac{df}{dx}$.
\textcolor{white}{hemmelig tekst}

\mysubsection{Regneregler}
Der gælder at
\begin{center}
		\begin{tabular}{@{}l l@{}}
		$f(x)$      & $f'(x)$  				\\ \toprule
		$c$			& $0$ 					\\ \midrule
		$x$			& $1$					\\ \midrule
		$x^n$  		& $nx^{n-1}$			\\ \midrule
		$e^x$  		& $e^x$					\\ \midrule
		$e^{cx}$  	& $ce^{cx}$				\\ \midrule
		$a^x$  		& $a^x\ln a $			\\ \midrule
		$\ln x$ 	& $\frac{1}{x}$			\\ \midrule
		$\cos x$  	& $-\sin x$				\\ \midrule
		$\sin x$  	& $\cos x$				\\ \midrule
		$\tan x$ 	& $1+\tan^2(x)$		\\ \bottomrule  
	\end{tabular}
\end{center}
\mysubsection{Generelle regneregler}
Der gælder at
\begin{align*}
&(cf)'(x)=cf'(x)\\
&(f\pm g)'(x)=f'(x)\pm g'(x)\\
&(fg)'(x)=f'(x)g(x)+f(x)g'(x)\\
&\Big(\frac{f}{g}\Big)'(x)=\frac{f'(x)g(x)-f(x)g'(x)}{g^2(x)}\\
&\frac{d}{dx}f(g(x))=f'(g(x))g'(x).
\end{align*}
Den sidste regneregel kaldes \emph{kæderglen}.
        \mysection{Ubestemte integraler}
En funktion $f$ har \emph{stamfunktion} $F$ hvis
\begin{equation*}
F'(x)=f(x).
\end{equation*}
Det ubestemte integral af $f$ er
\begin{equation*}
\int f(x)\, dx =F(x)+k,
\end{equation*}
hvor $F'(x)=f(x)$ og $k\in \R$.
\mysubsection{Generelle regneregler}
\begin{align*}
&\int cf(x) \, d x=c\int f(x)\, dx\\
&\int f(x)\pm g(x) \, d x=\!\!\!\!\!\int f(x)\, dx\pm \!\!\!\!\int g(x) \, dx.\\
&\int f(x)g(x)\, dx=\!\!\!f(x)G(x)-\!\!\!\!\int\!\!\!\! f'(x)G(x)\, dx\\
&\int f(g(x))g'(x)\, dx =F(g(x))+k.
\end{align*}
Den 3. regel kaldes \emph{delvis integration} og den sidste kaldes \emph{integration ved substitution}.
\mysubsection{Regneregler}
Der gælder at
\begin{center}
		\begin{tabular}{@{}l l@{}}
		$f(x)$      & $\int f(x)\, dx$  				\\ \toprule
		$c$			& $cx+k$ 							\\ \midrule
		$x$			& $\frac{1}{2}x^2+k$				\\ \midrule
		$x^n$  		& $\frac{1}{n+1}x^{n+1}+k$			\\ \midrule
		$e^x$  		& $e^x+k$							\\ \midrule
		$e^{cx}$  	& $\frac{1}{c}e^{cx}+k$				\\ \midrule
		$\frac{1}{x}$ & $\ln(\vert x\vert)+k $			\\ \midrule
		$\ln x$ 	& $x\ln(x)-x+k$						\\ \midrule
		$\cos x$  	& $\sin x+k$						\\ \midrule
		$\sin x$  	& $-\cos x+k$						\\ \midrule
		$\tan x$ 	& $-\ln(\vert \cos(x)\vert)+k$		\\ \bottomrule  
	\end{tabular}
\end{center}
\mysubsection{Integration ved substitution}
Givet et integral på formen $\int f(g(x))g'(x)\, dx$ anvendes metoden:
\begin{enumerate}
	\item Lad $u=g(x)$.
	\item Udregn $\frac{du}{dx}$ og isoler $dx$.
	\item Substituer $g(x)$ og $dx$.
	\item Udregn integralet mht. $u$.
	\item Substituer tilbage.
\end{enumerate}
\mysection{Besemte integraler}
Det bestemte integral af $f$ i intervallet $[a,b]$ til
\begin{align*}
\int_a^b f(x)\, dx =[F(x)]_a^b=F(b)-F(a),
\end{align*}
hvor $F$ er en stamfunktion til $f$.
\mysubsection{Generelle regneregler}
\begin{align*}
&\int_a^b cf(x) \, d x=c\int_a^b f(x)\, dx\\
&\int_a^b \!\!f(x)\pm g(x) \, d x=\!\!\!\!\int_a^b\!\!f(x)\, dx\pm \!\!\!\!\int_a^b\!\! g(x) \, dx\\
&\int_a^b\!\!\!\!\! \!\!\!f(x)g(x)\, dx\!\!=\!\![f(x)G(x)]_a^b\!\!-\!\!\!\!\int_a^b\!\!\!\!\!\!\!\! f'(x)G(x)\, dx\\
%&\phantom{\int_a^b f(x)g(x)\, dx=}-\int_a^b f'(x)G(x)\, dx\\
&\int_a^b f(g(x))g'(x) \, dx=[F(x)]_{g(a)}^{g(b)}.
\end{align*}
\mysubsection{Integration ved substitution}
Givet et integral på formen $\int_a^b f(g(x))g'(x)\, dx$ anvendes metoden
	\begin{enumerate}
	\item Lad $u=g(x)$.
	\item Udregn $\frac{du}{dx}$ og isoler $dx$.
	\item Substituer $g(x)$, $dx$ samt grænser.
	\item Udregn integralet mht. $u$.
\end{enumerate}
\vfill
        % \mysection{Differentialligninger}


\mysubsection{Løsningsformler}
\begin{center}
\begin{tabular}{@{}l l@{}}
Differentiallign.    & Fuldstændig løsn.				\\ \toprule
$f'(x)=k$				& $f(x)=kx+c$						\\ \midrule
$f'(x)=h(x)$			& $f(x)=\int h(x) \d x$				\\ \midrule
$f'(x)=kf(x)$			& $f(x)=ce^{kx}$					\\ \midrule
$f'(x)+ af(x) =b$		& $f(x)=\frac{b}{a}+ce^{-ax}$		\\ \bottomrule  
\end{tabular}
\end{center}

\mysubsection{Panzerformlen}
Differentialligningen 
\begin{equation*}
f'(x)+a(x)f(x)=b(x)
\end{equation*}
har fuldstændig løsning
\begin{equation*}
f(x)=e^{-A(x)}\int b(x) e^{A(x)}dx +ce^{-A(x)},
\end{equation*}
hvor $A'(x)=a(x)$.
        % \mysection{Vektorer i planen}
En vektor $\vec{u}$ i planen skrives som $\vec{u}=[x,y]$ hvor $x,y\in \R$.
\mysubsection{Regneregler}
For $\vec{u}=[x_1,y_1]$, $\vec{v}=[x_2,y_2]$, $c\in \R$ er 
\begin{align*}
    \vec{u}\pm\vec{v}=\begin{bmatrix}
        x_1\pm x_2\\y_1\pm y_2
    \end{bmatrix}, &&\vec{u}\cdot\vec{v}=x_1x_2+y_1y_2 ,\\
    c\vec{u}=\begin{bmatrix*}
        cx_1\\cy_1
    \end{bmatrix*}, && \!\!\!\det(\vec{u},\vec{v})=x_1y_2-x_2y_1
\end{align*}
Længden af $\vec{u}$ er $\left\Vert \vec{u}\right\Vert=\sqrt{x_1^2+y_1^2}$.
\mysubsection{Vinklen mellem to vektorer}
For vinklen $\theta$ mellem $\vec{u}$, $\vec{v}$ er 
\begin{align*}
    \cos\theta=\frac{\vec{u}\cdot\vec{v}}{\left\Vert \vec{u}\right\Vert \left\Vert \vec{v}\right\Vert }, && \sin\theta=\frac{\det(\vec{u},\vec{v})}{\left\Vert \vec{u}\right\Vert \left\Vert \vec{v}\right\Vert}
\end{align*}
Yderligere gælder 
\begin{enumerate}
    \item $\vec{u}$ og $\vec{v}$ er ortogonale $\Leftrightarrow $ $\vec{u}\cdot\vec{v}=0$.
    \item $\vec{u}$ og $\vec{v}$ er parallelle $\Leftrightarrow$ $\det(\vec{u},\vec{v})=0$.
\end{enumerate}
\
\mysection{Vektorer i rummet}
En vektor $\vec{u}$ i rummet skrives som $\vec{u}=[x,y,z]$ hvor $x,y,z\in \R$.
\mysubsection{Regneregler}
For $\vec{u}=[x_1,y_1,z_1]$, $\vec{v}=[x_2,y_2,z_2]$ og $c\in \R$ gælder 
\begin{align*}
    &\vec{u}\pm\vec{v}=\begin{bmatrix}
        x_1\pm x_2\\y_1\pm y_2\\z_1\pm z_2
    \end{bmatrix}, && c\vec{u}=\begin{bmatrix*}
        cx_1\\cy_1\\c z_1
    \end{bmatrix*},\\
    &\vec{u}\cdot\vec{v}=x_1x_2+y_1y_2+z_1z_2. &&
\end{align*}
Længden af $\vec{u}$ er $\left\Vert \vec{u}\right\Vert=\sqrt{x_1^2+y_1^2+z_1^2}$. Krydsproduktet er givet ved 
\begin{equation*}
    \vec{u}\times \vec{v}=\begin{bmatrix}
        y_1z_2-z_1y_2\\ z_1x_2-x_1z_2\\x_1y_2-y_1x_2
    \end{bmatrix}
\end{equation*}
\mysubsection{Vinklen mellem to vektorer}
For vinklen $\theta$ mellem $\vec{u}$ og $\vec{v}$ gælder 
\begin{align*}
    \cos\theta=\frac{\vec{u}\cdot\vec{v}}{\left\Vert \vec{u}\right\Vert \left\Vert \vec{v}\right\Vert }, && \sin\theta=\frac{\left\Vert \vec{u}\times \vec{v}\right\Vert }{\left\Vert \vec{u}\right\Vert \left\Vert \vec{v}\right\Vert}
\end{align*}
Yderligere gælder 
\begin{enumerate}
    \item $\vec{u}$ og $\vec{v}$ er ortogonale $\Leftrightarrow$ $\vec{u}\cdot\vec{v}=0$.
    \item $\vec{u}$ og $\vec{v}$ er parallelle $\Leftrightarrow$ $\vec{u}\times\vec{v}=0$.
\end{enumerate}
\mysection{Linjer og Planer}
Planen/linjen gennem punktet med stedvektor $\vec{x}_0$ med normalvektor $\vec{n}$ beskrives ved alle vektorer $\vec{x}$ der løser ligningen
\begin{equation*}
    \vec{n}\cdot(\vec{x}-\vec{x_0})=0
\end{equation*}
En linje i rummet/planen gennem punktet med stedvektor $\vec{x}_0$ og retning $\vec{r}$ har parameterfremstilling
\begin{align*}
    \vec{x}_0+t\vec{r}, \quad t\in \R.
\end{align*}
        \phantom{asdfjæasdjfæsalkdjfælsadjfælksadjfælksadjfælkadsjfælksadjfælsakdjfælsakdjfælsakdjfælksadjfælakdsjfældsakjfæl}
        \vfill
    \end{multicols*}
    }
\restoregeometry
\addtocontents{toc}{\protect\setcounter{tocdepth}{1}}
\chapter{Opgaver}
\section{Math101 opgaver til 1. gang}

\begin{enumerate}
\item Omskriv følgende tal til brøker hvor nævneren er 7:
\begin{align*}
2,&& \frac{2}{14},&&  \frac{50}{35},&& \frac{3}{-7}.% ,&& \frac{12}{16},&&30.
\end{align*}

\item Udregn følgende potenser:
\begin{align*}
3^2,&& (-1)^{3},&& 2^3,&& 5^2,&& (-2)^{-2}.
\end{align*}

\item Udregn følgende:
\begin{align*}
\sqrt{0},&& \sqrt{9},&& \sqrt{\frac{1}{36}},&& \sqrt{\sqrt{16}}.
\end{align*}
\item Reducer følgende udtryk
\begin{align*}
(x-1)^2-(x-1)(x+1),&& (3x+y)^2-(x^2+5xy).
\end{align*}

\item Udregn følgende tal (forkort mest muligt):
\begin{align*}
\frac{2}{3}-\frac{4}{3},&&4\c\frac{3}{8},&& \frac{1}{2}-\frac{1}{3}+\frac{1}{4},&& \frac{5}{4}\c \frac{3}{2},&& \frac{\frac{3}{2}}{\frac{1}{4}}.
\end{align*}

\item Udregn følgende potenser
\begin{align*}
\Big(\frac{3}{2}\Big)^3,&& \frac{2^2}{2^5},&& 3^2\c 3^{-2},&& \frac{2^{-10}}{2^{-11}}.
\end{align*}

\item Reducer følgende brøker:
\begin{align*}
\frac{x^2+4-4x}{x-2},&& \frac{4x^2-4}{4x+4},&&\frac{2x^2+6x}{x^2+9+6x}.
\end{align*}

\item Reducer følgende udtryk:
\begin{align*}
\frac{\sqrt{8}}{2},&& \frac{2}{\sqrt{2}},&& \frac{\sqrt{27}}{\sqrt{54}},&& \frac{4}{\sqrt{8}}.
\end{align*}

\item Udregn følgende tal
\begin{align*}
\frac{4}{5}\Big(\frac{1}{3}+\frac{5}{12}\Big),&&\Big(\frac{1}{2}+\frac{1}{4}\Big)\Big(\frac{12}{20}-\frac{1}{5}\Big).
\end{align*}

\item Reducer udtrykkene:
\begin{align*}
\frac{(xy^2)^2}{xy^3},&& \frac{(x^2)^{-1}}{x},&& (3x)^2x^3.
\end{align*}




\item Omskriv til potenser:
\begin{align*}
x\sqrt{x},&& \sqrt{x^5},&& \frac{\sqrt{x}}{x^2},&& x^2\sqrt[3]{x}.
\end{align*}



















\end{enumerate}


\newpage
\section{Math101 exercises}
\begin{enumerate}
\item Solve the equations:
\begin{align*}
4x+2=26,&& -3x-5=0,&&-5x+7=-28,&& 8x+13=5.
\end{align*}

\item Solve the equations:
\begin{align*}
x^2=9,&&(x-3)(x+7)=0,&& 2x^2-6x+4=0,&& x^2+4x-5=0.
\end{align*}

\item Solve the equations:
\begin{align*}
3x+7=-(2x+3),&& 3(x-4)=2(x+1),&& -3x-2=-x+3.
\end{align*}

\item Solve the equations:
\begin{align*}
2x^2-6x=0,&& 3x^2-2x=0,&& x^2=\frac{1}{2}x,&& 25\Big(\frac{x}{2}\Big)^2= 1.
\end{align*}

\item Determine $a$ such that the equation
\begin{align*}
ax-\frac{1}{2}=7x+\frac{3}{2}
\end{align*}
has no solution. 

\item Factorize the polynomials to reduce the fractions:
\begin{align*}
\frac{x^2-25}{x^2+4x-5},&&\frac{x^2-3x+2}{x^2-5x+6},&& \frac{(x+2)(x^2-3x-10)}{x^2+4x+4}.
\end{align*}

\item Determine $a$ such that $x=2$ becomes a solution to the equation
\begin{align*}
\frac{a}{4}+ax=1.
\end{align*}

\item Solve the equations:
\begin{align*}
-(x+3)+2x=2(x-1)-x-1 ,&& 3(x-3)+2=3x-5.
\end{align*}


\item Solve the equations:
\begin{align*}
\frac{2}{3}\Big(x- \frac{4}{5}\Big)=\frac{2}{3},&& \frac{1}{3}(x-2)=-\frac{2}{5}\Big(x-\frac{3}{4}\Big)
\end{align*}

\item Solve the equations:
\begin{align*}
-\frac{1}{4}x^2-2x=-5,&& \frac{1}{2}x^2+3x=-\frac{5}{2},&& x^2-\frac{5}{6}x+\frac{1}{6}=0,&& 2x^2=1000.
\end{align*}



\item Solve the equations:
\begin{align*}
\sqrt{2}x+1=\sqrt{2}+5,&& \pi(2x-6)=\sqrt{8}x+12,&& \sqrt{2}(2\sqrt{2}x+\sqrt{8})=x-1.
\end{align*}

	\item 	Write quadratic equations on the form $x^2+bx+c=0$ with roots
	\begin{align*}
	1 \textup{ and } 1,&& \frac{1}{3} \textup{ and } -1,&&  -\sqrt{2} \textup{ and } \sqrt{8}.
	\end{align*}
	
	
	\item For which values of $b$ does the equation
	\begin{align*}
	\frac{7}{6}x^2+bx+\frac{21}{2}=0,
	\end{align*}
	have exactly one solution?
	
	
	\item Solve the equations:
	\begin{align*}
	x^4-3x^2+2=0,&& x^4=\frac{17}{4}x^2-1.
	\end{align*}
	(Hint: substitute $y=x^2$.)
	

	
\end{enumerate}
\subsection{Opgaver}

\begin{enumerate}
\item Udregn følgende potenser
\begin{align*}
1^{999},&&0^{123},&& 2^4,&&5^3,&& 3^4,&& 6^2.
\end{align*}
\item Udregn følgende potenser
\begin{align*}
(-3)^3,&& 6^{-2},&& \Big(\frac{1}{2}\Big)^{3},&& \Big(\frac{1}{3}\Big)^{-1},&& (-2)^4,&& 10^{-3}.
\end{align*}
\item Udregn følgende potenser
\begin{align*}
\frac{5^3}{5^2},&& \frac{2}{2^{-3}},&& 3^2\c 3^2,&& 2^{-3}(-2)^3,&& \frac{3^{12}}{3^9}.
\end{align*}
\item Antag at $0\leq x\leq1$. Bestem med udgangspunkt i \href{https://www.geogebra.org/m/Kdr3GHkr}{GeoGebra} :
\begin{enumerate}
\item Hvilke værdier af $a$ opfylder $x^a\leq x$?
\item Hvilke værdier af $a$ opfylder $x^a \geq x$?
\item Hvad sker med $x^a$ hvis $a$ vokser?
\item Hvad sker med $x^a$ hvis $a$ aftager?
\end{enumerate}
\item Antag at $x>1$. Bestem med udgangspunkt i \href{https://www.geogebra.org/m/Kdr3GHkr}{GeoGebra}:
\begin{enumerate}
\item Hvilke værdier af $a$ opfylder $x^a\leq x$?
\item Hvilke værdier af $a$ opfylder $x^a \geq x$?
\item Hvad sker med $x^a$ hvis $a$ vokser?
\item Hvad sker med $x^a$ hvis $a$ aftager?
\end{enumerate}
 
\item Reducer følgende udtryk
\begin{align*}
(2x^2)^2,&& \Big(\frac{(xy)^{3}}{x}\Big)^{-3},&& x^3\c (3x)^2x^{-4} \frac{x^0}{x^{-5}},&& \frac{(x^2)^3}{x^5}.
\end{align*}
\item Omskriv til tal på formen $2^n3^m$.
\begin{align*}
6^2,&& 3\cdot \Big(\frac{4}{3}\Big)^3,&& \frac{12^2}{2^2},&& 24\c 12^{-2}\c 6^3\c 3^{-4},&& \Big(\frac{4}{9}\Big)^{2}.
\end{align*}
\item Reducer følgende udtryk
\begin{align*}
\frac{9a^2b^5}{3(ab)^3},&& \frac{6a^3b^{-4}}{(2a^2b)^2},&& \frac{2x^{-4}y^3}{(2y^2x)^{-2}}.
\end{align*}

\item Vis at 
\begin{align*}
(a+b)^4=a^4+4a^3b+6a^2b^2+4	ab^3+b^4.
\end{align*}
(Hint: Brug at $(a+b)^4=((a+b)^2)^{2}$, samt kvadratsætningerne og Opgave~\ref{it:4} fra sidst.)
\item Reducer følgende udtryk
\begin{align*}
x^3(xy)^8y^{-4}zz^{-5},&& \frac{xz^2y^{-3}(xyz)^{-3}}{xyz^4},&& \Big(\frac{xy^4z^{-1}x^2y}{zx^5y^{2}(xy)^3}\Big)^{-2}
\end{align*}
\item Omskriv til tal på formen $\Big(\frac{1}{2}\Big)^m 3^n$.
\begin{align*}
\frac{ \Big( \frac{3}{4}\Big)^{3}2^4(3^{-2})^{3}}{3^{-3}2^{10}},&& \frac{2^3 6^3 12^3 3^{-8}}{4^2 9^3},&& \frac{(12)^{-1}\frac{3}{2^3}2^{-1}}{6^2}.
\end{align*}


\end{enumerate}

\newpage
\section{Math101 exercises}
\begin{enumerate}
	\item Calculate the following:
	\begin{align*}
	3^3,&& 2^{-1}, &&\Big(\frac{1}{-1}\Big)^{3},&&\Big(\frac{1}{2}\Big)^{-3},&& 123^0.
	\end{align*}
	\item Calculate the following:
	\begin{align*}
	\log_2(128),&& \log_{10}(100),&& \log_5\Big(\frac{1}{25}\Big),&& \ln(e^3),&&\log_{123}(1).
	\end{align*}
	
	\item Calculate the following:
	\begin{align*}
	\sin(\frac{\pi}{4})+\cos(\frac{\pi}{4}),&& \tan(\frac{\pi}{3})+\cos(\frac{\pi}{6}),&& \frac{\sin(\frac{\pi}{6})+\cos(\frac{\pi}{3})}{\sin(\frac{2\pi}{3})}.
	\end{align*}
	

	\item Calculate the following:
	\begin{align*}
	\log_{10}(4)+\log_{10}(250),&&\log_{10}(25)-\log_{10}(5)+\log_{10}(2),&& \log_3(54)+\log_3\Big(\frac{1}{2}\Big)
	\end{align*}
	
	\item Calculate the following:
	\begin{align*}
	\cos(-\frac{5\pi}{4}),&& \sin(\frac{5\pi}{3}),&&\tan(-\frac{5\pi}{4}),&& \cos(\frac{8\pi}{3}).
	\end{align*}
	
	
	\item Reduce the following expressions:
	\begin{align*}
	\ln(\sqrt{2})+\ln(2),&& \log_{10}(5^{3/2})+\frac{1}{2}\log_{10}(5)+\log_{10}(4),&& \frac{1}{4}\log_5(4^2+3^2).
	\end{align*}
	
	
	\item Calculate the following:
	\begin{align*}
	3^{\log_3(1)},&&e^{1+\ln(3)},&& 10^{-\log_{10}(7)},&& 7^{1-\log_7(9)},&& 4^{-\log_2(3)}.
	\end{align*}
	
	\item Calculate the following:
	\begin{align*}
	\cos(\frac{13\pi}{3}),&& \tan(\frac{12\pi}{6}),&& \sin(-\frac{10\pi}{4}),&& \tan(\frac{15 \pi}{5}).
	\end{align*}
	
	
	\item Solve the equations:
	\begin{align*}
	e^x=3,&& \ln(x)=4,&& \ln(2x-4)=\ln(8)+\ln(4),&& 3\log_{10}(x)=\log_{10}(27).
	\end{align*}
	

	\item Determine two different solutions to the equations:
	\begin{align*}
	\sin(x)=\frac{\sqrt{2}}{2},&& \cos(x-\pi)=-\frac{\sqrt{3}}{2},&& 2\cos^2(x)+5\cos(x)+2=0.
	\end{align*}

\item \label{it:trig3} In this exercise we determine exact values of sine and cosine for the angles $ \frac{\pi}{6}$ and $ \frac{\pi}{6} $.

\begin{enumerate}
	\item Show that $\sin(\frac{\pi}{6})=\frac{1}{2}$ by considering the triangle in Figure~\ref{fig:trig3}. (Hint: What kind of triangle is it?)
	
	\item Use the Pythagorean trigonometric identity ($ \cos^2(x)+\sin^2(x)=1 $) to show that $\cos(\frac{\pi}{6})=\frac{\sqrt{3}}{2}$.
	
	\item Show that $\sin(\frac{\pi}{3})=\frac{\sqrt{3}}{2}$. (Hint: Use that $ \sin(\frac{\pi}{3})=\sin(\frac{\pi}{6}+\frac{\pi}{6})=2\sin(\frac{\pi}{6})\cos(\frac{\pi}{6}) $)
	
	\item Use the Pythagorean trigonometric identity to show that $\cos(\frac{\pi}{3})=\frac{1}{2}$.
	
\end{enumerate}

\begin{figure}
	\centering
	\begin{tikzpicture}
	\begin{axis}[xmin=-1,xmax=1,ymin=-1,ymax=1,axis x line=center,
	axis y line=center, axis equal]
	\addplot[blue,domain=0:2*pi,thick, samples=100] ({cos(deg(x))},{sin(deg(x))});
	\addplot[domain=0:sqrt(3)/2,thick] {1/sqrt(3)*x};
	\addplot[domain=0:pi/6,thick,samples=100] ({0.2*cos(deg(x))},{0.2*sin(deg(x))}) node[label={[label distance=2pt]0.5:\small$\frac{\pi}{6}$},pos=1] {};
	\addplot[domain=0:sqrt(3)/2,thick,gray,dotted] {-1/sqrt(3)*x};
	\addplot[domain=-pi/6:0,thick,samples=100,gray,dotted] ({0.2*cos(deg(x))},{0.2*sin(deg(x))}) node[label={[label distance=2pt]0.5:\small$\frac{\pi}{6}$},pos=0] {};
	\addplot[dotted,gray,thick] coordinates {(sqrt(3)/2, -1/2) (sqrt(3)/2, 1/2)};
	\end{axis}
	\end{tikzpicture}
	\caption{Exercise~\ref{it:trig3}}
	\label{fig:trig3}
\end{figure}

\item \label{it:trig4} In this exercise we determine exact values of sine and cosine for the angle $ \frac{\pi}{4}$.
\begin{enumerate}
	\item Show that $\sin(\frac{\pi}{4})=\frac{\sqrt{2}}{2}$ by considering the triangle in Figure~\ref{fig:trig4}.(Hint: Pythagorean theorem) 
	\item Use the Pythagorean trigonometric identity to show that $\cos(\frac{\pi}{4})=\frac{\sqrt{2}}{2}$. 
\end{enumerate}

\begin{figure}
	\centering
	\begin{tikzpicture}
	\begin{axis}[xmin=-1,xmax=1,ymin=-1,ymax=1,axis x line=center,
	axis y line=center, axis equal]
	\addplot[blue,domain=0:2*pi,thick, samples=100] ({cos(deg(x))},{sin(deg(x))});
	\addplot[domain=0:sqrt(2)/2,thick] {1*x};
	\addplot[domain=0:pi/4,thick,samples=100] ({0.2*cos(deg(x))},{0.2*sin(deg(x))}) node[label={[label distance=2pt]0.5:\small$\frac{\pi}{4}$},pos=1] {};
	\addplot[domain=0:sqrt(2)/2,thick,gray,dotted] {-1*x};
	\addplot[domain=-pi/4:0,thick,samples=100,gray,dotted] ({0.2*cos(deg(x))},{0.2*sin(deg(x))}) node[label={[label distance=2pt]0.5:\small$\frac{\pi}{4}$},pos=0] {};
	\addplot[dotted,gray,thick] coordinates {({sqrt(2)/2}, -{sqrt(2)/2}) ({sqrt(2)/2}, {sqrt(2)/2})};
	\end{axis}
	\end{tikzpicture}
	\caption{Exercise~\ref{it:trig4}}
	\label{fig:trig4}
\end{figure}

\end{enumerate}
\subsection{Opgaver}

\begin{enumerate}
\item Løs ligningerne
\begin{align*}
8x+2=26,&& -3x-5=4,&&-6x+7=-29,&& 8x+11=5.
\end{align*}

\item Løs ligningerne
\begin{align*}
3x+7=-2x+2,&& 3(x-4)+2=2(x+1),&& -3x-4=-x+3.
\end{align*}

\item Løs ulighederne 
\begin{align*}
2x<4-5x,&& x-3>2-x,&& 2x-3\geq 2x,&& -2x\leq2(x-7).
\end{align*}

\item Betragt ligningen
\begin{align*}
ax+4=-x+b.
\end{align*}
Brug \href{https://www.geogebra.org/m/Q4Wh3Xrj}{GeoGebra} til at visualisere alle værdier af $a$ og $b$ så at
\begin{enumerate}
\item Ligningen har præcis en løsning.
\item Ligningen har ingen løsning.
\item Ligningen har uendeligt mange løsninger.
\end{enumerate}

\item Løs ligningerne
\begin{align*}
 3(x-2)+2=3x-8,&& -(x+1)+2x=2(x-1)-x+1
\end{align*}

\item Løs ligningerne 
\begin{align*}
\frac{1}{x-2}=5,&& \frac{x^2+8}{x+2}=x+2,&& \frac{5}{x-1}=\frac{7}{x},&&  \frac{x^2+9-6x}{2x^2-6x}=1.
\end{align*}

\item Løs ligningerne
\begin{align*}
\frac{2}{3}\Big(x- \frac{5}{2}\Big)=\frac{3}{6},&& \frac{3}{8}(4x-2)=-\frac{1}{3}\Big(x-\frac{3}{4}\Big)
\end{align*}

\item Løs ligningerne
\begin{align*}
\sqrt{2}x+4=8,&& \pi(x-1)=\sqrt{2}x+3,&& \sqrt{2}(2\sqrt{2}x-\sqrt{8})=2x+1.
\end{align*}

\item Angiv hvilken figur i planen afgrænses af følgende uligheder, for polygoner angiv hjørnernes koordinater, for cirkelskiver angiv radius og centrumskoordinater.
\begin{enumerate}
	\item $ 0\leq x\leq 2, 0\leq y\leq 2 $
	\item $ x^2+y^2\leq 4 $
	\item $ 0\leq x\leq 2, 0 \leq y\leq 2x+2 $.
\end{enumerate}


\item Opstil uligheder der beskriver de grå områder der ses i Figur~\ref{fig:ligninger1}
\begin{figure}
\centering
\begin{tikzpicture}
\draw[help lines, color=gray, dashed] (-4.9,-4.9) grid (4.9,4.9);
\draw[->,ultra thick] (-5,0)--(5,0) node[right]{$x$};
\draw[->,ultra thick] (0,-5)--(0,5) node[above]{$y$};

\draw[fill=gray] (-2,-3) circle (2);
\draw[fill=gray] (0,0)--(0,2)--(4,2)--cycle;
\draw[fill=gray] (-4,0)--(-2,0)--(-2,4)--(-4,4)--cycle;
\end{tikzpicture}
\caption{Find uligheder der beskriver de grå områder.}
\label{fig:ligninger1}
\end{figure}

\item Vis at 
\begin{align*}
ab\leq \frac{a^2}{2}+\frac{b^2}{2}.
\end{align*}
(Hint: Betragt $(a-b)^2$.)
Find derefter tal $a, b, c, d$ så
\begin{align*}
ab= \frac{a^2}{2}+\frac{b^2}{2},\quad \textup{og} \quad cd<\frac{c^2}{2}+\frac{d^2}{2}.
\end{align*}

\item Vis at
\begin{align*}
\sqrt{a+b}\leq \sqrt{a}+\sqrt{b},
\end{align*}
for $a,b\geq 0$.
(Hint: betragt $ (\sqrt{a}+\sqrt{b})^2 $). Find derefter $a,b,c,d\geq 0$ således at
\begin{align*}
\sqrt{a+b}=\sqrt{a}+\sqrt{b},\quad \textup{og}\quad \sqrt{c+d}<\sqrt{c}+\sqrt{d}.
\end{align*}

\end{enumerate}
\newpage
\section{Math101 opgaver til 6. gang}
\begin{enumerate}

	\item Differentier funktionerne
	\begin{align*}
	f_1(x)=\sqrt{x^2+1},&& f_2(x)=\frac{x}{2x+1},&& f_3(x)=x\sin(x).
	\end{align*}

	\item Differentier funktionerne:
	\begin{align*}
	f_1(x)=xe^x,&& f_2(x)=2x^2\cos(x),&& f_3(x)=\ln(x)e^x,&& f_4(x)=\sin(x)\cos(x).
	\end{align*}

	\item Differentier funktionerne (lad evt. være med at forkorte):
	\begin{align*}
	f_1(x)=\frac{x}{x-1},&&f_2(x)=\frac{x^2-x+1}{3x+2},&&f_3(x)=\frac{x^2}{x^3-2x^2}.
	\end{align*}
	
	\item Differentier funktionerne
	\begin{align*}
	f_1(x)=(3x-1)^\frac{4}{3},&& f_2(x)=\ln(x^2+3x),&& f_3(x)=e^{2-x},&&f_4(x)=\sin(x^3).
	\end{align*}
	
	\item \label{it:diff24} Bestem den afledede af funktionen $f(x)=(x-1)e^x$.
	
	
	\item \label{it:diff23}Bestem den afledede af funktionen $f(x)=x\ln(x)-x$.
	
	\item Differentier funktionerne 
	\begin{align*}
	f_1(x)=e^{x^3},&&f_2(x)=\cos^2(x),&&f_3(x)=\sin^3(x)&&f_4(x)=2\tan(x^2).
	\end{align*}
		
	\item\label{it:diff21} Vis at 
	\begin{align*}
	\frac{d}{dx} \tan x= 1+\tan^2x.
	\end{align*}
	(Hint: Brug $\tan x=\frac{\sin x}{\cos x}$)

	\item Differentier funktionen $f(x)=\frac{xe^x}{\cos(x)}$.
		
	\item Differentier funktionen
	\begin{align*}
	f(x)=\cos^2(\sqrt{x^2+1}).
	\end{align*}
		
	\item Differentier funktionerne
	\begin{align*}
	f_1(x)=\frac{\cos^2(x)}{\sin(x)},&&f_2(x)=\frac{e^{x^2}}{x},&&f_3(x)= \frac{x\cos(x)}{e^x}
	\end{align*}
	
	
	
	\item Differentier funktionerne
	\begin{align*}
	f(x)=\frac{x^2e^x}{-x\ln(x)},&& g(x)=xe^x\ln x,&& h(x)=\tan(x)e^{x}\cos(x)x^2
	\end{align*}
	
	
\end{enumerate}

\subsection{Opgaver}

\begin{enumerate}
	\item  Lad $A=\{1,2,3,4,5\}$ og $ B=\{ 3,5,\pi , 1,-1 \} $. Bestem en funktion $f\colon A\to B$ såedes at
	\begin{enumerate}
		\item $ f $ er injektiv og surjektiv.
%		\item $f $ er ikke injektiv men surjektiv.
%		\item $f$ er injektiv men ikke surjektiv.
		\item $f$ er hverken injektiv eller surjektiv.
	\end{enumerate}
	
	
	\item Kan cirklen med ligning $x^{2} +(y-1)^{2}=1$ beskrives som grafen af en funktion? Begrund dig svar, og i bekræftende tilfælde bestem da funktionen.
	
	\item Givet mængderne 
	\begin{align*}
	A&= \{0,2,4,6,8\},\\
	B&=\{\text{Hund},\textup{Hest},\textup{Struds},\textup{Fugleedderkop},\textup{Laks},\textup{Mariehøne} \}
	\end{align*}
	bestem om følgende sammenhænge mellem $A$ og $B$ er funktioner. I bekræftende fald bestem også om de er injektive og/eller surjektive.
	\begin{enumerate}
		\item Lad $f\colon B\to A$ være givet ved
		\[f(x)=\textup{antal ben }x \textup{ har}.\]
		\item Lad $g$ være givet ved
		\begin{align*}
		g(0)&=\textup{Hund},\quad g(2)=\textup{Struds},\quad g(4)=\textup{Laks},\\ g(6)&=\textup{Fugleedderkop},\quad g(8)=\textup{Hest}
		\end{align*}
		
		\item Lad $ h\colon A\to B$ være givet ved at 
		\[h(x)= \textup{Dyrene i }B \textup{ som har } x \textup{ ben}.\] 
	\end{enumerate}

	\item Lad $f(x)=x^2-1$, $g(x)=\frac{1}{1+x}$. Bestem
	\begin{enumerate}
		\item $(f+g)(2)$
		\item $ \frac{f}{g}(-2) $
		\item $(fg)(0)$
		\item $\frac{g}{f}(x)$
		\item $(g-f)(x)$.
	\end{enumerate}

	\item Bestem værdimængden for funktionen $f\colon \mathbb{R}\to \mathbb{R}$ givet ved $f(x)=-3x^2+9$.
	
	\item \label{it:fun1} I Figur~\ref{fig:fun1} ses graferne for forskellige funktioner med definitionsmængde $[-2,2]$ og codomæne $[-2,2]$. Bestem for hver funktion om den er injektiv, surjektiv og/eller bijektiv.
	\begin{figure}
		\centering
	\begin{tikzpicture}
\begin{axis}[xmin=-2.1,xmax=2.1,ymin=-2.1,ymax=2.1,axis x line=center,
axis y line=center, restrict y to domain=-2:2]
\addplot[thick,blue,samples =300] {-x^2+2};
\addplot[thick,red,samples=100] {(1+x^2)^-1};
\addplot[thick,green, samples =200] {x };
\addplot[thick, black,samples=200] { e^-x -3/2};
\end{axis}
	\end{tikzpicture}
	\caption{Opgave~\ref{it:fun1}.}
	\label{fig:fun1}
	\end{figure}

	\item Bestem en mængde $D\subset \R$ således at funktionen $f\colon D\to \R$ givet ved $f(x)=x^2$ bliver injektiv.
	
	\item \label{it:fun2} På Figur~\ref{fig:fun2} ses grafen for funktionen $f(x)=x^{-1}$. Brug grafen til at afgøre om $f$ er injektiv, surjektiv og/eller bijektiv på $\R$. Hvis ikke $f$ er bijektiv bestem så det størst mulige domæne og codomæne således at $f$ bliver bijektiv.
	
	\begin{figure}
		\centering
		\begin{tikzpicture}
		\begin{axis}[xmin=-5,xmax=5,ymin=-20,ymax=20,axis x line=center,
		axis y line=center, restrict y to domain=-50:50]
		\addplot[thick,blue,samples=400] {1/x};
		\end{axis}
		\end{tikzpicture}
		\caption{Opgave~\ref{it:fun2}.}
		\label{fig:fun2}
	\end{figure}

	\item Brug \href{https://www.geogebra.org/m/eEE7RXzU}{GeoGebra} til at bestemme for hvilke $a\in \{0,1,2,\dots,10\}$ funktionen $f\colon \R\to \R$ givet ved $f(x)=x^a$ er injektiv, surjektiv eller bijektiv. 
	
	\item Bestem den størst mulige definitionsmængde for funktionerne
	\begin{align*}
	f(x)=\sqrt{-x^2+x+2},&& g(x)=\frac{1}{(1+x^2)^\frac{1}{2}},&& h(x)=\frac{2}{\sqrt{x+2}}.
	\end{align*}
	
	\item \label{it:fun3exc} Figur~\ref{fig:fun3exc} viser forskellige kurver i planen. Argumenter for hvilke kurver der beskriver grafen for en funktion. 
	
 	\begin{figure}
		\centering
		\begin{tikzpicture}[
		%declare function={func(\x)= (\x<-3)*(-2)+ and (\x>=-3,\x<=-1)*(sqrt(1-(x+2)^2)-2)+ and (\x>-1)*(-2); }
		declare function={
			func(\x)= (\x < -3) * (-2)   +
			and(x >= -3,\x<=-1) * (sqrt(1-(x+2)^2)-2)     +
			and(x >= -1,\x<=1) * (-sqrt(1-(x)^2)-2)     +
			and(x >= 1,\x<=3) * (sqrt(1-(x-2)^2)-2)     +
			(\x>3) * (-2)
			;
			}
		]
		\begin{axis}[xmin=-5,xmax=5,ymin=-5,ymax=5,axis x line=center,
	axis y line=center, restrict y to domain =-5:5]
		\addplot[data cs=polar, thick,blue,samples=200, domain= 0:180] ({x}, {3*sin(x)*cos(x)/(sin(x)^3+cos(x)^3)});
		\addplot[thick,red,samples=400] {func(x)};
		\addplot[data cs=polar, thick,green,samples=200, domain= 0:180] ({x}, {sec(x)+2*cos(x)});
		\end{axis}
		\end{tikzpicture}
		\caption{Opgave~\ref{it:fun3exc}.}
		\label{fig:fun3exc}
	\end{figure}

	\item Skitser grafen for en funktion som opfylder alle nedenstående punkter:
	\begin{enumerate}
		\item har domæne $[-2,4[$, og codomæne $[-2,4]$,
		\item går gennem punkterne $(-1,3)$ og $(2,-2)$,
		\item skærer y-aksen i $-2$,
		\item ikke skærer x-aksen.
		\end{enumerate}
	\item I \href{https://www.geogebra.org/m/eEE7RXzU}{GeoGebra} er en kurve afbildet. Kurven afhænger af en parameter $a$. Bestem for hvilke (om nogen) $a\in \{-3,-2,\dots,2,3\}$ kurven beskriver grafen for en funktion.
\end{enumerate}

\newpage
\section{Math101 opgaver til 8. gang}

\begin{enumerate}
	\item \label{it:int21} Udregn følgende ubestemte integraler
	\begin{align*}
	\int x\cos x\dd x,&& \int x\ln x\dd x,&& \int xe^x\dd x
	\end{align*}
	
	\item Udregn følgende ubestemte integraler:
	\begin{align*}
	\int \cos(2x)\, dx,&& \int (1-x)^2\, dx,&& \int e^{2x-3}\, dx.
	\end{align*}
	
	\item Udregn følgende bestemte integraler
	\begin{align*}
	\int_0^{2\pi} x\cos(x)\, dx,&& \int_{-1}^0 xe^x\, dx,&& \int_1^2 x\ln(x)\, dx.
	\end{align*}
	
	\item Udregn følgende bestemte integraler
	\begin{align*}
	\int_{-\frac{\pi}{9}}^0 \sin(3x) \, dx,&& \int_0^2 xe^{x^2}\, dx,&& \int_1^2 \frac{2x+1}{x^2+x-1}\, dx.
	\end{align*}
	
	\item Udregn følgende integraler
	\begin{align*}
	\int (x+1)\sin(x) \dd x,&& \int_1^3 (2x-1) e^{2x} \dd x.
	\end{align*}
	
	\item Udregn følgende integraler
	\begin{align*}
	\int_{-\frac{\pi}{4}}^{\frac{\pi}{4}} \cos(x)\sin^2(x)\, dx ,&& \int_0^1 \frac{x^2}{(1+x^3)^2}\, dx
	\end{align*}
	
	\item Udregn følgende integraler
	\begin{align*}
	\int x^2e^x\, dx,&& \int_0^\pi x^2\sin(x)\, dx
	\end{align*}
	
	\item Udregn følgende integraler
	\begin{align*}
	\int \frac{1}{\sqrt{2x-1}}\, dx,&& \int_0^{\sqrt{\frac{\pi}{6}}} 6x\cos(x^2)\sqrt{\sin(x^2)}\, dx
	\end{align*}
	
	\item Udregn
	\begin{align*}
	\int_0^\pi \sin(x)\cos(x)\, dx.
	\end{align*}
\end{enumerate}

\chapter{Repetition}\label{ch:rep}
Dette afsnit indeholder repetitionsopgaver som dækker de fleste af de emner der er gennemgået i disse noter. Hvis man kan løse disse opgaver uden de store problemer, så har man et fornuftigt udgangspunkt til at komme helskindet igennem diverse matematikkurser på Aalborg Universitet.
\section{Repetition}
\begin{enumerate}
	\item Udregn
	\begin{align*}
	2+\frac{4}{2}\cdot 3,&& 1-\frac{3}{2}\cdot 4,&& \frac{1}{3}+\frac{2}{5},&& \frac{2}{3}+1.
	\end{align*}
	
	\item Udregn
	\begin{align*}
	\frac{3^3\cdot 3^{-2}}{3^4}\cdot \frac{(-3)}{3^{-2}},&& (-3\sqrt{5})^2,&&\frac{\frac{2}{3}+\frac{3}{4}}{\frac{5}{2}},&& \sqrt{7+(3\sqrt{2})^2}.
	\end{align*}
	
	\item Udregn
	\begin{align*}
	\sin(\frac{\pi}{6})+\cos(\frac{3\pi}{4})-\cos(-\pi),&& \log(20)+\log(50),&& \ln(e^3)+\ln(1).
	\end{align*}
	
	\item Reducer udrtykkene
	\begin{align*}
	(x+5)^2,&& (1-2x)^2,&& (2y-1)(2y+1),&& (x^2+y^2)-(2x^2-y^2).
	\end{align*}
	
	\item Reducer udtrykkene
	\begin{align*}
	\frac{x^2+y^2-2xy}{x^2-xy},&& \frac{x^2-y^2}{x^2-xy},&& \frac{y^2-x^2}{x+y}+x.
	\end{align*}
	
	\item Løs ligningerne
	\begin{align*}
	-2x+3=7,&& \frac{2}{3}x-3=\frac{6}{5},&&x^2-2x-3=0.
	\end{align*}
	
	\item Løs ligningerne
	\begin{align*}
	\frac{2}{x}+2x=3x,&& -2x^2+x+1=0.
	\end{align*}
	
	\item Bestem mindst en løsning til ligningerne
	\begin{align*}
	e^{2x-1}-1=0,&& \sin(x-\pi)=\frac{\sqrt{3}}{2},&& \ln(x-1)=\ln(12)-\ln(4).
	\end{align*}
	
	\item For hvilke værdier af $a$ er følgende udtryk sande
	\begin{align*}
	\frac{1}{1+a}=\frac{\sqrt{5}}{\sqrt{5}+a},&& \frac{1}{1+a}=\frac{1-a}{1-a^2},\\
	\frac{1}{1+a}=\frac{1+a}{a^2+2a+1},&& \frac{1}{1+a}=1+\frac{1}{a}.
	\end{align*}
	
	\item Differentier funktionerne
	\begin{align*}
	f(x)=2x^3-x^2+1,&& g(x)=2x^{-2}+x,&& h(x)=\frac{2}{x}+x.
	\end{align*}
	
	\item Bestem følgende integraler
	\begin{align*}
	\int 2x^2+1 \, dx,&& \int_0^1 x^2-3x+1\, dx,&& \int_0^4 \frac{1}{\sqrt{x}}+x \, dx.
	\end{align*}
	
	
	\item Differentier funktionerne
	\begin{align*}
	f(x)&=2x^3-x^{-2}+4x^{-1}-1,\\ g(x)&=3xe^3-\sqrt{x-1},\\ h(x)&=2xe^x-\sin(x^2-x).
	\end{align*}
	
	\item Udregn integralerne
	\begin{align*}
	\int 2e^{-x} \, dx,&& \int_0^1 4xe^{x^2}\, dx, && \int_0^2 \frac{2x}{\sqrt{x^2+5}}\, dx
	\end{align*}
\end{enumerate}

\chapter{Facit}
\section{Brøker}

\begin{enumerate}
\item Svarene er:
\begin{align*}
\frac{8}{4},&& \iffalse \frac{20}{4},&&  \frac{2}{4},&& \fi \frac{10}{4},&& \frac{4 \pi}{4},&& \frac{6}{4}.% ,&& \frac{3}{4},&&\frac{120}{4}.
\end{align*}
\item Svarene er:
\begin{align*}
\frac{21}{3},&& \iffalse \frac{9}{3},&&\frac{1}{3},&& \fi \frac{2}{3},&&\frac{4}{3},&&\frac{12\pi}{3 }.% && \frac{2}{3},&&\frac{3}{3e}.
\end{align*}
\item Svarene er:
\begin{align*}
2,&& \frac{5}{4},&& \frac{13}{12},&&1,&&-\frac{1}{3}.
\end{align*}
\item Svarene er:
\begin{align*}
\frac{3}{2},&&\frac{1}{12},&& \frac{7}{24},&& \frac{3}{10},&& \frac{14}{9},&&\frac{9}{25}.
\end{align*}
\item Svarene er:
\begin{align*}
4,&& \frac{3}{2},&& \frac{1}{4},&& 1,&& \frac{5}{4}.
\end{align*}
\item Svarene er:
\begin{align*}
x+1,&& x+2y,&& 2x+7,&&y,&& \frac{x+3}{2x},&& \frac{x-1}{3x}.
\end{align*}
\item Svarene er:
\begin{align*}
\frac{73}{50},&& -11,&& \frac{5}{8},&& 0,&& \frac{19}{2}.
\end{align*}
\item Svaret er:
\begin{align*}
1.
\end{align*}
\item Svarene er:
\begin{align*}
\frac{1}{2}= \frac{3}{6},&& \frac{6}{7}=\frac{42}{49},&& \frac{2x}{y}=\frac{2xy}{y^2},&& \frac{ \pi}{\sqrt{2}}=\frac{3\pi^3}{3 \pi^2 \sqrt{2}}.
\end{align*}
\item Svarene er:
\begin{align*}
\frac{9}{11},&& 14,&&\frac{1}{11}.
\end{align*}
\item Svaret er:
\begin{align*}
6.
\end{align*}
\item Vi regner på venstresiden og får
\begin{align*}
\frac{ \frac{1}{b}+1 }{1- \frac{a}{b}}=\frac{b}{b}\frac{ \frac{1}{b}+1 }{1- \frac{a}{b}}=\frac{1+b}{b-a},
\end{align*}
hvor $b\notin \{0,a\}$.
\item Vi regner på venstresiden og får
\begin{align*}
\frac{ \frac{a}{b}+1}{ \frac{b}{a}+1}=\frac{ \frac{a}{b}+\frac{b}{b} }{ \frac{b}{a}+\frac{a}{a} }=\frac{a+b}{b}\c \frac{a}{a+b}=\frac{a}{b}.
\end{align*}
hvor $a,b\neq 0$
\end{enumerate}

\newpage
\section{Kvadratsætninger}
\begin{enumerate}
\item Svarene er:
\begin{align*}
x^2+1+2x,&& 4x^2+9-12x,&& x^2,&& 9a^2+4b^2-6ab.
\end{align*}
\item Svarene er:
\begin{align*}
\frac{x+3}{2x},&& \frac{2x+3}{2x-3},&& \frac{2x+6}{x},&& \frac{x-2y}{2}.
\end{align*}
\item Centrumskoordinaterne og radierne er:
\begin{align*}
(0,0), r=1,&& (1,-1), r=5,&& (-2,0), r=2.
\end{align*}
\item Vi har at
\begin{align*}
99^2-101^2&=(99-101)(99+101)=-2(200)=-400,\\
999^2&=(1000-1)^2=1000^2+1^2-2000= 998001,\\
499^2-501^2&=(499-501)(499+501)=-2(1000)= -2000,\\ 
99998^2-100002^2&=(99998-100002)(99998+100002)=-4(200000)=-800000.
\end{align*}
\item Ved at reducere fås
\begin{align*}
8-4a,&& 2x,&&2x-3.
\end{align*}
\item Centrumskoordinaterne og radierne er:
\begin{align*}
(3,4),r=5,&& \Big( \frac{1}{2},-\frac{1}{2}\Big), r=1.
\end{align*}
\item \label{it:1ans} Arealet af figuren i Figur~\ref{fig:1} er $(a+b)^2$ hvilket figuren viser kan beskrives som $a^2+b^2+2ab$.
\item \label{it:2ans} Den højre del af Figur~\ref{fig:2} viser, at summen af det grå areal og det skraverede areal er $(a+b)(a-b)$. Den venstre figur viser, at dette areal er det samme som $a^2-b^2$.
\item \label{it:3ans} Det totale areal som ses i Figur~\ref{fig:3} kan beskrives både som $(a+b)^2$ og som $c^2+4( 1/2) ab$. Dette giver ligningen
\begin{align*}
c^2+4 \frac{1}{2}ab=(a+b)^2,
\end{align*}
som kan reduceres til $c^2=a^2+b^2$.

\item Svarene er:
\begin{align*}
a^2+36b^2+12ab,&& 16-a^2,&& x^2+\frac{1}{x^2}+2.
\end{align*}
\item Ved at sætte brøkerne på fælles nævner fås
\begin{align*}
\frac{7a +b}{4a^2-4b^2}-\frac{3}{4a+4b}-\frac{3}{4a-4b}&=\frac{7a+b-3(a-b)-3(a+b)}{4(a-b)(a+b)}\\
&=\frac{a+b}{4(a-b)(a+b)}\\
&=\frac{1}{4a-4b}.
\end{align*}
\item \label{it:4ans} Lader vi $d=b+c$ får vi
\begin{align*}
(a+b+c)^2= a^2+d^2+2ad&= a^2+(b+c)^2+2a(b+c)\\&=a^2+b^2+c^2+2bc+2ab+2ac.
\end{align*}
\item \label{it:ex13ans} Dividerer vi med $a$ får vi
\begin{align*}
x^2+\frac{b}{a}x+\frac{c}{a}=0.
\end{align*}
Hvis vi skal omskrive dette så vi får en parentes $(x+k)^2$ på venstresiden må $\frac{b}{a}x$ være det dobbelte produkt hvilket betyder, at
\begin{align*}
k=\frac{b}{2a}.
\end{align*}
For at samle parentesen skal vi så lægge $k^2$ til på begge sider, hvilket giver
\begin{align*}
x^2+2\frac{b}{2a} x+\frac{b^2}{4a^2}+\frac{c}{a}=\frac{b^2}{4a^2}.
\end{align*}
Samler vi parentesen og reducerer får vi
\begin{align*}
\Big( x+\frac{b}{2a}\Big)^2=\frac{b^2-4ac}{4a^2},
\end{align*}
hvilket medfører at $d=b^2-4ac$.

% \begin{figure}
% \centering
% \begin{tikzpicture}
% \draw (0,0)--(5,0)--(5,5)--(0,5)-- cycle;
% \draw[dashed] (0,4)--(5,4);
% \draw[dashed] (4,0)--(4,5);
% \node at (2,0) [label=below: $a$] {};
% \node at (4.5,0) [label=below:$b$] {};
% \node at (0,2) [label=left: $a$] {};
% \node at (0,4.5) [label=left:$b$] {};
% \node at (2,5) [label=above: $a$] {};
% \node at (4.5,5) [label=above:$b$] {};
% \node at (5,2) [label=right: $a$] {};
% \node at (5,4.5) [label=right:$b$] {};
% \end{tikzpicture}
% \caption{Opgave~\ref{it:1}}
% \label{fig:1ans}
% \end{figure}
% %
% \begin{figure}
% \centering
% \begin{tikzpicture}
% \draw (0,0)--(0,4)--(4,4)--(4,0)--cycle;
% \draw[pattern=north east lines] (0,0)--(3,0)--(3,1)--(0,4)--cycle;
% \draw[fill=gray] (3,1)--(0,4)--(4,4)--(4,1)--cycle;
% \node at (1.5,0) [label=below: $a-b$] {};
% \node at (3.5,0) [label=below:$b$] {};
% \node at (0,2) [label=left: $a$] {};
% \node at (2,4) [label=above: $a$] {};
% \node at (4,2.5) [label=right: $a-b$] {};
% \node at (4,0.5) [label=right:$b$] {};
% %%%Newfig
% \draw (8,0)--(11,0)--(11,5)--(8,5)--cycle;
% \draw[pattern=north east lines] (8,0)--(8,4)--(11,1)--(11,0)--cycle;
% \draw[fill=gray] (8,4)--(8,5)--(11,5)--(11,1)-- cycle;
% \node at (9.5,0) [label=below: $a-b$] {};
% \node at (8,2) [label= left: $a$] {};
% \node at (8,4.5) [label= left: $b$] {};
% \node at (9.5,5) [label= above: $a-b$] {};
% \node at (11,3) [label= right: $a$] {};
% \node at (11,0.5) [label= right: $b$] {};
% \end{tikzpicture}
% \caption{Opgave~\ref{it:2}}
% \label{fig:2ans}
% \end{figure}
% \begin{figure}
% \centering
% \begin{tikzpicture}[auto]
% \draw (0,0)--(0,5)--(5,5)--(5,0)--cycle;
% \draw (0,3)-- node {$c$} (2,0) ;
% \draw (0,3)--node {$c$} (3,5);
% \draw (3,5)--node {$c$} (5,2);
% \draw (2,0)-- node {$c$} (5,2);
% \node at (1,0) [label=below: $a$] {};
% \node at (3.5,0) [label=below: $b$] {};
% \node at (0,1.5) [label=left: $b$] {};
% \node at (0, 4) [label=left: $a$] {};
% \node at (1.5,5) [label=above: $b$] {};
% \node at (4,5) [label=above: $a$] {};
% \node at (5,1)  [label= right: $a$] {};
% \node at (5,3.5) [label=right: $b$] {};
% \end{tikzpicture}
% \caption{Opgave~\ref{it:3}}
% \label{fig:3ans}
% \end{figure}
\end{enumerate}
\section{Potenser}
\begin{enumerate}
\item Svarene er:
\begin{align*}
1,&&0,&& 16,&&125,&& 81,&& 36.
\end{align*}
\item Svarene er:
\begin{align*}
-27,&& \frac{1}{36},&& \frac{1}{8},&& 3,&& 16,&& \frac{1}{1000}.
\end{align*}
\item Svarene er:
\begin{align*}
5,&& 16,&& 81,&& -1,&&27.
\end{align*}
\item Svarene er:
\begin{enumerate}
\item $a\geq 1$.
\item $a\leq 1$.
\item $x^a$ aftager.
\item $x^a$ vokser.
\end{enumerate}
\item Svarene er:
\begin{enumerate}
\item $a\leq 1$.
\item $a\geq 1$.
\item $x^a$ vokser.
\item $x^a$ aftager.
\end{enumerate}
 
\item Svarene er:
\begin{align*}
4x^4,&& x^{-6}y^{-9},&& 9x^6,&& x.
\end{align*}
\item Svarene er:
\begin{align*}
2^2 3^2,&& 3^{-2}2^6,&& 2^2 3^2,&& 2^2 3^{-2},&& 2^4 3^{-4}
\end{align*}
\item Svarene er:
\begin{align*}
3a^{-1}b^2,&&\frac{3}{2}a^{-1}b^{-6},&& 8x^{-2}y^7.
\end{align*}

\item Ved at bruge hintet får vi
\begin{align*}
(a+b)^4=((a+b)^2)^2=(a^2+b^2+2ab)^2 og
\end{align*}
fra Opgave~\ref{it:4} får vi så at
\begin{align*}
(a^2+b^2+2ab)^2&= (a^2)^2+(b^2)^2+(2ab)^2+2a^2b^2+2a^2(2ab)+2b^2(2ab)\\
&=a^4+b^4+4a^2b^2+2a^2b^2+4a^3b+4ab^3\\
&=a^4+4a^3b+6a^2b^2+4ab^3+b^4.
\end{align*}
\item Svarene er:
\begin{align*}
x^{11}y^{4}z^{-4},&& x^{-3}y^{-7}z^{-5},&& x^{10}z^4.
\end{align*}
\item Svarene er:
\begin{align*}
\Big( \frac{1}{2}\Big)^{12},&& \Big(\frac{1}{2}\Big)^{-8}3^{-8},&& \Big(\frac{1}{2}\Big)^{8}3^{-2}.
\end{align*}


\end{enumerate}

\newpage
\section{Math101 answers}
\begin{enumerate}
	\item The answers are:
	\begin{align*}
	27,&& \frac{1}{2}, &&-1,&&8,&&1.
	\end{align*}
	\item The answers are:
	\begin{align*}
	7,&& 2,&& -2,&& 3,&&0.
	\end{align*}
	
	\item The answers are:
	\begin{align*}
	\sqrt{2},&& \frac{3\sqrt{3}}{2},&& \frac{2}{\sqrt{3}}.
	\end{align*}
	

	\item The answers are:
	\begin{align*}
	3,&&1,&& 3
	\end{align*}
	
	\item The answers are:
	\begin{align*}
	-\frac{\sqrt{2}}{2},&& -\frac{\sqrt{3}}{2},&&-1,&& -\frac{1}{2}.
	\end{align*}
	
	
	\item The answers are:
	\begin{align*}
	\frac{3}{2}\ln(2),&& 2,&& \frac{1}{2}.
	\end{align*}
	
	
	\item The answers are:
	\begin{align*}
	1,&&3e,&& \frac{1}{7},&& \frac{7}{9},&& \frac{1}{9}.
	\end{align*}
	
	\item The answers are:
	\begin{align*}
	\frac{1}{2},&& 0,&& -1,&& 0.
	\end{align*}
	
	
	\item The answers are: 
	\begin{align*}
	x=\ln(3),&& x=e^4,&& x=18,&& x=3.
	\end{align*}
	

	\item he answers could be:
	\begin{align*}
	x=\frac{\pi}{4},x=\frac{3\pi}{4},&& x=\frac{\pi}{6},x=-\frac{\pi}{6},&& x=\frac{2\pi}{3},x=\frac{4\pi}{3}.
	\end{align*}
	Note that there are infinitely many correct answers.

	\item \label{it:trig3} The answers could be:

\begin{enumerate}
	\item The triangle in Figure~\ref{fig:trig3} is equilateral since all angles are $\frac{\pi}{3}$. Since two sides of the triangle has length $1$ it follows that the last side must be of length $1$. Hence it follows that $\sin(\frac{\pi}{6})$, which is half of the dotted line, must be $\frac{1}{2}$.
	
	\item The Pythagorean trigonometric identity gives that $\sin^2 \frac{\pi}{6}+\cos^2\frac{\pi}{6}=1$ and solving for $\cos(\frac{\pi}{6})$ gives that $ \cos(\frac{\pi}{6})=\sqrt{1-\frac{1}{4}}=\frac{\sqrt{3}}{2} $.
	
	\item Using the hint we obtain that
	\begin{align*}
	\sin(\frac{\pi}{3})=\sin(2\frac{\pi}{6}) =2\sin(\frac{\pi}{6})\cos(\frac{\pi}{6})=2\frac{1}{2}\frac{\sqrt{3}}{2}=\frac{\sqrt{3}}{2}.
	\end{align*}
	
	\item It follows that
	\begin{align*}
	\sin^2 \frac{\pi}{3}+ \cos^2\frac{\pi}{3}=1\quad \Leftrightarrow\quad \cos^2\frac{\pi}{3}=1-\frac{3}{4}\quad \Leftrightarrow\quad \cos\frac{\pi}{3}=\sqrt{\frac{1}{4}}=\frac{1}{2}.
	\end{align*}
	
\end{enumerate}

\begin{figure}
	\centering
	\begin{tikzpicture}
	\begin{axis}[xmin=-1,xmax=1,ymin=-1,ymax=1,axis x line=center,
	axis y line=center, axis equal]
	\addplot[blue,domain=0:2*pi,thick, samples=100] ({cos(deg(x))},{sin(deg(x))});
	\addplot[domain=0:sqrt(3)/2,thick] {1/sqrt(3)*x};
	\addplot[domain=0:pi/6,thick,samples=100] ({0.2*cos(deg(x))},{0.2*sin(deg(x))}) node[label={[label distance=2pt]0.5:\small$\frac{\pi}{6}$},pos=1] {};
	\addplot[domain=0:sqrt(3)/2,thick,gray,dotted] {-1/sqrt(3)*x};
	\addplot[domain=-pi/6:0,thick,samples=100,gray,dotted] ({0.2*cos(deg(x))},{0.2*sin(deg(x))}) node[label={[label distance=2pt]0.5:\small$\frac{\pi}{6}$},pos=0] {};
	\addplot[dotted,gray,thick] coordinates {(sqrt(3)/2, -1/2) (sqrt(3)/2, 1/2)};
	\end{axis}
	\end{tikzpicture}
	\caption{Exercise~\ref{it:trig3}}
	\label{fig:trig3}
\end{figure}

\item \label{it:trig4} The answers could be:
\begin{enumerate}
	\item The triangle in Figure~\ref{fig:trig4} is a right triangle where each leg has length $1$. Using the Pythagorean theorem it follows that the hypotenuse must have length $\sqrt{1+1}=\sqrt{2}$. singe $\sin \frac{\pi}{4}$ is half the length of the hypotenuse we have that $\sin\frac{\pi}{4}=\frac{\sqrt{2}}{2}$. 
	\item It follows that
	\begin{align*}
	\cos \frac{\pi}{4}=\sqrt{1-\frac{1}{2}}=\frac{\sqrt{2}}{2}.
	\end{align*}
\end{enumerate}

\begin{figure}
	\centering
	\begin{tikzpicture}
	\begin{axis}[xmin=-1,xmax=1,ymin=-1,ymax=1,axis x line=center,
	axis y line=center, axis equal]
	\addplot[blue,domain=0:2*pi,thick, samples=100] ({cos(deg(x))},{sin(deg(x))});
	\addplot[domain=0:sqrt(2)/2,thick] {1*x};
	\addplot[domain=0:pi/4,thick,samples=100] ({0.2*cos(deg(x))},{0.2*sin(deg(x))}) node[label={[label distance=2pt]0.5:\small$\frac{\pi}{4}$},pos=1] {};
	\addplot[domain=0:sqrt(2)/2,thick,gray,dotted] {-1*x};
	\addplot[domain=-pi/4:0,thick,samples=100,gray,dotted] ({0.2*cos(deg(x))},{0.2*sin(deg(x))}) node[label={[label distance=2pt]0.5:\small$\frac{\pi}{4}$},pos=0] {};
	\addplot[dotted,gray,thick] coordinates {({sqrt(2)/2}, -{sqrt(2)/2}) ({sqrt(2)/2}, {sqrt(2)/2})};
	\end{axis}
	\end{tikzpicture}
	\caption{Exercise~\ref{it:trig4}}
	\label{fig:trig4}
\end{figure}

\end{enumerate}
\newpage
\section{Math101 answers}
\begin{enumerate}
	\item The answers are:
	\begin{align*}
	f_1'(x)=2,&& f_2'(x)=1-\sin(x),&& f_3'(x)=e^x,&&f_4'(x)=\frac{1}{2}x+\frac{1}{x}.
	\end{align*}
	
	
	\item The answers are $f'(0)=-1$ and $f'(1)=4$.

	\item The answers are:
	\begin{align*}
	f_1'(x)=3x^2,&& f_2'(x)=\frac{1}{2}x^{-\frac{1}{2}},&& f_3'(x)=-x^{-2},&&f_4'(x)=-2x^{-3},&&f_5(x)=-\frac{1}{2}x^{-\frac{3}{2}}.
	\end{align*}
	\item  The answers are:
\begin{enumerate}
	\item The first blue graph corresponds to the third red graph.
	\item The second blue graph corresponds to the second red graph.
	\item The third blue graph corresponds to the first red graph.
	\end{enumerate}
	
	\item The answers are:
	\begin{align*}
	f'(x)=6e^{2x}-\frac{1}{2x},&& g'(x)=\frac{1}{2}\cos x,&& h'(x)=\frac{1}{x}-\frac{1}{2}e^{-\frac{1}{6}x}.
	\end{align*}
	
	\item The answers are:
	\begin{align*}
	f'(x)=7x^6-8x^3-6x,&&g'(x)=-5x^4+6x^{\frac{1}{2}}+2x^{-3},&&h'(x)=\frac{1}{2}x^{-\frac{1}{2}}-2x^{-2}.
	\end{align*}
	
	
	\item The answers are:
	\begin{align*}
	x=0,&& x=0,x=6. 
	\end{align*}
	
	\item The answers are:
	\begin{align*}
	f'(x)=x^{-\frac{2}{3}},&& f'(x)=9x^2+8x.
	\end{align*}
		

	\item The answers are:	
	\begin{align*}
		f'(x)=-\frac{1}{2}x^{-\frac{3}{2}}-x^{-2},&& f'(x)=\frac{15}{4}x^{\frac{11}{4}},&&f(x)=-\frac{2}{x}.
	\end{align*}
	
	

	
	\item The answers are:
	\begin{enumerate}
		\item The first blue graph corresponds to the second red graph.
		\item The second blue graph corresponds to the third red graph.
		\item The third blue graph corresponds to the first red graph.
	\end{enumerate}

	
	\item The answers are:
	\begin{align*}
	f'(x)=\frac{-5}{x},&& f'(x)=3e^{3x}.
	\end{align*}
	
	\end{enumerate}
\newpage
\section{Andengradsligninger og flere ligninger med flere ubekendte}
 \begin{enumerate}
\item Svarene er:
\begin{align*}
x^2-x-2=0,&& x^2-\frac{7}{2}x+\frac{3}{2}=0,&& x^2-2 ,&& x^2-x-1=0.
\end{align*}

\item Svarene er:
\begin{align*}
x=\pm 6,&&x=1,x=-2,&& x=-1,x=-\frac{1}{2},&& x=-1,x=-3.
\end{align*}

\item Svarene er:
\begin{align*}
x=0,y=4
\end{align*}
\item Svarene er:
\begin{align*}
x=0,x=\frac{3}{2},&& x=0, x=7,&& x=0,x=\frac{2}{5},&& x=\pm \frac{7}{11}.
\end{align*}

\item Svarene er:
\begin{align*}
x=3,y=3
\end{align*}

\item Svarene er:
\begin{align*}
\frac{x+6}{x+1},&& x-1&&\frac{x+1}{x-2}.
\end{align*}


\item Svarene er:
\begin{align*}
x=1,x=-4,&&x=-1,x=4,&&  x=\frac{7\pm \sqrt{39}}{5},&& x=\pm 5.
\end{align*}



\item Svarene er:
\begin{enumerate}
\item I en afstand af $\frac{3}{10}m$ fra den ene ende af stangen.
\item I en afstand af $\frac{7-\sqrt{23}}{20}m$ fra den ene ende af stangen.
\end{enumerate}

\item Svarene er:
\begin{alignat*}{5}
x=6,y=6,z=6.
\end{alignat*}


\item Svarene er: 
\begin{align*}
x=\pm 3, && x=-2,x=-\frac{2}{3},&& x=\pm \frac{1}{6}.
\end{align*}

\item Svarene er:
\begin{align*}
 x=-1,x=\frac{3}{4},&& x=-7,x=10,&&x=3.
\end{align*}

\item Svaret er $b=\pm 4$.

\item Svaret er $a>4$.

\item Svarene er:
\begin{align*}
x=\pm 2,&& x=\pm 3,&& x=2,x=-1.
\end{align*}



\item \label{it:2polyans} Polynomierne skærer hinanden i $(-1,2)$ og $(3,1))$.
%\begin{figure}
%\centering
%\begin{tikzpicture}
%\begin{axis}[xmin=-2,xmax=4,ymin=0,ymax=5,axis x line=center,
%  axis y line=center, ticks=none]
%\addplot[thick,blue,samples =100] {1/8*x*x-1/2*x+11/8};
%\addplot[thick,red,samples=100] {-3/4*x*x+5/4*x+4};
%\node[fill, circle, inner sep=1pt] at (axis cs:-1,2) {};
%\node[fill, circle, inner sep=1pt] at (axis cs:3,1) {};
%\end{axis}
%\end{tikzpicture}
%\caption{Opgave~\ref{it:2poly}}
%\label{fig:2poly}
%\end{figure}

\item Svarene er:
\begin{align*}
x=7,y=8,z=9,w=10.
\end{align*}

\item \label{it:phians} Svaret er
\begin{align*}
\phi=\frac{1+\sqrt{5}}{2}.
\end{align*}

%
%\begin{figure}
%	\centering
%	\begin{tikzpicture}
%	\draw (0,0)-- node[above] {$b$} (3,0) -- node[above] {$a$} ({3+3*(1+sqrt(5))/2},0);
%	\node[fill, inner sep =1pt,circle] at (3,0) {};
%	\end{tikzpicture}
%	\caption{Opgave~\ref{it:phi}}
%	\label{fig:phi}
%\end{figure}

\item Svarene er:
\begin{enumerate}
	\item $A(4)=\frac{1}{2^4}=\frac{1}{16}$
	\item Ved at løse de to ligninger med to ubekendte
	\begin{align*}
	ab&=\frac{1}{16}\\
	\frac{b}{a}&=\frac{\frac{a}{2}}{b},
	\end{align*}
	fås $ a=\frac{\sqrt[4]{2}}{4} $ og $ b= \frac{\sqrt[4]{2^3}}{8}$
	\item 	Ved at isolere $b$ i den anden af de to ligninger med to ubekendte
	\begin{align*}
	ab&=\frac{1}{2^n}\\
	\frac{b}{a}&=\frac{\frac{a}{2}}{b},
	\end{align*}
	får vi at $b=\frac{a}{\sqrt{2}}$. Indsættes dette i den første ligning får vi at 
	\begin{align*}
	\frac{a^2}{\sqrt{2}}=2^{-n}.
	\end{align*}
	Ganger vi igennem med $ \sqrt{2} $ og anvender regneregler for rødder og potenser får vi at
	\begin{align*}
	a=2^{\frac{-2n+1}{4}}
	\end{align*}
	og indsætter vi dette i formlen $b=\frac{a}{\sqrt{2}}$ får vi
	\begin{align*}
	b=2^{\frac{-2n-1}{4}}.
	\end{align*}
\end{enumerate}

%\begin{figure}
%	\centering
%	\begin{tikzpicture}
%	\draw (0,0)-- node[left] {$b$} (0,3)-- node[above] {$a$} ({sqrt(18)},3)--({sqrt(18)},0)--cycle;
%	\draw[dashed] ({sqrt(9/2)},3)--({sqrt(9/2)},0);
%	\node at (({sqrt(9/2)}/2,0) [label= below: $\frac{a}{2}$] {};
%	\end{tikzpicture}
%	\caption{A papirformat}
%	\label{fig:2deglig1}
%\end{figure}

\item Fra Opgave~\ref{it:ex13} har vi allerede at
\begin{align*}
\Big( x+\frac{b}{2a}\Big)^2=\frac{b^2-4ac}{4a^2},
\end{align*}
og tager vi kvadratroden på begge sider fås
\begin{align*}
x+\frac{b}{2a}=\pm\frac{\sqrt{b^2-4ac}}{2a}.
\end{align*}
Hvis vi trækker $ \frac{b}{2a} $ fra på begge sider får vi den velkendte formel
\begin{align*}
x=\frac{-b\pm \sqrt{b^2-4ac}}{2a}.
\end{align*}

\end{enumerate}

\newpage
\section{Funktioner}
\begin{enumerate}
	\item  Svarene kunne være:
	\begin{enumerate}
		\item En mulighed er $ f(1)=3,f(2)=5,f(3)=\pi,f(4)=1,f(5)=-1 $, men der er 119 andre rigtige svar..
		\item En mulighed kunne være $f(x)=\pi$ for alle $x\in A$.
	\end{enumerate}
	
	
	\item En cirkel kan ikke beskrives som grafen for en funktion da denne funktion i såfald ville have $x$-værdier som skulle sendes i to $y$-værdier. Man kan dog beskrive enhver cirkel ved hjælp at to funktioner, en øvre og nedre halvcirkel, givet ved $y=b\pm \sqrt{r^2-(x-a)^2}$. 
	
	\item Svarene er:
	\begin{enumerate}
		\item $f$ er en funktion som er surjektiv men ikke injektiv.
		\item $g$ er en funktion som er injektiv men ikke surjektiv.
		\item $h$ er ikke en funktion da $h(4)=\textup{Hest},\textup{Hund}$.
	\end{enumerate}

	\item Svarene er:
	\begin{enumerate}
		\item $(f+g)(2)=\frac{10}{3}$
		\item $ \frac{f}{g}(-2) =-3$
		\item $(fg)(0)=-1$
		\item $\frac{g}{f}(x)=\frac{1}{x^3+x^2-x-1}$
		\item $(g-f)(x)=\frac{-x^3-x^2+x+2}{1+x}$.
	\end{enumerate}

	\item Værdimængden for funktionen $f\colon \mathbb{R}\to \mathbb{R}$ givet ved $f(x)=-3x^2+9$ er $f(\mathbb{R})=]-\infty,9]$.
	
	\item \label{it:fun1} Svarene er:
	\begin{enumerate}
		\item Blå: Surjektiv men ikke injektiv.
		\item Rød: Hverken injektiv eller surjektiv.
		\item Grøn: Injektiv og surjektiv.
		\item Sort: Injektiv men ikke surjektiv.
	\end{enumerate}
%	\begin{figure}
%		\centering
%	\begin{tikzpicture}
%\begin{axis}[xmin=-2.1,xmax=2.1,ymin=-2.1,ymax=2.1,axis x line=center,
%axis y line=center, restrict y to domain=-2:2]
%\addplot[thick,blue,samples =300] {-x^2+2};
%\addplot[thick,red,samples=100] {(1+x^2)^-1};
%\addplot[thick,green, samples =200] {x };
%\addplot[thick, black,samples=200] { e^-x -3/2};
%\end{axis}
%	\end{tikzpicture}
%	\caption{Opgave~\ref{it:fun1}.}
%	\label{fig:fun1}
%	\end{figure}

	\item Så længe $D\subset [0,\infty[$ eller $D\subset ]-\infty,0]$ vil $f$ være injektiv.
	
	\item \label{it:fun2} Funktionen $f$ er injektiv da 
	\begin{align*}
	\frac{1}{x}=\frac{1}{y}\quad\Leftrightarrow\quad y=x.
	\end{align*}
	Den er ikke surjektiv da $f(x)\neq 0$ for alle $x\in \R\neq \{0\}$, dog er alle andre punkter med i værdimængden for $f$. Derfor bliver $f$ bijektiv hvis domænet ændres til $\R\setminus\{0\}$.
	
%	\begin{figure}
%		\centering
%		\begin{tikzpicture}
%		\begin{axis}[xmin=-5,xmax=5,ymin=-20,ymax=20,axis x line=center,
%		axis y line=center, restrict y to domain=-50:50]
%		\addplot[thick,blue,samples=400] {1/x};
%		\end{axis}
%		\end{tikzpicture}
%		\caption{Opgave~\ref{it:fun2}.}
%		\label{fig:fun2}
%	\end{figure}

	\item Når $a$ er ulige er $f$ bijektiv og når $a$ er lige er $f$ hverken injektiv eller surjektiv.
	
	\item Svarene er:
	\begin{align*}
	D(f)=[-1,2],&& D(g)=\R,&& D(h)=]-2,\infty[.
	\end{align*}
	
	\item \label{it:fun3} Kun den røde kurve i Figur~\ref{fig:fun3} er grafen for en funktion.
	
 	% \begin{figure}
	% 	\centering
	% 	\begin{tikzpicture}[
	% 	%declare function={func(\x)= (\x<-3)*(-2)+ and (\x>=-3,\x<=-1)*(sqrt(1-(x+2)^2)-2)+ and (\x>-1)*(-2); }
	% 	declare function={
	% 		func(\x)= (\x < -3) * (-2)   +
	% 		and(x >= -3,\x<=-1) * (sqrt(1-(x+2)^2)-2)     +
	% 		and(x >= -1,\x<=1) * (-sqrt(1-(x)^2)-2)     +
	% 		and(x >= 1,\x<=3) * (sqrt(1-(x-2)^2)-2)     +
	% 		(\x>3) * (-2)
	% 		;
	% 		}
	% 	]
	% 	\begin{axis}[xmin=-5,xmax=5,ymin=-5,ymax=5,axis x line=center,
	% axis y line=center, restrict y to domain =-5:5]
	% 	\addplot[data cs=polar, thick,blue,samples=200, domain= 0:180] ({x}, {3*sin(x)*cos(x)/(sin(x)^3+cos(x)^3)});
	% 	\addplot[thick,red,samples=400] {func(x)};
	% 	\addplot[data cs=polar, thick,green,samples=200, domain= 0:180] ({x}, {sec(x)+2*cos(x)});
	% 	\end{axis}
	% 	\end{tikzpicture}
	% 	\caption{Opgave~\ref{it:fun3}.}
	% 	\label{fig:fun3}
	% \end{figure}

	\item\label{it:exercisetegning} I Figur~\ref{fig:fun35} ses et eksempel på en sådan funktion.
	\begin{figure}
		\centering
		\begin{tikzpicture}
		\begin{axis}[xmin=-2,xmax=4,ymin=-2.1,ymax=4,axis x line=center,
		axis y line=center, restrict y to domain =-5:5]
		\addplot[blue,thick,domain=-2:-1] {3};
		\addplot[blue,thick,domain=-1:4] {-2};
	\node[fill, circle, inner sep=1pt] at (axis cs:-1,3) {};
	\node[circle,inner sep=1pt,draw=black,fill=white] at (axis cs:-1,-2) {};
		\end{axis}
		\end{tikzpicture}
		\caption{Opgave~\ref{it:exercisetegning}.}
		\label{fig:fun35}
	\end{figure}
	
	
	\item Uanset værdien af $a\in \{-3-2,\dots,2,3\}$ vil kurven aldrig være grafen for en funktion.
\end{enumerate}
\section{Delvis integration og integration ved substitution}

\begin{enumerate}
	\item \label{it:int21ans} Svarene er:
	\begin{align*}
	x\sin(x)+\cos(x)+c,&& \frac{1}{2}x^2\ln(x)-\frac{1}{4}x^2+c,&& (x-1)e^x+c
	\end{align*}
	
	\item Svarene er:
	\begin{align*}
	\frac{1}{2}\sin(2x)+c,&& -\frac{(1-x)^3}{3}+c,&& \frac{1}{2} e^{2x-3}+c.
	\end{align*}
	
	\item Svarene er:
	\begin{align*}
	0,&& 2e^{-1}-1,&& 2\ln(2)-\frac{3}{4}.
	\end{align*}
	
	\item Svarene er:
	\begin{align*}
	-\frac{1}{6},&& \frac{e^4-1}{2},&& \ln(5).
	\end{align*}
	
	\item Svarene er:
	\begin{align*}
	-(x+1)\cos(x)+\sin(x)+c,&& 2e^{6}.
	\end{align*}
	
	\item Svarene er:
	\begin{align*}
	\frac{\sqrt{2}}{6},&& \frac{1}{6}.
	\end{align*}
	
	\item Svarene er:
	\begin{align*}
	e^x(x^2-2x+2) +c,&& \pi^2-4.
	\end{align*}
	
	\item Svarene er:
	\begin{align*}
	\sqrt{2x-1}+c,&& \frac{\sqrt{2}}{2}
	\end{align*}
	
	\item Svaret er:
	\begin{align*}
	0.
	\end{align*}
\end{enumerate}
\section{Repetition}
\begin{enumerate}
	\item Udregn
	\begin{align*}
	2+\frac{4}{2}\cdot 3,&& 1-\frac{3}{2}\cdot 4,&& \frac{1}{3}+\frac{2}{5},&& \frac{2}{3}+1.
	\end{align*}
	
	\item Udregn
	\begin{align*}
	\frac{3^3\cdot 3^{-2}}{3^4}\cdot \frac{(-3)}{3^{-2}},&& (-3\sqrt{5})^2,&&\frac{\frac{2}{3}+\frac{3}{4}}{\frac{5}{2}},&& \sqrt{7+(3\sqrt{2})^2}.
	\end{align*}
	
	\item Udregn
	\begin{align*}
	\sin(\frac{\pi}{6})+\cos(\frac{3\pi}{4})-\cos(-\pi),&& \log(20)+\log(50),&& \ln(e^3)+\ln(1).
	\end{align*}
	
	\item Reducer udrtykkene
	\begin{align*}
	(x+5)^2,&& (1-2x)^2,&& (2y-1)(2y+1),&& (x^2+y^2)-(2x^2-y^2).
	\end{align*}
	
	\item Reducer udtrykkene
	\begin{align*}
	\frac{x^2+y^2-2xy}{x^2-xy},&& \frac{x^2-y^2}{x^2-xy},&& \frac{y^2-x^2}{x+y}+x.
	\end{align*}
	
	\item Løs ligningerne
	\begin{align*}
	-2x+3=7,&& \frac{2}{3}x-3=\frac{6}{5},&&x^2-2x-3=0.
	\end{align*}
	
	\item Løs ligningerne
	\begin{align*}
	\frac{2}{x}+2x=3x,&& -2x^2+x+1=0.
	\end{align*}
	
	\item Bestem mindst en løsning til ligningerne
	\begin{align*}
	e^{2x-1}-1=0,&& \sin(x-\pi)=\frac{\sqrt{3}}{2},&& \ln(x-1)=\ln(12)-\ln(4).
	\end{align*}
	
	\item For hvilke værdier af $a$ er følgende udtryk sande
	\begin{align*}
	\frac{1}{1+a}=\frac{\sqrt{5}}{\sqrt{5}+a},&& \frac{1}{1+a}=\frac{1-a}{1-a^2},\\
	\frac{1}{1+a}=\frac{1+a}{a^2+2a+1},&& \frac{1}{1+a}=1+\frac{1}{a}.
	\end{align*}
	
	\item Differentier funktionerne
	\begin{align*}
	f(x)=2x^3-x^2+1,&& g(x)=2x^{-2}+x,&& h(x)=\frac{2}{x}+x.
	\end{align*}
	
	\item Bestem følgende integraler
	\begin{align*}
	\int 2x^2+1 \, dx,&& \int_0^1 x^2-3x+1\, dx,&& \int_0^4 \frac{1}{\sqrt{x}}+x \, dx.
	\end{align*}
	
	
	\item Differentier funktionerne
	\begin{align*}
	f(x)&=2x^3-x^{-2}+4x^{-1}-1,\\ g(x)&=3xe^3-\sqrt{x-1},\\ h(x)&=2xe^x-\sin(x^2-x).
	\end{align*}
	
	\item Udregn integralerne
	\begin{align*}
	\int 2e^{-x} \, dx,&& \int_0^1 4xe^{x^2}\, dx, && \int_0^2 \frac{2x}{\sqrt{x^2+5}}\, dx
	\end{align*}
\end{enumerate}
\end{document}
