\section{Potenser}
\noindent Når vi tænker på potenser, tænker vi på tal på formen
\begin{align*}
\textup{potens} = \textup{grundtal}^{\textup{eksponent}},
\end{align*}
hvor både grundtallet og eksponenten kan være alle tal, dog med den undtagelse at grundtallet og eksponenten ikke må være lig $0$ på samme tid.

Hvis eksponenten er et positivt heltal, så som $1,2,3,\ldots$ osv., så udregner man potensen ved at gange grundtallet med sig selv det antal gange der står i eksponenten. Det kan skrives matematisk som
\begin{align*}
x^n = \underbrace{x \cdot x \cdot \ldots \cdot x}_{\textup{n gange}}.
\end{align*}
En potens med negativ eksponent er det det samme som en brøk hvor nævneren er den samme potens men med positiv eksponent og tælleren er $1$:
\begin{align*}
x^{-a} = \frac{1}{x^a},
\end{align*} 
hvor $a$ kan være alle tal.

Hvis en potens har eksponent $0$, så definerer vi potensen til at være lig med $1$, altså er
\begin{align*}
x^0=1,
\end{align*}
for alle tal $x$ bortset fra $0$.
\paragraph{Eksempler:}
\begin{enumerate}
\item Udregn $3^4$:
\begin{align*}
3^4=3\cdot 3 \cdot 3 \cdot 3 = 81. 
\end{align*}
\item Udregn $9871^0$:

Da eksponenten er lig med nul, har vi pr. definition at
\begin{align*}
9871^0=1.
\end{align*}
\item Udregn $0^4$:
\begin{align*}
0^4 = 0 \cdot 0 \cdot 0 \cdot 0 = 0.
\end{align*}
\end{enumerate}
\paragraph*{Regneregler:}
For potenser har vi følgende regneregler, som vi vil benytte igen og igen i de resterende kursusgange og i vil se dem i gentagende gange i jeres videre studieforløb. Det er derfor en god idé at øve sig på disse.

\begin{enumerate}
\item At gange to potenser med samme grundtal er det samme som grundtallet opløftet i summen af de to eksponenter:
\begin{align*}
x^a \cdot x^b = x^{a+b}.
\end{align*}
\item At dividere to potenser med samme grundtal er det samme som at opløfte grundtallet i forskellen af de to eksponenter:
\begin{align*}
\frac{x^a}{x^b}=x^{a-b}.
\end{align*}
\item At gange to potenser sammen med samme eksponent er det samme som at gange grundtallene sammen først og derefter opløfte i eksponenten:
\begin{align*}
x^a \cdot y^a = (x\cdot y)^a.
\end{align*}
\item At dividere to potenser med samme eksponent er det samme som at dividere de to grundtal og så opløfte i eksponenten:
\begin{align*}
\frac{x^a}{y^a}= \Big( \frac{x}{y} \Big)^a.
\end{align*}
\item At opløfte en potens i en eksponent er det samme som at op at opløfte grundtallet i de to eksponenter ganget samme:
\begin{align*}
(x^a)^b=x^{a \cdot b}.
\end{align*}
\end{enumerate}

\paragraph{Eksempler:}
\begin{enumerate}
\item Udregn $\frac{(2\cdot3)^2}{2^3}$:
\begin{align*}
\frac{(2\cdot 3)^2}{2^3} = \frac{2^2 \cdot 3^2}{2^3}=\frac{4\cdot 9}{8}=\frac{36}{8}=\frac{9}{2}.
\end{align*}
\item Udregn $\Big(\frac{2^3}{3}\Big)^2$:
\begin{align*}
\Big(\frac{2^3}{3}\Big)^2=\frac{(2^3)^2}{3^2}=\frac{2^{2 \cdot 3}}{3^2}=\frac{2^6}{3^2}=\frac{64}{9}.
\end{align*}
\item Reducer $\frac{(ab)^n}{a^n}$:
\begin{align*}
\frac{(ab)^n}{a^n} = \frac{a^nb^n}{a^n}=b^n.
\end{align*}
\item Reducer $\frac{(ab)^n-a^n}{(ab)^n}$:
\begin{align*}
\frac{(ab)^n-a^n}{(ab)^n} = \frac{a^nb^n-a^n}{a^nb^n}= \frac{a^n(b^n-1)}{a^nb^n}= \frac{b^n-1}{b^n}.
\end{align*}
\end{enumerate}
Da $-x=(-1) \cdot x$ får vi ved at benytte regneregel 3. ovenfor, at 
\begin{align*}
(-x)^n=((-1) \cdot x)^n=(-1)^n \cdot x^n.
\end{align*}
Det betyder at hvis vi opløfter et tal i en lige eksponent får vi et positivt tal og hvis vi opløfter et negativt tal i en ulige eksponent, så får vi et negativt tal.
\paragraph{Eksempel:}
\begin{enumerate}
\item Udregn $(-x)^2$:
\begin{align*}
(-x)^2=((-1) \cdot x)^2 = (-1)^2 \cdot x^2=x^2.
\end{align*}
\end{enumerate} 
























