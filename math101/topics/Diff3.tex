\section{Kædereglen}
\noindent Vi er nu kommet til at studere, hvordan man differentiere sammensatte funktioner. Vi husker, at en sammensat funktion er en funktion på formen
\begin{align*}
(f \circ g)(x) = f(g(x)).
\end{align*}
Eksempler på sammensatte funktioner er
\begin{align*}
\cos(x^3), \quad \sin(\sqrt{x}), \quad \tan\Big(\frac{1}{x}\Big), \quad (\sin x)^2, \quad \sqrt{\tan x}, \quad \frac{1}{\cos x}, \quad \frac{1}{x^2}.
\end{align*}

\paragraph*{Regneregler:}
Vi har følgende regneregel for at differentiere sammensatte funktioner.
\begin{enumerate}
\item $(f \circ g)'(x) = \frac{d}{dx}(f(g(x))) = f'(g(x))g'(x)$.
\end{enumerate}
Denne regel kaldes ofte for kædereglen. Bemærk, at kædereglen siger at hvis vi skal differentiere en sammensat funktion, så gør vi det ved at differentiere den ydre funktion og sætte den indre funktion ind på $x$ plads deri, og så gange den indre funktion differentieret på.

Bemærk, at ligesom ved produkter af funktioner, kan det nogle gange være smart at omskrive en funktion ved hjælp at potensregneregler. F.eks. kan den sammensatte funktion $ \frac{1}{x^2}$ omskrives til
\begin{align*}
\frac{1}{x^2}=x^{-2}.
\end{align*}
\paragraph*{Eksempler:}
\begin{enumerate}
\item Differentier $\cos (x^2)$:

Vi ser, at $f(x)=\cos x$ er den ydre funktion og $g(x)=x^2$ er den indre funktion. Derudover har vi, at $f'(x)=-\sin x$, $f'(g(x))=-\sin(x^2)$ og $g'(x)=2x$. Indsætter vi det i kædereglen får vi, at
\begin{align*}
\frac{d}{dx}\cos(x^2) = -\sin (x^2) 2x = -2x\sin(x^2).
\end{align*}
\item Differentier $e^{x^2+3x}$:

Vi ser, at $f(x)=e^x$ er den ydre funktion og $g(x)=x^2+3x$ er den indre funktion. Derudover har vi, at $f'(x)=e^x$, $f'(g(x))=e^{x^2+3x}$ og $g'(x)=2x+3$. Indsætter vi det i kædereglen får vi, at
\begin{align*}
\frac{d}{dx} e^{x^2+3x}=e^{x^2+3x} (2x+3) = 2xe^{x^2+3x} + 3e^{x^2+3x}.
\end{align*}
\item Differentier $\displaystyle \frac{1}{x^2}$ først ved at bruge kædereglen og dernæst ved at omskrive den ved brug af potensregneregler:

Vi bruger først kædereglen. Vi ser, at $f(x)=\frac{1}{x}$ er den ydre funktion og $g(x)=x^2$ er den indre funktion. Derudover har vi, at $f'(x)=-\frac{1}{x^2}$, $f'(g(x))= -\frac{1}{(x^2)^2}=-\frac{1}{x^4}$ og $g'(x)=2x$. Indsætter vi det i kædereglen får vi, at
\begin{align*}
\frac{d}{dx}\frac{1}{x^2} = -\frac{1}{x^4} \cdot 2x = \frac{-2}{x^3}.
\end{align*}

Dernæst differentierer vi funktionen ved at omskrive den ved hjælp af potensregneregler. Vi husker, at $\frac{1}{x^2}=x^{-2}$, hvilket giver at
\begin{align*}
\frac{d}{dx}\frac{1}{x^2} =\frac{d}{dx} x^{-2}=-2x^{-3}=\frac{-2}{x^3}.
\end{align*}
\item Udregn $(f\circ g)'(2)$ givet at $f'(4)=3$, $g(2)=4$ og $g'(2)=5$:

Vi har fra kædereglen, at
\begin{align*}
(f \circ g)'(x)=\frac{d}{dx}(f(g(x))) = f'(g(x))\cdot g'(x).
\end{align*}
Indsætter vi nu de værdier vi har fået oplyst, får vi
\begin{align*}
(f \circ g)'(2) = f'(g(2))\cdot g'(2)=f'(4)\cdot 5 = 3 \cdot 5 = 15.
\end{align*}
\end{enumerate}

















