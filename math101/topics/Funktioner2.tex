\section{Sammensatte og inverse funktioner}
Sidste gang beskæftigede vi os med funktioner $f \colon X \to Y$, hvor $X$ og $Y$ var mængder. Hvis vi har en anden funktion $g \colon Y \to Z$, hvor $Z$ også er en mængde, så kan man betragte sammensætningen af de to funktioner $g \circ f  \colon X \to Z$ (se Figur~\ref{fig:funktioner2et}) som er bestemt udfra formlen $(g \circ f)(x)=g(f(x))$ (bemærk at $g \circ f$ skal læses som at vi først anvender $f$ og dernæst $g$). Måden vi udregner $g \circ f$ er at indsætte funktionen $f$ på den ubekendte variabels plads  i $g$. I sådan et tilfælde kalder vi $f$ for den indre funktion, $g$ for den ydre funktion og $g \circ f$ for den sammensatte funktion.

\begin{figure}[!htbp]
  \pgfplotsset{width=0.5\textwidth,compat=1.11}
  \centering
  \begin{tikzpicture}
  \draw \boundellipse{0,0}{0.7}{1.4};
  \draw \boundellipse{5,0}{0.7}{1.4};
  \draw \boundellipse{10,0}{0.7}{1.4};
  \node[] at (0.45,1.7) [label=left:$X$]{};
  \node[] at (5.45,1.7) [label=left:$Y$]{}; 
  \node[] at (10.45,1.7) [label=left:$Z$]{}; 
  \node[] at (2.7,1.1) [label=left:$f$]{};
  \node[] at (7.7,1.1) [label=left:$g$]{};
  \node[] at (5.6,3.0) [label=left:$g \circ f$]{};
  \draw[thick,->] (0.5,1.6) arc (170:10:4.5 and 1.0);
  \draw[thick,->] (1.1,0.3) arc (150:30:1.5 and 0.8);
  \draw[thick,->] (6.1,0.3) arc (150:30:1.5 and 0.8);
 \end{tikzpicture}
  \caption{En sammensat funktion}
  \label{fig:funktioner2et}
\end{figure}

\paragraph*{Eksempler:}
\begin{enumerate}
\item Givet den sammensatte funktion $\frac{1}{x^2}$, find funktioner $f$ og $g$ så $f(g(x))=\frac{1}{x^2}$:

Vi genkender funktionerne $\frac{1}{x}$ og $x^2$ og ser at $x^2$ er sat ind på $x$ plads i $\frac{1}{x}$. Det betyder at hvis vi sætter $g(x)=x^2$ og $f(x)=\frac{1}{x}$ så har vi at $f(g(x))=\frac{1}{x^2}$.
\item Lad $f(x) = x^2+3x + \cos(x)$ og $g(x) = \tan(x)$ og bestem forskriften $(f \circ g)(x)$:

Vi indsætter $g(x)$ på $x$ plads i $f(x)$ og får:
\begin{align*}
(f \circ g)(x) = f(g(x))= (\tan (x))^2 + 3\tan(x) + \cos(\tan(x)).
\end{align*} 
\item Lad $f (x) = x^3$ og $g(x) = \sqrt{x}$ og bestem både $(g \circ f)(2)$ og $(f \circ g)(2)$:

Vi har at $f(2)=2^3 = 8$ og $g(2)=\sqrt{2}$, hvilket giver:
\begin{align*}
(g \circ f) (2) = g(f(2)) = g(8)=\sqrt{8}=\sqrt{4 \cdot 2} = \sqrt{4}\sqrt{2} = 2\sqrt{2}, \\
(f \circ g)(2) = f(g(2)) = f(\sqrt{2}) = (\sqrt{2})^3 = \sqrt{2^3} = \sqrt{8} = 2\sqrt{2}.
\end{align*}
\item Generelt kender vi funktioner så som 
\begin{align*}
\cos (x), \quad \sin (x),\quad \tan (x),\quad  x^2,\quad  \sqrt{x},\quad  \frac{1}{x}.
\end{align*} 
Hvis vi på $x$ plads i de forskellige funktioner indsætter en anden funktion, så vil vi få en sammensat funktion, som f.eks.
\begin{align*}
\cos(x^3), \quad  \sin (\sqrt{x}), \quad  \tan \Big( \frac{1}{x} \Big),\quad  (\sin(x))^2, \quad  \sqrt{\tan(x)}, \quad  \frac{1}{\cos(x)}.
\end{align*}
Bemærk, at der kan være forskel på at være den ydre og den indre funktion, f.eks. er $\sin (x^2)$ og $(\sin (x))^2$ ikke altid det samme.
\end{enumerate}
\paragraph*{Inverse funktioner:}
Hvis vi har to funktioner $f \colon X \to Y$ og $g \colon Y \to X$, som i Figur~\ref{fig:funktioner2to}, som opfylder at 
\begin{align*}
f(g(y))=y \qquad \textup{ og } \qquad g(f(x))=x,
\end{align*}
for alle $x \in X$ og $y \in Y$, så siges $f$ at være den inverse funktion til $g$ og ligelede siges $g$ at være den inverse funktion til $f$, hvilket også noteres med $g=f^{-1}$. Det betyder at den inverse funktion er en funktion der sender elementet $f(x)$ tilbage i det element det kom fra og tilsvarende for $g$. 

For at en sådan funktion kan eksistere skal $f$ være bijektiv, altså både injektiv og surjektiv. Hvis den ikke er injektiv så vil der være flere $x$'er der bliver sendt over i det samme $y$ men det betyder, at $f^{-1}(y)$ skal sende $y$ i mere end et punkt, hvilket ikke er muligt for en funktion. Derudover skal $f$ være surjektiv, da $f^{-1}$ skal kunne anvendes på alle elementer i $Y$ og det kan den ikke, medmindre $f$ rammer alle elementer i $Y$. 


\begin{figure}[!htbp]
  \pgfplotsset{width=0.5\textwidth,compat=1.11}
  \centering
  \begin{tikzpicture}
  \draw \boundellipse{0,0}{0.7}{1.4};
  \draw \boundellipse{5,0}{0.7}{1.4};
  \node[] at (0.45,1.7) [label=left:$X$]{};
  \node[] at (5.45,1.7) [label=left:$Y$]{}; 
  \node[] at (2.7,1.1) [label=left:$f$]{};
  \node[] at (2.7,-1.1) [label=left:$f^{-1}$]{};
  \draw[thick,->] (1.1,0.3) arc (150:30:1.5 and 0.8);
  \draw[thick,->] (3.7,-0.3) arc (330:210:1.5 and 0.8);
 \end{tikzpicture}
  \caption{En invers funktion}
  \label{fig:funktioner2to}
\end{figure}

\paragraph*{Eksempler:}
\begin{enumerate}
\item Lad $f \colon [0,\infty) \to [0,\infty)$ og $g(x) \colon [0,\infty) \to [0,\infty)$ være givet ved henholdsvis $f(x)=x^2$ og $g(x)=\sqrt{x}$, så har vi at
\begin{align*}
f(g(x))=(\sqrt{x})^2=x \qquad \textup{ og } \qquad g(f(x)) = \sqrt{x^2} = x,
\end{align*}
så $g=f^{-1}$. Bemærk, at vi ikke kan udvide $f$ og $g$ til hele $\mathbb{R}$, da de så ikke er bijektive.
\item Lad $f \colon \mathbb{R} \to \mathbb{R}$ og $g \colon \mathbb{R} \to \mathbb{R}$ være givet ved henholdsvis $f(x)=3x+2$ og $g(x)=\frac{1}{3}x-\frac{2}{3}$, så har vi at
\begin{align*}
f(g(x))= 3\Big(\frac{1}{3}x - \frac{2}{3}\Big) + 2 = x - 2 + 2 = x \\
g(f(x)) = \frac{1}{3}(3x+2) - \frac{2}{3} = x + \frac{2}{3} - \frac{2}{3} = x ,
\end{align*}
hvilket betyder at $g=f^{-1}$.
\end{enumerate}