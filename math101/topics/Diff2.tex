\section{Differentiation af produkter og kvotienter}
\noindent Sidste gang betragtede vi hvordan man differentiere to funktioner lagt sammen eller en konstant gange en funktion. Denne gang vil vi studere, hvordan man differentierer to funktioner, der er ganget sammen eller divideret med hinanden.

Hvis vi har to funktioner $f$ og $g$ der begge afhænger af $x$ så husker vi, at vi at deres produkt (de to funktioner ganget sammen) og kvotient (de to funktioner divideret) er givet på følgende måde
\begin{align*}
(fg)(x)=f(x) \cdot g(x) \qquad \textup{ og } \qquad \Big(\frac{f}{g}\Big) (x) = \frac{f(x)}{g(x)},
\end{align*}
hvor kvotienten kun giver mening, hvis $g(x) \neq 0$.

Eksempler på produkter og kvotienter af funktioner er
\begin{align*}
xe^x, \quad \frac{1}{x} \cdot \cos x = \frac{\cos x}{x},\quad \sin x \cdot \ln(x),\quad \frac{\ln x}{\sqrt{x}}, \quad \sqrt{x}\cdot x.
\end{align*}
Bemærk, at det sidste eksempel er et produkt af to funktioner, der begge afhænger af $x$, men hvis vi omskriver kvadratroden ved hjælp af vores potensregler får vi
\begin{align*}
\sqrt{x}\cdot x = x^{\frac{1}{2}} \cdot x = x^\frac{3}{2},
\end{align*}
som ikke er et produkt af funktioner. Det kan derfor nogle gange være nemmere, at omskrive et produkt af funktioner før man differentierer.

\paragraph*{Regneregler:}
Vi har følgende regneregler for at differentiere produkter og kvotienter.
\begin{enumerate}
\item $(fg)'(x)=f'(x)g(x)+f(x)g'(x)$.
\item $\displaystyle \Big( \frac{f}{g} \Big)'(x) = \frac{f'(x)g(x)-f(x)g'(x)}{(g(x))^2}$, hvis $g(x) \neq 0$.
\end{enumerate}
Man kan vise (som i kommer til at gøre i en senere opgaveregning) at regnereglen for at differentiere kvotienter, følger ud fra regnereglen om at differentiere produkter.

\paragraph*{Eksempler:}
\begin{enumerate}
\item Differentier $xe^x$:

Vi sætter $f(x)=x$ og $g(x)=e^x$ og får at $f'(x)=1$ og $g'(x)=e^x$. Ved at indsætte det i regneregel $1$. får vi
\begin{align*}
\frac{d}{dx}(xe^x)=1 \cdot e^x + x \cdot e^x = e^x + x e^x = e^x(1+x).
\end{align*}
\item Differentier $\frac{\cos x}{x}$:

Vi sætter $f(x)=\cos x$ og $g(x) = x$, og får at $f'(x) = -\sin x$, $g'(x)=1$ og $(g(x))^2=x^2$. Indsætter vi det i regneregel $2.$ får vi 
\begin{align*}
\frac{d}{dx}\Big(\frac{\cos x}{x}\Big) = \frac{-\sin (x) \cdot x - \cos (x) \cdot 1 }{x^2} = \frac{-(x\sin x + \cos x)}{x^2} = - \frac{x\sin x + \cos x}{x^2}.
\end{align*}
\item Differentier $\sqrt{x} \cdot x$ først ud fra regneregel $1.$ og dernæst ved at omskrive udtrykket ved hjælp af potensregnereglerne:

Vi gør det først ved at bruge regneregel $1.$ Lad $f(x)=\sqrt{x}$ og $g(x)=x$ så har vi at $f'(x)= \frac{1}{2\sqrt{x}}$ og $g'(x)=1$. Indsætter vi nu det i regneregel nummer $1.$ får vi
\begin{align*}
\frac{d}{dx}(\sqrt{x}\cdot x)=\frac{1}{2\sqrt{x}} \cdot x + \sqrt{x} \cdot 1 = \frac{x}{2\sqrt{x}}+\sqrt{x} = \frac{1}{2}\sqrt{x}+\sqrt{x} = \frac{3}{2}\sqrt{x},
\end{align*}
hvor vi i den tredje lighed har brugt at $x=\sqrt{x}\cdot \sqrt{x}$.

Hvis vi omskriver $\sqrt{x}\cdot x$ ved hjælp at potensregneregler, så vi tidligere at 
\begin{align*}
\sqrt{x}\cdot x = x^\frac{3}{2},
\end{align*}
hvilket giver at
\begin{align*}
\frac{d}{dx}(\sqrt{x}\cdot x) = \frac{d}{dx}x^\frac{3}{2}=\frac{3}{2}x^\frac{1}{2}=\frac{3}{2}\sqrt{x}.
\end{align*}
\end{enumerate}