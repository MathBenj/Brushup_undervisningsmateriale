\section{Andengradsligninger og to ligninger med to ubekendte}
Indtil videre har vi for det meste kun betragtet førstegradsligninger, som er ligninger på formen $ax+b=0$, hvor $a$ kan være et hvilket som helst tal med undtagelse af $0$ og $c$ kan være alle tal. Det næste vi skal undersøge er andengradsligninger, som er på formen $ax^2+bx+c=0$, hvor $a$ kan være alle tal bortset fra $0$, og $b,c$ kan være alle tal.

Vi finder løsningerne til en andengradsligning ud fra formlen
\begin{align*}
x = \frac{-b \pm \sqrt{b^2-4ac}}{2a} = \frac{-b \pm \sqrt{d}}{2a},
\end{align*}
hvor $a,b,c$ er de samme tal som indgår i den andengradsligning vi er ved at løse og $d=b^2-4ac$ kaldes diskriminanten. Løsningerne til en andengradsligning kaldes ofte for rødderne af andengradsligningen. 

Vi ser først på antallet af løsninger til en andengradsligning:
\begin{enumerate}
\item Hvis $d>0$ så har andengradsligningen præcis $2$ løsninger, givet ved
\begin{align*}
x_1 = \frac{-b+\sqrt{b^2-4ac}}{2a} \qquad \textup{ og } \qquad x_2=\frac{-b - \sqrt{b^2-4ac}}{2a}.
\end{align*}
\item Hvis $d=0$ så har andengradsligningen præcis $1$ løsning, givet ved
\begin{align*}
x = \frac{-b}{2a}.
\end{align*}
\item Hvis $d<0$ så har andengradsligningen ingen reelle løsninger (I modsætning til hvad mange får at vide i gymnasiet, så har den stadig løsninger, men dem vil I komme til at se i kurset \emph{Calculus}, og vi vil ikke komme mere ind på dem her).
\end{enumerate}

\paragraph*{Eksempler:}
\begin{enumerate}
\item Løs andengradsligningen $x^2+5x+4 = 0$:

Vi ser at $a=1$, $b=5$ og $c=4$. Dernæst udregner vi diskriminanten
\begin{align*}
d=b^2-4ac = 5^2-4\cdot 1 \cdot 4 = 25-16=9.
\end{align*}
Indsætter vi nu i løsningsformlen får vi
\begin{align*}
x=\frac{-b\pm\sqrt{d}}{2a} = \frac{-5\pm \sqrt{9}}{2 \cdot 1} = \frac{-5 \pm 3}{2} = \begin{cases} -1 \\
-4
\end{cases}.
\end{align*}
\item Løs andengradsligningen $x^2-3x+10=8$:

Før vi kan bruge løsningsformlen skal vi have højresiden til at være lig nul
\begin{align*}
x^2-3x+10=8 \Leftrightarrow x^2-3x +2 = 0.
\end{align*}
Vi ser at $a=1$, $b=-3$ og $c=2$, hvilket medfører at
\begin{align*}
d = b^2-4ac = (-3)^2 - 4 \cdot 1 \cdot 2 = 9-8 =1.
\end{align*}
Det giver os rødderne
\begin{align*}
x=\frac{-b \pm \sqrt{d}}{2a} = \frac{-(-3) \pm \sqrt{1}}{2 \cdot 1} = \frac{3 \pm 1}{2} =  \begin{cases} 2 \\
1
\end{cases}.
\end{align*}
\end{enumerate}


\paragraph*{Specialtilfælde:}
Hvis vi har nogle bestemte andengradspolynomier, så kan vi simplificere den generelle løsning.
\begin{enumerate}
\item Hvis $b=0$ så har vi andengradsligningen $ax^2+c=0$ det kan vi omskrive for at finde rødderne
\begin{align*}
ax^2+c=0 \Leftrightarrow x^2 = \frac{-c}{a} \Leftrightarrow x = \pm \sqrt{\frac{-c}{a}},
\end{align*}
givet at fortegnet på $a$ og $c$ er forskellige.
\item Hvis $c=0$ har vi andengradsligningen $ax^2+bx = 0$ og ved at sætte $x$ udenfor en parentes får vi
\begin{align*}
ax^2+bx = 0 \Leftrightarrow x(ax+b)=0.
\end{align*}
Nulreglen giver så at rødderne er
\begin{align*}
x_1=0 \qquad \textup{ og } \qquad x_2= \frac{-b}{a}.
\end{align*}
\end{enumerate}

\paragraph*{Faktorisering:}
Hvis vi har en andengradsligning $ax^2+bx+c=0$, så kan man omskrive venstresiden til
\begin{align*}
ax^2+bx+c = a(x-r_1)(x-r_2),
\end{align*}
hvor $r_1$ og $r_2$ er rødder til den givne andengradsligning. Dette kaldes at faktorisere sin andengradsligning. Det viser også at enhver andengradsligning er entydigt bestemt ud fra $a$ samt sine rødder, da vi, såfremt vi kender disse, kan genskabe den andengradsligning de kommer fra.

\paragraph*{Eksempel:}
\begin{enumerate}
\item Reducer udtrykket $\displaystyle \frac{2x^2+2x-4}{x-1}$.

Først finder vi rødderne for vores andengradspolynomium. Vi har at $d = b^2-4ac = 2^2-4\cdot 2 \cdot (-4) = 36$, hvilket medfører at vi har rødderne
\begin{align*}
x = \frac{-b\pm \sqrt{d}}{2a} = \frac{-2 \pm 6}{4} = \begin{cases} 1 \\ -2 \end{cases}.
\end{align*} 
Det betyder at vi kan faktorisere vores andengradspolynomium til
\begin{align*}
2x^2+2x-4 = 2(x-1)(x+2).
\end{align*}
Nu kan vi så reducere vores udtryk til
\begin{align*}
\frac{2x^2+2x-4}{x-1} = \frac{2(x-1)(x+2)}{x-1} = 2(x+2)=2x+4.
\end{align*}
\end{enumerate}
\paragraph*{To ligninger med to ubekendte:}
Indtil videre har vi kun betragtet én ligning med en ubekendt (ofte $x$) ad gangen. Det er ikke altid muligt at løse en ligning med flere ubekendte, men hvis vi har lige så mange ligninger som ubekendte så er det ofte muligt at løse dem. Vi vil her betragte to forskellige metoder til at løse sådanne ligningssystemer; \emph{substitutionsmetoden} og \emph{lige store koefficienters metode}. Vi vil primært holde os til to ligninger med to ubekendte men i kurset \emph{Lineær Algebra} vil i lære hvordan man effektivt kan løse flere ligninger med flere ubekendte.

Vi vil løse de følgende to ligninger med to ubekendte
\begin{align}
2x+y+3&=2y-4 \label{eq:1flereubekendte} \\
4x+2&=5y \label{eq:2flereubekendte}
\end{align}
først ved substitutionsmetoden og derefter ved brug lige store koefficienters metode.

I substitutionsmetoden starter vi med at isolere den ene af de ubekendte variable i en af ligningerne. Hvis vi starter med at isolere $x$ i \eqref{eq:1flereubekendte} får vi
\begin{align*}
2x+y+3 = 2y-4 &\Leftrightarrow 2x = y-7 \\
&\Leftrightarrow x = \frac{y-7}{2}.
\end{align*}
Det udtryk  kan vi så indsætte i \eqref{eq:2flereubekendte} så vi får én ligning med en ubekendt som vi kan løse
\begin{align*}
4x+2 = 5y &\Leftrightarrow 4 \cdot \frac{y-7}{2} + 2 = 5y \\
&\Leftrightarrow 2y-14 + 2 = 5y \\
&\Leftrightarrow -12 = 3y \\
&\Leftrightarrow y = -4.
\end{align*}
Indsætter vi nu dette udtryk for $y$ i \eqref{eq:2flereubekendte} får vi
\begin{align*}
4x+2=5 \cdot (-4) &\Leftrightarrow 4x+2=-20 \\
&\Leftrightarrow 4x = -22 \\
&\Leftrightarrow x = \frac{-22}{4} \\
&\Leftrightarrow x = \frac{-11}{2},
\end{align*}
hvilket betyder at løsningen til vores to ligninger med to ubekendte er $x= \frac{-11}{2}$ og $y=-4$. Bemærk at vi kunne også have indsat vores udtryk for $y$ i \eqref{eq:1flereubekendte}.


Idéen i lige store koefficienters metode er igen at omskrive de to ligninger til én ligning med en ubekendt, som vi godt kan løse og så derefter indsætte den variabel vi har fundet i en af de to givne ligninger, så vi igen har én ligning med én ubekendt bare med den anden ubekendte variabel.

Fra \eqref{eq:1flereubekendte} har vi, at $2x + y + 3 = 2y-4$, og ganger vi med $2$ på begge sider får vi at $4x+2y+6 = 4y-8$. Vi husker at vi gerne må trække noget fra på den ene side af en ligning, så længe vi trækker det samme fra på den anden side. Det betyder, at hvis vi trækker $4x + 2y +6$ fra på den ene side i \eqref{eq:2flereubekendte} så kan vi trække $4y-8$ fra på den anden side uden at ændre ved vores lighed. Det giver
\begin{align*}
4x+2-(4x+2y+6)= 5y-(4y-8) &\Leftrightarrow 4x+2-4x-2y-6= 5y-4y+8 \\
&\Leftrightarrow -4 -2y = y+8 \\
&\Leftrightarrow -12 = 3y \\
&\Leftrightarrow y=-4.
\end{align*}
Grunden til at vi gangede \eqref{eq:1flereubekendte} med $2$ før vi trak den fra \eqref{eq:2flereubekendte}, var for at få det samme til til at stå foran $x$ i begge ligninger. Det medførte at alle $x$'erne gik ud med hinanden i \eqref{eq:2flereubekendte}, og vi fik derfor én ligning med en ubekendt, som vi godt kunne løse. Vi mangler stadig at finde $x$, men det kan vi gøre ved at indsætte $y=-4$ i enten \eqref{eq:1flereubekendte} eller \eqref{eq:2flereubekendte} og så isolere $x$. Indsætter vi $y=-4$ i \eqref{eq:2flereubekendte} får vi
\begin{align*}
4x + 2 = 5 \cdot (-4) &\Leftrightarrow 4x+2 = -20 \\
&\Leftrightarrow 4x=-22 \\
&\Leftrightarrow x= \frac{-11}{2}.
\end{align*}
Det betyder at løsningen til vores to ligninger med to ubekendte er $x=\frac{-11}{2}$ og $y=-4$.
























%To typiske andengradspolynomier er afbilledet i Figur ~\ref{fig:andengradspolynomium1} og~\ref{fig:andengradspolynomium2}. De tre tal $a,b,c$ der indgår i et andengradspolynomium har direkte indflydelse på hvordan grafen for andengradspolymiet ser ud.
%
%\begin{figure}[!htbp]
%\begin{minipage}{0.49\textwidth}
%\centering
%  \pgfplotsset{width=0.5\textwidth,compat=1.11}
%  \centering
%  \begin{tikzpicture}
%  \begin{axis}[ 
%    xmin=-2.0,
%    xmax=2.0,
%    ymin=-2.0,
%    ymax=2.0,
%    axis equal,
%    %axis lines=middle,
% ticks=none,
%xlabel={},
%ylabel={},
%  ]
%\addplot[thick,blue,samples=300]{x^2+x-1}; 
%\node[below] at (2.2,0.0) {$x$}; 
%\node[left] at (0.0,1.8) {$y$}; 
%\end{axis}
% \end{tikzpicture}
%  \caption{$f(x)=x^2+x-1$}
%  \label{fig:andengradspolynomium1}
%\end{minipage}
%\begin{minipage}{0.49\textwidth}
%  \pgfplotsset{width=0.5\textwidth,compat=1.11}
%  \centering
%  \begin{tikzpicture}
%  \begin{axis}[ 
%    xmin=-2.0,
%    xmax=2.0,
%    ymin=-2.0,
%    ymax=2.0,
%    axis equal,
%    %axis lines=middle,
% ticks=none,
%xlabel={},
%ylabel={},
%  ]
%\addplot[thick,blue,samples=300]{-1*x^2+x+1}; 
%\node[below] at (2.2,0.0) {$x$}; 
%\node[left] at (0.0,1.8) {$y$}; 
%\end{axis}
% \end{tikzpicture}
%  \caption{$f(x)=-x^2 + x + 1$}
%  \label{fig:andengradspolynomium2}
%\end{minipage}
%\end{figure}
%
