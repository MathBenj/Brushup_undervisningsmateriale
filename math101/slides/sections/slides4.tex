\section{Inverse funktioner}
\begin{frame}{Inverse funktioner}
\begin{itemize}
		\setlength\itemsep{1em}
	\item To funktioner $f\colon X\to Y$ og $g\colon Y\to X$ er hinandens \emph{inverse} hvis
	\begin{align*}
	f(g(y))=y,\quad \textup{og}\quad g(f(x))=x 
	\end{align*}
	 for alle $x$ i $X$ og $y$ i $Y$.
	 \item<9-> Eksempel: $f(x)=x^2$ og $g(x)=\sqrt{x}$ begge defineret på $[0,\infty[$ er inverse funktioner.
	 \item<10-> Eksempel: $f(x)=\frac{1}{x}$ defineret på $\R\setminus\{0\}$ er sin egen invers.
	 
	 %\item I Figur~\ref{fig:fun4} er to inverse funktioner skitseret.
\end{itemize}
	 \begin{figure}[!htbp]
	\pgfplotsset{width=0.5\textwidth,compat=1.11}
	\centering
	\resizebox{5cm}{!}{%
		\begin{tikzpicture}
		\draw[visible on=<2->] \boundellipse{0,0}{0.7}{1.4};
		\draw[visible on=<2->] \boundellipse{5,0}{0.7}{1.4};
		\node[visible on=<2->] at (0.45,1.7) [label=left:\only<2->{$X$}]{};
		\node[visible on=<2->] at (5.45,1.7) [label=left:\only<2->{$Y$}]{}; 
		\node[circle,fill,inner sep=1pt,visible on=<3->] (x) at (0,1) [label=below:\only<3->{$x$}] {};
		\node[circle,fill,inner sep=1pt,visible on=<7->] (gy) at (0,-1) [label=above:\only<7->{$g(y)$}] {};
		\node[circle,fill,inner sep=1pt,visible on=<4->] (fx) at (5,1) [label=below:\only<4->{$f(x)$}] {};
		\node[circle,fill,inner sep=1pt,visible on=<6->] (y) at (5,-1) [label=above:\only<6->{$y$}] {};
		\draw[thick,->,visible on=<4->] (x) to[bend left]node[pos =0.5, label=below: \only<4->{$f$}] {} (fx);
		\draw[thick,->,visible on=<5->] (fx) to[bend left] node[pos =0.5, label=above:\only<5->{$g$}] {} (x);
		\draw[thick,->,visible on=<7->] (y) to[bend left]node[pos =0.5, label=above: \only<7->{$g$}] {} (gy);
		\draw[thick,->,visible on =<8->] (gy) to[bend left]node[pos =0.5, label=below: \only<8->{$f$}] {} (y);
		\end{tikzpicture}%
	}%
%	\caption{Inverse funktioner.}	 
%	\label{fig:fun4}%
\end{figure}
\end{frame}

\section{Logaritmer og eksponentialfunktioner}
\begin{frame}{Logaritmer og eksponentialfunktioner}
\begin{itemize}
			\setlength\itemsep{1em}
	\item<1-> For ethvert positivt $a\neq 1$ kalder vi funktionen $f_a\colon \R\to ]0,\infty[$ givet ved $f_a(x)=a^x$ for \emph{eksponentialfunktionen med grundtal $a$}.
	\item<2-> Funktionen $f_a(x)=a^x$ har en invers funktion $\log_a\colon ]0,\infty[\to \R$ som kaldes \emph{logaritmen med grundtal $a$}.
	\item<3-> Hvis $a=e$ så skriver vi $\ln$ i stedet for $\log_e$ og hvis $a=10$ skriver vi $\log$ i stedet for $\log_{10}$.
	\item<4-> Der gælder at
	\begin{align*}
	log_a(a^x)=x\quad \textup{og} \quad a^{\log_a(y)}=y,
	\end{align*}
	for alle $x\in \R$ og $y\in ]0,\infty[$.
	\item<5-> Eksempler: Udregn
\end{itemize}
	\begin{align*}
\onslide<5->{\log_2(8)\setbeamercovered{}\onslide<6->{=\log_2(2^3)}\onslide<7->{=3},}&&\onslide<8->{\log_{10}(10000)\setbeamercovered{}\onslide<9->{=\log_{10}(10^4)\onslide<10->{=4}},}&&\onslide<11->{\log_a(1)\setbeamercovered{}\onslide<12->{=\log_a(a^0)\onslide<13->{=0}}.}
\end{align*}
\end{frame}

\begin{frame}{Logaritmer og eksponentialfunktioner}{Regneregler}
\begin{itemize}
			\setlength\itemsep{1em}
	\item<1-> Når vi arbejder med eksponentialfunktioner kan vi anvende potensregneregler.
	\item<2-> For logaritmer har vi følgende regneregler
	\end{itemize}
	\begin{align*}
	\onslide<2->{\log_a(xy)&=\log_a(x)+\log_a(y),}\\\onslide<3->{\log_a(\frac{x}{y})&=\log_a(x)-\log_a(y),}\\ \onslide<4->{
	\log_a(x^r)&=r\log_a(x).}%,&&\log_a(x)=\frac{\ln(x)}{\ln(a)}.
	\end{align*}
	\begin{itemize}
	\item<5-> Eksempler: Udregn
\end{itemize}
\begin{align*}
&\onslide<5->{\log(50)+\log(20)\setbeamercovered{}\onslide<6->{=\log(50\cdot 20)\onslide<7->{=\log(1000)}\onslide<8->{=3,}}}\\
&\onslide<9->{2^{2+\log_2(5)}\setbeamercovered{}\onslide<10->{=2^22^{\log_2(5)}\onslide<11->{=4\cdot 5}\onslide<12->{=20,}}}\\
&\onslide<13->{9^{\log_3(2)}\setbeamercovered{}\onslide<14->{=(3^2)^{\log_3(2)}\onslide<15->{=3^{2\log_3(2)}\onslide<16->{=3^{\log_3(2^2)}\onslide<17->{=4.}}}}}
\end{align*}
\end{frame}

\section{Trigonometriske funktioner}
\begin{frame}{Trigonometriske funktioner}
\begin{itemize}
	\item<1-> Vi definerer de trigonometriske funktioner ud fra enhedscirklen.
\end{itemize}
\begin{figure}
	\centering
	\resizebox{5cm}{!}{%
		\begin{tikzpicture}
		\begin{axis}[xmin=-0.1,xmax=1.2,ymin=-0.1,ymax=1.2,axis x line=center,
		axis y line=center, axis equal, xtick={0,1},ytick={0,1}]
		%PERMANENT STUFF
		\addplot[blue,domain=0:pi/2,thick, samples=100] ({cos(deg(x))},{sin(deg(x))});
		\addplot[domain=0:sqrt(3)/2,thick] {1/sqrt(3)*x};
		\addplot[domain=0:pi/6,thick,samples=100] ({0.2*cos(deg(x))},{0.2*sin(deg(x))}) node[label=right:{\small$\theta$},pos=0.5] {};
		%SHOW THIS FIRST
		\addplot[dotted,gray,thick,visible on=<2->] coordinates {(sqrt(3)/2,0) (sqrt(3)/2, 1/2)};
		\node[visible on=<3->] at (axis cs: {sqrt(3)/4},-0.05) {$\cos(\theta)$};
		\addplot[thick,red,domain=0:sqrt(3)/2,visible on=<3-4>] {0};
		%SHOW THIS SECOND
		%\node[visible on=<3->,label={[label distance=0cm,text depth=-1ex,rotate=90]:$\sin(\theta)$}] at (axis cs:-0.1,1/4) {};
		\node[visible on=<5->] at (axis cs: -0.05,1/4) {\rotatebox{90}{$\sin(\theta)$}};
		\addplot[dotted, gray,thick,domain=0:sqrt(3)/2,visible on=<4->] {1/2};	
		\addplot[thick,red,visible on=<5-6>]  coordinates { (0,0) (0,1/2)};
		%SHOW THIS THIRD
		\addplot[thick,domain=sqrt(3)/2:1.2,visible on= <6->] {1/sqrt(3)*x};
		\addplot[thick,dotted,visible on=<6->] coordinates {(1,0) (1,1.2)};
		%\node[label={[label distance=0cm,text depth=-1ex,rotate=90]:$\tan(\theta)$},visible on=<4->] at (axis cs:1.1,1/4) {};
		\node[visible on=<7->] at (axis cs: 1.05,1/4) {\rotatebox{90}{$\tan(\theta)$}};
		\addplot[red,thick,visible on=<7->] coordinates {(1,0) (1, 1/sqrt(3)};

		\end{axis}
		\end{tikzpicture}%
	}%
\end{figure}
\begin{itemize}
	\item<8-> Bemærk at $\tan(\theta)=\frac{\sin(\theta)}{\cos(\theta)}$.
\end{itemize}
\end{frame}

\begin{frame}{Trigonometriske funktioner}{Eksakte værdier}
\begin{itemize}
	\item For særlige vinkler kan vi bestemme eksakte værdier af de trigonometriske funktioner.
\end{itemize}
\begin{minipage}{0.49\textwidth}
\resizebox{5cm}{!}{%	
\begin{tikzpicture}
\begin{axis}[xmin=-0.1,xmax=1.2,ymin=-0.1,ymax=1.2,axis x line=center,
axis y line=center, axis equal, xtick={0,1/2,sqrt(2)/2,sqrt(3)/2,1},ytick={0,1/2,sqrt(2)/2,sqrt(3)/2,1}, xticklabels={$0$, $\frac{1}{2}$,$\frac{\sqrt{2}}{2}$,$\frac{\sqrt{3}}{2}$,$1$}, yticklabels={$0$, $\frac{1}{2}$,$\frac{\sqrt{2}}{2}$,$\frac{\sqrt{3}}{2}$,$1$}]
%PERMANENT STUFF
\addplot[blue,domain=0:pi/2,thick, samples=100] ({cos(deg(x))},{sin(deg(x))});
\node[circle,fill,inner sep=1pt,label=above right:$0$] at (axis cs: 1,0) {};
\node[circle,fill,inner sep=1pt,label=above right:$\frac{\pi}{2}$] at (axis cs: 0,1) {};
%PI/6
\addplot[dotted,thick,visible on =<3->,red] coordinates {(sqrt(3)/2,0) (sqrt(3)/2, 1/2)};
\node[circle,fill,inner sep=1pt,label=right:$\frac{\pi}{6}$,red] at (axis cs: {sqrt(3)/2},1/2) {};
\addplot[dotted,thick,domain=0:sqrt(3)/2,visible on =<3->,red] {1/2};	
%PI/4
\addplot[dotted,thick,visible on =<4->,green] coordinates {(sqrt(2)/2,0) ({sqrt(2)/2}, {sqrt(2)/2})};
\node[circle,fill,inner sep=1pt,label=right:$\frac{\pi}{4}$,green] at (axis cs: {sqrt(2)/2},{sqrt(2)/2}) {};
\addplot[dotted,thick,domain=0:sqrt(2)/2,visible on =<4->,green] {sqrt(2)/2};	
%PI/3
\addplot[dotted,gray,thick,visible on =<5->] coordinates {(1/2,0) (1/2,{sqrt(3)/2})};
\node[circle,fill,inner sep=1pt,label=right:$\frac{\pi}{3}$] at (axis cs: 1/2,{sqrt(3)/2}) {};
\addplot[dotted, gray,thick,domain=0:1/2,visible on =<5->] {sqrt(3)/2};
\end{axis}
\end{tikzpicture}%
}%
\end{minipage}
\begin{minipage}{0.49\textwidth}
\begin{table}[h!]
	\begin{tabular}{@{} lccc @{}}
	\toprule 
		$\theta$			& $\sin \theta$			& $\cos \theta$ 		& $\tan \theta$ 		\\ \midrule
		0					\onslide<2->{&0						&1						&0			}\\
		$ \frac{\pi}{6} $	\onslide<3->{&$\frac{1}{2}$			&$\frac{\sqrt{3}}{2}$	&$\frac{1}{\sqrt{3}}$}	\\
		$ \frac{\pi}{4} $	\onslide<4->{&$\frac{\sqrt{2}}{2}$	&$\frac{\sqrt{2}}{2}$	&$1$				}	\\
		$ \frac{\pi}{3} $	\onslide<5->{&$\frac{\sqrt{3}}{2}$	&$\frac{1}{2}$			&$\sqrt{3}$			}	\\
		$ \frac{\pi}{2} $	\onslide<6->{&1						&0						&					}	\\
		\bottomrule  
	\end{tabular}
\end{table}
\end{minipage}
\end{frame}

\begin{frame}{Trigonometriske funktioner}{Eksempler}
\begin{itemize}
			\setlength\itemsep{1em}
	\item<1-> Når I skal løse opgaver så tegn altid enhedscirklen og udnyt symmetri.
	\item<2-> Eksempler: Udregn
\end{itemize}
\begin{align*}
\onslide<2->{\cos(\frac{2\pi}{3})\setbeamercovered{}\onslide<6->{=-\cos(\frac{\pi}{3})=-\frac{1}{2}},}&& \onslide<6->{\sin(9\pi)\setbeamercovered{}\onslide<7->{=\sin(\pi)=0},}&& \onslide<7->{\sin(-\frac{5\pi}{4})\setbeamercovered{}\onslide<12>{=\sin(\frac{\pi}{4})=\frac{\sqrt{2}}{2}}.}
\end{align*}
\onslide<2->{%
\begin{figure}
	\resizebox{4.5cm}{!}{%	
		\begin{tikzpicture}
		\begin{axis}[xmin=-1.1,xmax=1.1,ymin=-1.1,ymax=1.1,axis x line=center,
		axis y line=center, axis equal, xtick={-1,1},ytick={-1,1}, xticklabels={$-1$,$1$}, yticklabels={$-1$,$1$}]
		%PERMANENT STUFF
		\addplot[blue,domain=0:2*pi,thick, samples=100] ({cos(deg(x))},{sin(deg(x))});
%		%FIRST
		\node[circle,fill,inner sep=1pt,label=right:\only<3-5>{$\frac{\pi}{3}$}, visible on=<3-5>] at (axis cs: 1/2,{sqrt(3)/2}) {};
		\node[circle,fill,inner sep=1pt,label=left:\only<4-5>{$\frac{2\pi}{3}$},visible on=<4-5>] at (axis cs: -1/2,{sqrt(3)/2}) {};
		\addplot[dotted,gray,thick,visible on =<5-5>] coordinates {(1/2,0) (1/2,{sqrt(3)/2})};
		\addplot[dotted,gray,thick,visible on =<5-5>] coordinates {(-1/2,0) (-1/2,{sqrt(3)/2})};

		\node[circle,fill,inner sep=1pt,label=above left:\only<{6,10-}>{$\pi$},visible on={<6,10->}] at (axis cs: -1,0) {};		

		
		\node[circle,fill,inner sep=1pt,label=left:\only<11->{$-\frac{5\pi}{4}$},visible on=<11->] at (axis cs: {-sqrt(2)/2},{sqrt(2)/2}) {};
		\node[circle,fill,inner sep=1pt,label=right:\only<7->{$-\frac{\pi}{4}$},visible on=<7->] at (axis cs: {sqrt(2)/2},{-sqrt(2)/2}) {};
		\node[circle,fill,inner sep=1pt,label=right:\only<7->{$\frac{\pi}{4}$},visible on=<7->] at (axis cs: {sqrt(2)/2},{sqrt(2)/2}) {};
		\node[circle,fill,inner sep=1pt,label=above:\only<8->{$-\frac{\pi}{2}$},visible on=<8->] at (axis cs: 0,-1) {};		
		\node[circle,fill,inner sep=1pt,label=left:\only<9->{$-\frac{3\pi}{4}$},visible on=<9->] at (axis cs: {-sqrt(2)/2},{-sqrt(2)/2}) {};		
		\addplot[dotted,gray,thick,visible on =<12>,domain=-sqrt(2)/2:sqrt(2)/2]  {sqrt(2)/2};

		
		


		
%		\addplot[dotted, gray,thick,domain=0:1/2,visible on =<5->] {sqrt(3)/2};
%
%		%PI/4
%		\addplot[dotted,thick,visible on =<4->,green] coordinates {(sqrt(2)/2,0) ({sqrt(2)/2}, {sqrt(2)/2})};
%		\node[circle,fill,inner sep=1pt,label=right:$\frac{\pi}{4}$,green] at (axis cs: {sqrt(2)/2},{sqrt(2)/2}) {};
%		\addplot[dotted,thick,domain=0:sqrt(2)/2,visible on =<4->,green] {sqrt(2)/2};	
%		%PI/3

		\end{axis}
		\end{tikzpicture}%
	}%
\end{figure}%
}
\end{frame}













