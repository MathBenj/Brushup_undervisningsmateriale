\section{Differentialregning}
\begin{frame}{Differentialregning}
\begin{itemize}
	\item<1-> Differentialregning omhandler bestemmelse af hældninger af funktioner.
	\item<2-> Vi definerer en funktions hældning vha. sekanter.
\end{itemize}
\begin{figure}
	\resizebox{5cm}{!}{%	
		\begin{tikzpicture}
		\begin{axis}[xmin=-2.1,xmax=2.3,ymin=-0.5,ymax=2.7,axis x line=center,
		axis y line=center, axis equal,ticks=none]
		\addplot[thick,blue,samples=300] {ln(x+2)/ln(2)+1/2};
		\node[circle,fill,inner sep=1pt,label=above left:\only<{3-}>{$(x_0,f(x_0)$)},visible on={<3->}] at (axis cs: -1,1/2) {};		
		\node[circle,fill,inner sep=1pt,label=below:\only<{3-}>{$ x_0 $},visible on={<3->}] at (axis cs: -1,0) {};
		
		\node[circle,fill,inner sep=1pt,label=below:\only<{4-4}>{$ x_0+3 $},visible on={<4-4>}] at (axis cs: 2,0) {};
		\node[circle,fill,inner sep=1pt,visible on =<4->] at (axis cs: 2,5/2) {};
		\addplot[thick,black,visible on=<4-4>] {2*x/3+7/6};
		
		
	\node[circle,fill,inner sep=1pt,label=below:\only<{5}>{$ x_0+1 $},visible on={<5>}] at (axis cs: 0,0) {};
	\node[circle,fill,inner sep=1pt,visible on =<5->] at (axis cs: 0,3/2) {};
	\addplot[thick,black,visible on=<5>] {x+3/2};
	\addplot[thick,gray,dashed,visible on=<5->] {2*x/3+7/6};
	
	\node[circle,fill,inner sep=1pt,label=below:\only<{6}>{$ x_0+\frac{1}{2}$},visible on={<6>}] at (axis cs: -1/2,0) {};
	\node[circle,fill,inner sep=1pt,visible on =<6->] at (axis cs: -1/2,{ln(3/2)/ln(2)+1/2}) {};
	\addplot[thick,black,visible on=<6>] {2*ln(3/2)/ln(2)*x+ln(81/8)/ln(4)};
	\addplot[thick,gray,dashed,visible on=<6->] {x+3/2};
	

	\addplot[thick,gray,dashed,visible on=<7->] {2*ln(3/2)/ln(2)*x+ln(81/8)/ln(4)};
	\addplot[thick,red,visible on =<7->] {1/ln(2)*(x+1)+1/2};
	
		\end{axis}
		\end{tikzpicture}%
	}%
\end{figure}
\end{frame}

\begin{frame}{Differentialregning}
\begin{itemize}
			\setlength\itemsep{1em}
	\item<1-> En funktion $f$ er differentiabel i $x_0$ hvis grænsen
	\begin{align*}
	f'(x_0)=\lim_{h\to 0}\frac{f(x_0+h)-f(x_0)}{h}
	\end{align*}
	eksisterer.
	\item<2-> Bemærk at $f'(x)$ betegner hældningen af $f$ i $x$.
	\item<3-> Vi anvender ofte notationen
	\begin{align*}
	f'(x)=\frac{d}{dx} f(x)=\frac{df}{dx}(x).
	\end{align*}
\end{itemize}
\end{frame}

\section{Regneregler for kendte funktioner}
\begin{frame}{Differentialregning}{Regneregler}
	\vspace{-0.5cm}
\begin{itemize}
			\setlength\itemsep{1em}
	\item<1-> Vi har følgende regneregler:
\end{itemize}
\begin{minipage}{0.49\textwidth}
	\centering
	\begin{tabular}{@{}l c@{}}
\onslide<1->{$f(x)$      & $f'(x)$}  				\\ \toprule
\onslide<2->{$c$			& $0$} 					\\ \midrule
\onslide<3->{$x$			& $1$}					\\ \midrule
\onslide<4->{$x^n$  		& $nx^{n-1}$}			\\ \midrule
\onslide<5->{$e^x$  		& $e^x$}					\\ \midrule
\onslide<6->{$e^{cx}$  	& $ce^{cx}$}				\\ \bottomrule
	\end{tabular}
\end{minipage}
\begin{minipage}{0.49\textwidth}
	\centering
\begin{tabular}{@{}l c@{}}
\onslide<1->{$f(x)$      & $f'(x)$}  				\\ \toprule
\onslide<7->{$a^x$  		& $a^x\ln a $}			\\ \midrule
\onslide<8->{$\ln x$ 	& $\frac{1}{x}$}			\\ \midrule
\onslide<9->{$\cos x$  	& $-\sin x$}				\\ \midrule
\onslide<10->{$\sin x$  	& $\cos x$}				\\ \midrule
\onslide<11->{$\tan x$ 	& $1+\tan^2(x)$}		\\ \bottomrule  
\end{tabular}
\end{minipage}
\begin{itemize}
			\setlength\itemsep{1em}
	\item<12-> Eksempler: Differentier funktionerne 
\end{itemize}
\begin{align*}
\onslide<12->{f(x)=\sqrt{x}\setbeamercovered{}\onslide<13->{=x^{\frac{1}{2}}},}&& \onslide<14->{g(x)=\frac{1}{x}\setbeamercovered{}\onslide<15->{=x^{-1}},}&& \onslide<16->{h(x)=\ln(x^3)\setbeamercovered{}\onslide<17->{=3\ln(x)}.}
\end{align*}
\end{frame}
\section{Generelle regneregler}
\begin{frame}{Differentialregning}{Regneregler}
\begin{itemize}
			\setlength\itemsep{1em}
	\item<1-> Vi har følgende generelle regneregler
	\end{itemize}
	\begin{align*}
	\onslide<1->{(cf)'(x)&=cf'(x)}\\
	\onslide<2->{(f\pm g)'(x)&=f'(x)\pm g'(x).}
	\end{align*}
	\begin{itemize}
	\item<3-> Eksempler: Differentier funktionerne 
\end{itemize}
\begin{align*}
\onslide<3->{f(x)=2x+1-\frac{1}{x},}&& \onslide<5->{ g(x)=3x^{-2}-2e^{-x}+\cos(x)}\\
\setbeamercovered{} \onslide<4->{f'(x)=2+x^{-2},}&& \setbeamercovered{}\onslide<6->{ g'(x)=-6x^{-3}+2e^{-x}-\sin(x)}
\end{align*}
\end{frame}
