%%%
%%% Macros
%%%
%%
%% Theorem-like environments 
%%
%
% The theorem style follows the recommendations of AMS (see amsthm documentation page 7)
%
\theoremstyle{plain}
\newtheorem{theorem}{Theorem}[section]
\newtheorem{lemma}[theorem]{Lemma}
\newtheorem{proposition}[theorem]{Proposition}
\newtheorem{corollary}[theorem]{Corollary}
%
\theoremstyle{definition}
\newtheorem{definition}[theorem]{Definition}
\newtheorem{example}[theorem]{Example}
%
\theoremstyle{remark}
\newtheorem{remark}[theorem]{Remark}
%%
%% Notation
%%
%
% Spaces
%
\newcommand{\Cinf}{C^\infty}
\newcommand{\No}{\mathbf{N}_0}
\newcommand{\N}{\mathbf{N}}	
\newcommand{\Z}{\mathbf{Z}}
\newcommand{\R}{\mathbf{R}}
\newcommand{\C}{\mathbf{C}}
\newcommand{\SR}{\mathscr{S}(\R^d)}
\newcommand{\SpR}{\mathscr{S}'(\R^d)}
\newcommand{\sH}{\mathscr{H}}
%
% Functions/Operators
% 
\newcommand{\abs}[1]{\vert #1\vert}
\newcommand{\pair}[2]{\langle #1,#2\rangle}
\newcommand{\ip}[2]{\langle #1,#2\rangle}
\newcommand{\norm}[1]{\Vert #1\Vert}
\newcommand{\Op}{\mathrm{Op}}
\newcommand{\dist}{\mathrm{dist}}
\newcommand{\jn}[1]{\langle#1\rangle}
\renewcommand{\d}{\; d}


\newcommand{\boundellipse}[3]% center, xdim, ydim
{(#1) ellipse (#2 and #3)
}