\newpage
\section{Math101 facit til 3. gang}
\begin{enumerate}
	
	\item Svarene er: $f(-1)=2$ og $f(2)=17$.
	
	\item Svarene er:
	\begin{itemize}
		\item Nej fordi i såfald skulle $f(0)$ være lig med både $2$ og $0$ samtidig.
		\item $f_+(x) = 1+\sqrt{1-x^2}$.
		\item $f_-(x) = 1-\sqrt{1-x^2}$
	\end{itemize}	
	
	\item Svaret er $(f\circ g)(x)=x$.
	

	\item Svarene er:
	\begin{align*}
	D(f)=\R\setminus\{1\},&& D(g)=\R\setminus \{-1,1\},&& D(h)=[\frac{3}{2},\infty[.
	\end{align*}	
	
	\item  Svarene er $(f\circ g)(1)=\frac{\sqrt{2}}{2}$ og $(g\circ f)(1)=\frac{1}{2}$, hvorfor $f\circ g\neq g\circ f$?
	
	\item Skæringspunktet er $(\frac{1}{4},\frac{7}{4})$.	
	
		\item Svarene er $(f\circ g)(x)=1$ og $(g\circ f)(x)=5$.
	
		\item Svarene er:
	\begin{align*}
	D(f)=\R,&& D(g)=\R\setminus\{1,3\},&& D(h)=[0,2].
	\end{align*}
	
	
	

	\item Tag $f(x)=e^x$ og $g(x)=2x^2-1$.
	
	\item Skæringspunktet er $(-1,1)$.
	
	\item Tag $f(x)=x^2$, $g(x)=\sin(x)$ og $h(x)=3x$.
	
	\item Svarene er:
	\begin{align*}
	f(g(x))=\frac{3x^2}{(1-2x)^2},&& f(h(x))=\frac{3}{x},&& h(g(x))=\frac{1}{\sqrt{x}}+2,\\ h(f(x))=\sqrt{3}\frac{1}{x-2}+2,&&g(f(h(x)))=\frac{x}{3}.
	\end{align*} 
	
		\item Nej.
	
	\item \label{it:fun5} I Figur~\ref{fig:fun5} ses en funktion som opfylder:
	\begin{enumerate}
		\item har domæne $[-1,1]$,
		\item går gennem punkterne $(-1,0)$ og $(1,1)$,
		\item skærer $y$-aksen i $-1$,
	\end{enumerate}
	Bemærk at der findes mange korrekte svar.
	
	\begin{figure}
		\centering
		\begin{tikzpicture}
		\begin{axis}[xmin=-1,xmax=1,ymin=-1,ymax=1,axis x line=center,
		axis y line=center, restrict y to domain =-5:5]
		\addplot[thick,blue,samples=200, domain= -1:0] {-x-1};
		\addplot[thick,blue,samples=200, domain= 0:1] {(2*x-1)};
		\end{axis}
		\end{tikzpicture}
		\caption{Opgave~\ref{it:fun5}.}
		\label{fig:fun5}
	\end{figure}
\end{enumerate}