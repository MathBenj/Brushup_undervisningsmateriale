\section{Brøker}
Brøker er tal på formen
\begin{align*}
\frac{a}{b},
\end{align*}
hvor $a,b$ er tal samt $b\neq 0$. $a$ er \emph{tælleren} og $b$ er \emph{nævneren}.
\subsection{Regneregler}
Der gælder
\begin{align*}
\frac{a}{c}\pm\frac{b}{c}&=\frac{a\pm b}{c},&&\frac{a}{b}\frac{c}{d}=\frac{ac}{bd},&&\frac{\frac{a}{b}}{\frac{c}{d}}=\frac{ad}{bc},\\
a\frac{b}{c}&=\frac{ab}{c},&&\frac{\frac{a}{b}}{c}=\frac{a}{bc},&&\frac{a}{\frac{b}{c}}=\frac{ac}{b}.
\end{align*}
\subsection{Forkorte/Forlænge Brøker}
Fælles faktorer kan forkortes:
\begin{align*}
\frac{a}{b}=\frac{ac}{bc}
\end{align*}

\section{Potenser}
Potenser er tal på formen
\begin{align*}
x^a.
\end{align*}
$x$ er \emph{grundtallet} og $a$ er \emph{eksponenten}.
\subsection{Regneregler}
Der gælder
\begin{align*}
x^ax^b=x^{a+b},&& \frac{x^a}{x^b}=x^{a-b},&&(xy)^a=x^ay^a,\\
\Big(\frac{x}{y}\Big)^a=\frac{x^a}{y^a},&&(x^a)^b=x^{ab},&& x^{-a}=\frac{1}{x^a}.
\end{align*}

\section{Rødder}
Hvis $x\geq 0$ og $n\in \Z_+$ så findes et tal $\sqrt[n]{x}>0$ så
\begin{align*}
(\sqrt[n]{x})^n=x.
\end{align*}
Bemærk at $\sqrt[n]{x}=x^{\frac{1}{n}}$.
\subsection{Regneregler}
Der gælder
\begin{align*}
\sqrt[n]{x}=x^{\frac{1}{n}},&& \sqrt[n]{x^m}=x^{\frac{m}{n}}=(\sqrt[n]{x})^m,\\
\sqrt[n]{xy}=\sqrt[n]{x}\sqrt[n]{y},&& \sqrt[n]{\frac{x}{y}}=\frac{\sqrt[n]{x}}{\sqrt[n]{y}}.
\end{align*}

\section{Kvadratsætninger}
Der gælder
\begin{align*}
(a+b)^2&=a^2+b^2+2ab\\
(a-b)^2&=a^2+b^2-2ab\\
(a+b)(a-b)&=a^2-b^2.
\end{align*}