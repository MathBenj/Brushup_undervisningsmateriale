\section{Grænseværdier og kontinuitet}
\noindent Når vi senere vil snakke om differentiabilitet, vil vi snakke om at tage grænsen af sekanthældningen, men hvad mener vi med at tage ``grænsen'', det vil vi nu specificere. Vi siger at $f(x)$ har en grænseværdi $L^-$ fra venstre for $x$ gående mod $a$ fra venstre og noterer det med
\begin{align*}
\lim_{x \to a^-} f(x) = L^-,
\end{align*}
hvis $f(x)$ kan komme lige så tæt på $L^-$ som vi ønsker, ved at lade $x$ komme tættere og tættere på $a$ fra venstre. På tilsvarende vis siger vi, at $f(x)$ har en grænseværdi $L^+$ fra højre for $x$ gående mod $a$ fra højre og notere det med 
\begin{align*}
\lim_{x \to a^+} f(x) = L^+.
\end{align*}
Hvis der gælder at 
\begin{align*}
\lim_{x \to a^-} f(x)=L^-=L^+=\lim_{x \to a^+} f(x),
\end{align*}
så siger vi, at $f(x)$ har en grænseværdi $L$ for $x$ gående mod $a$ og notere det med
\begin{align*}
\lim_{x \to a } f(x) =L.
\end{align*}
Konceptet med en grænseværdi er ikke let at forstå, derfor illustrerer vi det nu ved hjælp af figurer, og dernæst med nogle simple eksempler. Hvis vi betragter Figur~\ref{fig:graenseret} ser vi, at hvis vi lader $x$ nærme sig $a$ fra henholdsvis højre og venstre så vil vi få
\begin{align*}
\lim_{x \to a^-} f(x)=L \qquad \textup{ og } \qquad \lim_{x \to a^+} f(x)=L,
\end{align*}
hvilket medfører at funktionen $f$ har en grænseværdi i $a$ som er
\begin{align*}
\lim_{x \to a} f(x) =L.
\end{align*}
Hvis vi derimod betragter Figur~\ref{fig:graenserto}, ser vi at 
\begin{align*}
\lim_{x \to a^-} f(x)=L^- \qquad \textup{ og } \qquad \lim_{x \to a^+}=L^+,
\end{align*}
hvor $L^- \neq L^+$. Det betyder at grænsen fra højre og grænsen for venstre ikke er lig med hinanden i punktet $a$, og dermed har $f$ ikke en grænseværdi for $x$ gående mod $a$.

Bemærk, at man kan godt snakke om $f$ har en grænseværdi for $x$ gående mod $a$ selvom $f(a)$ slet ikke er defineret.
\begin{figure}[!htbp]
\begin{minipage}{0.49\textwidth}
\centering
\begin{tikzpicture}
	\begin{axis}[xmin=-1.5,xmax=1.5,ymin=-1.5,ymax=1.5,ticks=none,restrict y to domain=-1.5:1.5]
	\addplot[thick, samples=100,domain=-1.1:1.1] {x^3};
	\draw[dashed] (axis cs:1,0) -- (axis cs:1,1);	
	\node[below] at (axis cs:1,0){$a$};
	\draw[dashed] (axis cs:0,1) -- (axis cs:1,1);	
	\node[left] at (axis cs:0,1){$L^-=L^+=L$};
	\node[above left=3 pt] at (axis cs:1.1,1.1){$f(x)$};
\end{axis}
\end{tikzpicture}
\caption{Grænseværdi i $x=a$.}
\label{fig:graenseret}
\end{minipage}
\begin{minipage}{0.49\textwidth}
 \centering
\begin{tikzpicture}
	\begin{axis}[xmin=-1.3,xmax=1.7,ymin=-1.2,ymax=1.5,ticks=none,restrict y to domain=-1:1.5]
	\addplot[domain=-1.1:1,thick, samples=100] {x^2};
	\addplot[domain=1:10,thick, samples=100] {-x+2.3};
	\draw[dashed] (axis cs:1,0) -- (axis cs:1,1);	
	\node[below] at (axis cs:1,0){$a$};
	\draw[fill] (axis cs:1,1) circle[radius=0.1em];
	\draw[fill=white] (axis cs:1,1.3) circle[radius=0.1em];
	\node[above right] at (axis cs:-1.1,1.2){$f(x)$};
	\draw[dashed] (axis cs:0,1) -- (axis cs:1,1);	
	\node[left] at (axis cs:0,1){$L^-$};
	\draw[dashed] (axis cs:0,1.3) -- (axis cs:1,1.3);	
	\node[left] at (axis cs:0,1.3){$L^+$};
\end{axis}
\end{tikzpicture}
\caption{Ingen grænseværdi i $x=a$.}
\label{fig:graenserto}
\end{minipage}
\end{figure}

\paragraph*{Eksempler:}
\begin{enumerate}
\item Lad $f(x)=x^3$ og vis at $f$ har en grænseværdi for $x$ gående mod $2$ og bestem denne.

Vi ser, at vi kan lade $f(x)$ komme lige så tæt på $8$ som vi ønsker ved at lade $x$ komme tættere og tættere på $2$ fra både højre og venstre, hvilket medfører at
\begin{align*}
\lim_{x \to 2^-} f(x) = 8 \qquad \textup{ og } \qquad \lim_{x \to 2^+} f(x) = 8.
\end{align*}
Da grænsen fra højre og venstre er den samme har vi 
\begin{align*}
\lim_{x \to 2} x^3=8.
\end{align*}
\item Har funktionen
\begin{align*}
f (x) = 
\begin{cases}
x^2 & \textup{ hvis } x \leq 1, \\
-x+3 & \textup{ hvis } x > 1,
\end{cases}
\end{align*}
en grænseværdi for $x$ gående mod $1$?

Vi forklarer først kort hvad vi mener med den måde $f$ er defineret på. Hvis vores $x$-værdi er mindre eller lig $1$ så er $f(x)=x^2$, mens vi for alle $x$-værdier der er skarpt større end $1$ har at $f(x)=-x+3$. Hvis $f$ er givet på denne måde, siger man, at $f$ er en gaffelfunktion.

Vi finder grænsen for $f(x)$ når $x$ går mod $1$ fra henholdsvis venstre og højre
\begin{align*}
\lim_{x \to 1^-} f(x) = \lim_{x \to 1^-} x^2 = 1 \qquad \textup{ og } \qquad \lim_{x \to 1^+} f(x)= \lim_{x \to 1^+} -x+3 = 2, 
\end{align*} 
da $f(x)=x^2$ kan komme vilkårligt tæt på $1$ ved at lade $x$ komme tættere og tættere på $1$ og tilsvarende kan vi lade $f(x)=-x+3$ komme vilkårligt tæt på $2$. Da
\begin{align*}
\lim_{x \to 1^-} f(x) \neq \lim_{x \to 1^+} f(x),
\end{align*}
har $f$ ikke en grænseværdi for $x$ gående mod $1$.
\end{enumerate}
\paragraph*{Regneregler:}
Lad $f$ og $g$ være to funktioner med grænseværdier når $x$ går mod $a$. Så har vi følgende regneregler for grænseværdierne:
\begin{enumerate}
\item $\displaystyle\lim_{x \to a} (f+g)(x) = \displaystyle\lim_{x \to a} f(x) + \lim_{x \to a} g(x)$.
\item $\displaystyle\lim_{x \to a} cf(x) = c \displaystyle\lim_{x \to a} f(x)$, hvor $c \in \mathbb{R}$.
\item $\displaystyle\lim_{x \to a} (fg)(x) = \displaystyle\lim_{x \to a} f(x) \cdot \lim_{x \to a} g(x)$.
\item $\displaystyle \lim_{x \to a} \Big(\frac{f}{g}\Big)(x) = \frac{\displaystyle\lim_{x \to a} f(x)}{\displaystyle\lim_{x \to a} g(x)}$, hvis $\displaystyle\lim_{x \to a} g(x) \neq 0$.
\end{enumerate}

\paragraph*{Kontinuitet:}
En funktion $f$ siges at være kontinuert i punktet $a$, hvis
\begin{align}\label{eq:graenserkontinuitetet}
\lim_{x \to a} f(x) = f(a).
\end{align}
Det betyder, at $f$ er kontinuert i $a$ hvis den har en grænseværdi i punktet $a$ der er lig med $f(a)$.

Vi siger, at $f$ er en kontinuert funktion, hvis den er kontinuert i alle punkter i dens definitionsmængde.  

Hvis $f$ og $g$ er kontinuerte funktioner, så har vi fra \eqref{eq:graenserkontinuitetet}, at
\begin{align*}
\lim_{x \to a} f(g(x)) = f(\lim_{x \to a} g(x)) = f(g(a)).
\end{align*}
Bemærk, at funktioner så som $x,ax+b,ax^2+bx+c,\frac{1}{x},\sqrt{x},\ln(x), e^x, \sin(x),\cos(x)$ alle er kontinuerte på passende domæne.

\paragraph*{Eksempler:}
\begin{enumerate}
\item Vis at funktionen 
\begin{align*}
f(x)=
\begin{cases}
2x+3 & x \neq 0, \\
3 & x=0,
\end{cases}
\end{align*}
er kontinuert i $0$.

Vi finder først grænseværdien for $f$ når $x$ går mod $0$, ved at finde grænseværdien fra henholdsvis venstre og højre
\begin{align*}
\lim_{x \to 0^-} f(x) = \lim_{x \to 0^-} 2x+3=3 \qquad \textup{ og } \qquad \lim_{x \to 0^+} f(x) = \lim_{x \to 0^+} 2x+3 = 3,
\end{align*}
hvilket medfører at $\displaystyle\lim_{x \to 0} f(x) = 3$.

Vi ser derudover at $f(0)=3$, så 
\begin{align*}
\lim_{x \to 0} f(x) = 3 = f(0),
\end{align*}
hvilket betyder at $f$ er kontinuert i $0$.
\item Vis at funktionen 
\begin{align*}
f(x)=
\begin{cases}
2x+3 & x \neq 0, \\
4 & x=0,
\end{cases}
\end{align*}
ikke er kontinuert i $0$.

Vi har fra den forrige opgave at $\lim_{x \to 0} f(x)=3$ men da $f(0) = 4$, har vi at
\begin{align*}
\lim_{x \to 0} f(x) \neq f(0),
\end{align*}
og dermed er $f$ ikke kontinuert i $0$.
\end{enumerate}








