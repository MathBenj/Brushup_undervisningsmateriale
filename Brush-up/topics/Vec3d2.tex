\section{Planer i rummet}
\noindent Det næste vi vil studere er, hvordan man kan beskriver linjer og planer i rummet (i.e. i tre dimensioner). Vi vil betragte parameterfremstillingen for en linje i rummet, samt planens ligning.

Hvis vi får givet et fast punkt $A=(x_0,y_0,z_0)$ på den linje vi gerne vil bestemme samt en vektor $\vec{r}=\begin{bmatrix}r_1\\r_2\\r_3 \end{bmatrix}$ som er parallel med vores linje (en sådan vektor kaldes for en retningsvektor), så er parameterfremstillingen givet ved
\begin{align}\label{eq:vec3d2et}
\begin{bmatrix}x \\ y \\ z\end{bmatrix} = \begin{bmatrix}x_0 \\y_0 \\ z_0\end{bmatrix}  +t
\begin{bmatrix}r_1 \\r_2 \\ r_3 \end{bmatrix},
\end{align}
hvor $t \in \mathbb{R}$. Det skal forståes således at vi starter med et punkt $(x_0,y_0,z_0)$ på vores linje og så går vi i retningen af vores retningsvektor (som er parallel med vores linje) og dermed kan vi beskrive samtlige punkter på vores linje, ved at ændre på $t$, som bestemmer længden vi går.

\paragraph*{Eksempler:}
\begin{enumerate}
\item Lad $A=(2,2,2)$ og $\vec{r}= \begin{bmatrix}2 \\ 0 \\ 2\end{bmatrix}$ og bestem parameterfremstillingen for linjen:

Vi indsætter i \eqref{eq:vec3d2et} og får
\begin{align*}
\begin{bmatrix}x \\ y \\ z\end{bmatrix} = \begin{bmatrix}2 \\2 \\ 2\end{bmatrix}  +t
\begin{bmatrix}2 \\0 \\ 2 \end{bmatrix}.
\end{align*}
\item Find skæringspunkterne mellem kuglen $x^2+y^2+z^2=3$ og linjen beskrevet ved parameterfremstillingen
\begin{align*}
\begin{bmatrix}x \\ y \\ z \end{bmatrix} = \begin{bmatrix}2 \\2 \\ 2\end{bmatrix}  +t
\begin{bmatrix}-1 \\-1 \\ -1 \end{bmatrix}:
\end{align*}

Ud fra parameterfremstillingen får vi de tre ligninger
\begin{align*}
x &= 2 - t. \\
y &= 2 - t. \\
z &= 2 - t.
\end{align*}
Vi indsætter nu disse i kuglens ligning og får
\begin{align*}
3 = x^2+y^2+z^2 = (2-t)^2+(2-t)^2+(2-t)^2 = 3(2-t)^2=3t^2-12t+12.
\end{align*}
Det giver os andengradsligningen
\begin{align*}
3t^2-12t+9 = 0,
\end{align*}
som vi kan løse 
\begin{align*}
t = \frac{-b \pm \sqrt{b^2-4ac}}{2a}  = \frac{12 \pm \sqrt{36}}{6} = \frac{12 \pm 6}{6} = \begin{cases} 3 \\ 1 \end{cases}.
\end{align*}
\end{enumerate}
Ved at indsætte $t=3$ og $t=1$ i vores ligninger for $x$ og $y$ får vi de to skæringspunkter  $(-1,-1,-1)$ og $(1,1,1)$.


\paragraph*{Planens ligning:}
Hvis vi får givet et fast punkt $A=(x_0,y_0,z_0)$, som ligger på den plan vi gerne vil bestemme, samt en vektor $\vec{n}=\begin{bmatrix}a \\ b \\ c \end{bmatrix}$ som står vinkelret på planen (en sådan vektor kaldes for en normalvektor) så har vi for ethvert punkt $B=(x,y,z)$ der ligger på vores plan, at
\begin{align*}
\vec{n} \cdot \overrightarrow{AB} = 0 \qquad \Leftrightarrow \qquad \begin{bmatrix}a \\ b \\ c\end{bmatrix} \cdot \begin{bmatrix} x - x_0 \\ y - y_0 \\ z - z_0 \end{bmatrix} =0,
\end{align*}
da de to vektorer er ortogonale. Hvis vi udregner prikproduktet får vi ligningen
\begin{align}\label{eq:vec3d2to}
a(x-x_0) + b(y-y_0) + c(z-z_0) = 0,
\end{align}
som kaldes planens ligning i rummet.

Hvis vi får givet et punkt $A=(x_1,y_1,z_1)$ samt et plan $\alpha$ med ligning
\begin{align*}
ax+by+cz+d=0,
\end{align*}
så kan vi bestemme afstanden fra vores punkt til planen ud fra formlen
\begin{align}\label{eq:vec3d2afstandpunkttilplan}
\dist(A,\alpha)= \frac{\abs{ax_1+by_1+cz_1 + d}}{\sqrt{a^2+b^2+c^2}}.
\end{align}

\paragraph*{Eksempler:}
\begin{enumerate}
\item Lad $\vec{n}=\begin{bmatrix}0 \\ 1 \\ 2\end{bmatrix}$ og $A=(4,0,3)$ og bestem planens ligning:

Vi indsætter i \eqref{eq:vec3d2to} og får
\begin{align*}
0 \cdot (x - 4) + 1 \cdot (y-0)  + 2(z-3) \Leftrightarrow y + 2z - 6 = 0.
\end{align*}
\item Bestem afstanden fra punktet $A=(0,1,0)$ til plnanen $\alpha$ med ligning
\begin{align*}
2x+2y+z-9=0:
\end{align*}

Vi benytter \eqref{eq:vec3d2afstandpunkttilplan} og får 
\begin{align*}
\dist(A,\alpha)=\frac{\abs{2 \cdot 0 + 2 \cdot 1 + 1 \cdot 0 - 9}}{\sqrt{2^2+2^2+1^2}}=\frac{\abs{-7}}{\sqrt{9}} = \frac{7}{3}.
\end{align*}
\item Bestem skæringen mellem de to planer $\alpha$ og $\beta$ givet ved ligningerne
\begin{align*}
\alpha \colon x-3y+z-1 = 0 \qquad \textup{ og } \qquad \beta \colon 2x-5y-2z+4 = 0:
\end{align*}

Vi benytter de lige store koefficienters metode ved at tage $2$ gange ligningen for $\alpha$ og trække fra ligningen for $\beta$ og dernæst at tage $2$ gange ligningen for $\alpha$ og lægge til ligningen for $\beta$. Så får vi de to ligninger
\begin{align*}
y -4z + 6 = 0 \qquad \textup{ og } \qquad 4x - 11y + 2 = 0.
\end{align*}
Vi ser, at $y$ indgår i begge ligninger, så hvis vi lader $y=t$ og isolerer $z$ i den ene ligning samt $x$ i den anden ligning, så får vi
\begin{align*}
z = \frac{1}{4}t + \frac{3}{2} \qquad \textup{ og } \qquad x = \frac{11}{4}t - \frac{1}{2}.
\end{align*}
Det giver os at parameterfremstillingen for skæringslinjen er
\begin{align*}
\begin{bmatrix}x \\ y \\ z\end{bmatrix} = \begin{bmatrix} \frac{-1}{2} \\ 0 \\ \frac{3}{2} \end{bmatrix} + t \begin{bmatrix} \frac{11}{4} \\ 1 \\ \frac{1}{4} \end{bmatrix}.
\end{align*}
\end{enumerate}












