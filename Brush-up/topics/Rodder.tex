\section{Rødder}
\noindent Vi har tidligere betragtet potenser og ligesom at der for plus og gange findes regneoperationer der gør det modsatte, henholdsvis minus og divider, er der en invers (modsat) regneoperation af potenser, som kaldes rødder. 

Når vi snakker om rødder tænker vi på tal på formen
\begin{align*}
\mathrm{rod} = \sqrt[\mathrm{rodeksponent}]{\mathrm{grundtal}}.
\end{align*}


Vi siger at $y$ er den $n$'te rod af $x$ hvis vi kan opløfte $y$ i $n$ og få $x$, eller sagt på en anden måde
\begin{align*}
y=\sqrt[n]{x} \quad \textup{ hvis } \quad  y^n=x,
\end{align*}
hvor $x$ kan være alle tal og $n$ kan være alle positive heltal. Bemærk at dette betyder at hvis man opløfter et tal i $n$ og dernæst tager den $n$'te rod (eller i modsatte rækkefølge), så vil de to regneoperationer gå ud med hinanden. Vi vil kun betragte rødder med negativ grundtal hvis $n$ er ulige.

Kvadratrødderne af $a$ er de $b$ der opfylder at $b^2=a$. Det betyder at hvis $b$ er en kvadratrod så vil $-b$ også være en kvadratrod da $(-b)^2=b^2=a$. Hvis man bliver spurgt om at finde kvadratrodden af et tal, vil man medmindre der specifikt er angivet andet, altid nøjes med at give den positive værdi.

Hvis $n=2$ vil vi ofte bare skrive $\sqrt{a}$.
\paragraph*{Eksempler:}
\begin{enumerate}
\item Udregn $\sqrt{81}$:
\begin{align*}
\textup{Da } \quad 81=9^2 \quad \textup{ så er } \quad \sqrt{81}=9. 
\end{align*}
\item Udregn $\sqrt[3]{-8}$:
\begin{align*}
\textup{Da } \quad -8=(-2)^3 \quad \textup{ så er } \quad \sqrt[3]{-8}=-2. 
\end{align*}
\item Udregn $\sqrt[4]{81}$:
\begin{align*}
\textup{Da } \quad 81=3^4 \quad \textup{ så er } \quad \sqrt[4]{81}=3. 
\end{align*}
\end{enumerate}

\paragraph{Regneregler:}
Vi har følgende regneregler for rødder og igen er et utrolig vigtigt at man forstår at bruge disse, da de vil blive brugt igen og igen senere i kurset.

\begin{enumerate}
\item Vi har følgende sammenhæng mellem potenser og rødder:
\begin{align*}
\sqrt[n]{x}=x^{1/n}.
\end{align*}
\item Mere generelt har vi:
\begin{align*}
\sqrt[n]{x^m}=x^{\frac{m}{n}}=(\sqrt[n]{x})^m
\end{align*}
\item At gange to $n$'te rødder sammen er det samme som at tage den $n$'te rod af de to grundtal ganget sammen:
\begin{align*}
\sqrt[n]{x}\cdot \sqrt[n]{y}=\sqrt[n]{x\cdot y}.
\end{align*}
\item At dividere to $n$'te rødder med hinanden er det samme som at dividere de to grundtal og så tage den $n$'te rod:
\begin{align*}
\frac{\sqrt[n]{x}}{\sqrt[n]{y}}= \sqrt[n]{\frac{x}{y}}.
\end{align*}
\end{enumerate}

\paragraph*{Eksempler:}
\begin{enumerate}
\item Udregn $\sqrt[3]{5^6}$:
\begin{align*}
\sqrt[3]{5^6} = 5^\frac{6}{3} = 5^2 = 25.
\end{align*}
\item Udregn $\sqrt{144}$:
\begin{align*}
\sqrt{144}=\sqrt{9 \cdot 16} = \sqrt{9}\cdot \sqrt{16} = 3 \cdot 4 = 12.
\end{align*}
\item Udregn $\sqrt{\frac{144}{81}}$:
\begin{align*}
\sqrt{\frac{144}{81}} = \frac{\sqrt{144}}{\sqrt{81}}=\frac{12}{9}=\frac{4}{3}.
\end{align*}
\item Reducer $\sqrt[3]{\frac{(\sqrt{ab})^6}{b^3}}$:
\begin{align*}
\sqrt[3]{\frac{(\sqrt{ab})^6}{b^3}} = \sqrt[3]{\frac{(\sqrt{ab})^{2\cdot 3}}{b^3}} = \sqrt[3]{\frac{\big((\sqrt{ab})^2 \big)^3}{b^3}} = \sqrt[3]{\frac{(ab)^3}{b^3}}= \sqrt[3]{\frac{a^3b^3}{b^3}} = \sqrt[3]{a^3}=a.
\end{align*}
\end{enumerate}




