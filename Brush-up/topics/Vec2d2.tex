\section{Linjer i planen}
\noindent Det næste vi vil studere er, hvordan man kan beskrive linjer i planen (i.e. i to dimensioner). Vi vil betragte to metoder, kaldet linjens ligning og parameterfremstillingen for en linje.

Vi starter med at studere linjens ligning. Hvis vi får givet et fast punkt $A=(x_0,y_0)$ som ligger på den linje vi gerne vil bestemme, samt en vektor $\vec{n}=\begin{bmatrix}a \\ b\end{bmatrix}$ som står vinkelret på linjen (en sådan vektor kaldes for en normalvektor) så har vi for ethvert punkt $B=(x,y)$ der ligger på vores linje, at
\begin{align*}
\vec{n} \cdot \overrightarrow{AB} = 0 \qquad \Leftrightarrow \qquad \begin{bmatrix}a \\ b\end{bmatrix} \cdot \begin{bmatrix} x - x_0 \\ y - y_0 \end{bmatrix} =0,
\end{align*}
da de to vektorer er ortogonale. Hvis vi udregner prikproduktet får vi ligningen
\begin{align}\label{eq:vec2d2et}
a(x-x_0) + b(y-y_0) = 0,
\end{align}
som kaldes linjens ligning i planen.

\paragraph*{Eksempler:}
\begin{enumerate}
\item Lad $\vec{n}=\begin{bmatrix}0 \\ 1\end{bmatrix}$ og $A=(4,0)$ og bestem linjens ligning:

Vi indsætter i \eqref{eq:vec2d2et} og får
\begin{align*}
0 \cdot (x - 4) + 1 \cdot (y-0) = 0 \Leftrightarrow y=0,
\end{align*}
hvilket viser at vores linje er $x$-aksen i et koordinatsystem.
\item Bestem linjens ligning for den linje der går gennem punkterne $A=(1,1)$ og $B=(2,3)$ og bestem om punktet $(-1,-1)$ ligger på linjen:

Da vi ikke er givet nogen normalvektor, starter vi med at bestemme sådan en. Vi ser at vektoren
\begin{align*}
\overrightarrow{AB} = \begin{bmatrix} 2-1 \\ 3-1 \end{bmatrix}= \begin{bmatrix} 1 \\ 2 \end{bmatrix},
\end{align*}
ligger på linjen. Vi finder nu en normalvektor ved at tage hatvektoren til $\overrightarrow{AB}$, hvilket giver
\begin{align*}
\vec{n}=\hat{\overrightarrow{AB}}=\begin{bmatrix} -2 \\ 1 \end{bmatrix}.
\end{align*}
Indsætter vi nu $\vec{n}$ og punktet $A$ i \eqref{eq:vec2d2et} får vi linjens ligning
\begin{align*}
-2\cdot (x-1)+1 \cdot (y-1) = 0 &\Leftrightarrow -2x +2 +y -1 = 0 \\
-2x+y+1=0.
\end{align*}
Bemærk, at vi kunne have benyttet punktet $B$ i stedet for $A$. Vi tjekker nu om punktet $(-1,-1)$ løser ligningen
\begin{align*}
-2 \cdot(-1) + (-1) +1 = 2 - 1+1 = 2,
\end{align*}
hvilket viser at punktet $(-1,-1)$ ikke ligger på linjen.
\end{enumerate}

\paragraph*{Parameterfremstilling:}
En anden måde at beskrive en linje i planen er ved parameterfremstillingen. Hvis vi får givet et fast punkt $A=(x_0,y_0)$ på den linje vi gerne vil bestemme samt en vektor $\vec{r}=\begin{bmatrix}r_1\\r_2 \end{bmatrix}$ som er parallel med vores linje (en sådan vektor kaldes for en retningsvektor), så er parameterfremstillingen givet ved
\begin{align}\label{eq:vec2d2to}
\begin{bmatrix}x \\ y\end{bmatrix} = \begin{bmatrix}x_0 \\y_0\end{bmatrix}  +t
\begin{bmatrix}r_1 \\r_2 \end{bmatrix},
\end{align}
hvor $t \in \mathbb{R}$. Det skal forstås således at vi starter med et punkt $(x_0,y_0)$ på vores linje og så går vi i retningen af vores retningsvektor (som er parallel med vores linje) og dermed kan vi beskrive samtlige punkter på vores linje, ved at skifte på $t$, som bestemmer længden vi går.

Bemærk, at i linjens ligning bruger vi en vektor der står vinkelret på linjen, mens vi i parameterfremstillingen bruger en vektor der er parallel med linjen.

\paragraph*{Eksempler:}
\begin{enumerate}
\item Lad $A=(2,2)$ og $\vec{r}= \begin{bmatrix}2 \\ 0\end{bmatrix}$ og bestem parameterfremstillingen for linjen:

Vi indsætter i \eqref{eq:vec2d2to} og får
\begin{align*}
\begin{bmatrix}x \\ y\end{bmatrix} = \begin{bmatrix}2 \\2\end{bmatrix}  +t
\begin{bmatrix}2 \\0 \end{bmatrix}.
\end{align*}
\item Bestem parameterfremstillingen for linjen der går gennem punkterne $A=(3,4)$ og $B=(8,1)$:

Vi bestemmer først en retningsvektor 
\begin{align*}
\vec{r}=\overrightarrow{AB} =  \begin{bmatrix}8-3 \\ 1-4\end{bmatrix} =\begin{bmatrix}5 \\ -3\end{bmatrix}.
\end{align*}
Vi indsætter nu $\vec{r}$ og $A$ i \eqref{eq:vec2d2to} og får
\begin{align*}
\begin{bmatrix}x \\ y\end{bmatrix} = \begin{bmatrix}3 \\4\end{bmatrix}  +t
\begin{bmatrix}5 \\-3 \end{bmatrix}.
\end{align*}
\item Find skæringspunkterne mellem cirklen $x^2+y^2=2$ og linjen beskrevet ved parameterfremstillingen
\begin{align*}
\begin{bmatrix}x \\ y\end{bmatrix} = \begin{bmatrix}2 \\2\end{bmatrix}  +t
\begin{bmatrix}-1 \\-1 \end{bmatrix}:
\end{align*}

Ud fra parameterfremstillingen får vi de to ligninger
\begin{align*}
x &= 2 -t. \\
y &= 2 -t.
\end{align*}
Vi indsætter nu disse i cirklens ligning og får
\begin{align*}
2 = x^2+y^2 = (2-t)^2+(2-t)^2 = 2(2-t)^2 = 2( 4 +t^2 -4t) = 2t^2-8t+8.
\end{align*}
Det giver os andengradsligningen
\begin{align*}
2t^2-8t+6 = 0,
\end{align*}
som vi kan løse 
\begin{align*}
t = \frac{-b \pm \sqrt{b^2-4ac}}{2a} = \frac{8 \pm \sqrt{(-8)^2-4\cdot 2 \cdot 6}}{2 \cdot 2} = \frac{8 \pm \sqrt{16}}{4} = \frac{8 \pm 4}{4} = \begin{cases} 3 \\ 1 \end{cases}.
\end{align*}
\end{enumerate}
Ved at indsætte $t=3$ og $t=1$ i vores ligninger for $x$ og $y$ får vi de to skæringspunkter  $(-1,-1)$ og $(1,1)$.












