\section{Ligninger og uligheder}
\noindent En ligning består af to udtryk, en højre- og en venstreside, hvor mindst en af siderne indeholder en ubekendt variabel som vi gerne vil bestemme, samt et lighedstegn der binder de to sider sammen. Vores mål med ligninger er at bestemme alle de tal, som når vi sætter dem ind på den ubekendte variables plads gør at venstre siden er lig med højre siden. Dette kaldes også at løse ligningen.

\paragraph*{Eksempel:}
\begin{enumerate}
\item Løs ligningen $x + 2=7$:

Vi ser at $x=5$ er den eneste løsning til ligningen.
\item Løs ligningen $x=\sqrt{-2}$:

Vi husker fra tidligere kursusgange at dette er ensbetydende med at finde de $x$ der opfylder at $x^2=-2$, men da alle tal opløftet i anden giver noget positivt, har denne ligning ingen reelle løsning.
\item Løs ligningen $x^2 = 9$:

Vi ser at $x=3$ og $x=-3$ er de eneste løsninger til ligningen.

\item Løs ligningen $x+1 = \frac{1}{2}(2x+2)$:

Hvis vi reducere højresiden får vi
\begin{align*}
\frac{1}{2}(2x+2) = x+1,
\end{align*}
hvilket er sandt for alle $x$. Denne ligning har derfor uendeligt mange løsninger.
\end{enumerate}
Disse eksempler er alle meget ligetil og vi kan ved at betragte dem aflæse løsningerne. Det er dog ikke altid muligt og i disse tilfælde vil vi gerne kunne reducere vores ligning til en ny ligning med præcis samme løsningsmængde, men hvor vi nemt kan aflæse løsningerne. Sådanne ligninger kaldes ækvivalente ligninger og noteres med en biimplikationspil $\Leftrightarrow$.

\paragraph*{Regneregler:}
Når vi reducerer vores højre- og venstreside kan vi tænke på det som at de to er lig med hinanden. Det betyder at hvis vi gør noget på den ene side, så for at beholde ligheden er vi nød til at gøre præcis det samme på den anden side.
\begin{enumerate}
\item Vi må lægge tal til og trække fra, så længe vi gør det samme på begge sider af lighedstegnet, f.eks.:
\begin{align*}
a + x = b \Leftrightarrow a+x \pm c = b \pm c, 
\end{align*}
hvor $a,b,c$ kan være alle tal.
\item Vi må gange begge sider med det samme tal, med undtagelse af $0$, f.eks.:
\begin{align*}
a + x = b \Leftrightarrow (a + x)c = bc \Leftrightarrow ac + xc = bc. 
\end{align*}
\item Vi må dividere begge sider med det samme tal, f.eks.:
\begin{align*}
a + x = b \Leftrightarrow \frac{a+x}{c} = \frac{b}{c} \Leftrightarrow \frac{a}{c}  + \frac{x}{c} = \frac{b}{c}.
\end{align*}
\end{enumerate}
Når vi løser ligninger vil vi gerne samle de ubekendte variable på den ene side, så vores reducerede ligning kommer til at ligne de nemme eksempler fra tidligere.

\paragraph*{Eksempler:}
\begin{enumerate}
\item Løs ligningen $4x+7 = 3(x+8)$:

Vi ganger først ind i parentesen på højresiden og derefter isolerer vi $x$ på venstresiden:
\begin{align*}
4x+7=3(x+8)= 3x+24 &\Leftrightarrow 4x+7-7 = 3x+24-7 \\
&\Leftrightarrow 4x = 3x + 17 \\
&\Leftrightarrow 4x - 3x = 3x + 17 - 3x \\
&\Leftrightarrow x = 17.
\end{align*}
\item Løs ligningen $\frac{2x+1}{4x}=3$:

Vi ganger først med nævneren på venstresiden for at få brøken væk og isolerer derefter $x$ på højresiden:
\begin{align*}
\frac{2x+1}{4x}=3 &\Leftrightarrow \frac{2x+1}{4x} \cdot 4x = 3 \cdot 4x \\
&\Leftrightarrow 2x+1 = 12x \\
&\Leftrightarrow 2x +1 - 2x = 12x - 2x \\
&\Leftrightarrow 1 = 10x\\
&\Leftrightarrow \frac{1}{10} = \frac{10x}{10} \\
&\Leftrightarrow \frac{1}{10} = x. 
\end{align*}
\item Løs ligningen $\pi x = 3-2x$:

Vi isolere $x$ på venstresiden og trækker $x$ udenfor en parentes:
\begin{align*}
\pi x = 3-2x &\Leftrightarrow \pi x +2x = 3 - 2x + 2x \\
&\Leftrightarrow \pi x + 2x = 3 \\
&\Leftrightarrow x(\pi+2) = 3 \\
&\Leftrightarrow \frac{x(\pi + 2)}{\pi + 2} = \frac{3}{\pi +2} \\
&\Leftrightarrow x = \frac{3 }{\pi +2}.
\end{align*}
\end{enumerate}

\paragraph*{Nulreglen:}
Det er klart at hvis vi har to tal $a,b$ hvor ingen af dem er $0$, så giver $a \cdot b$ også noget der er forskelligt fra $0$. Det betyder at hvis vi har en ligning hvor to tal ganget sammen skal give $0$, så må det ene af de to tal være $0$. Det kan vi også skrive mere matematisk:
\begin{align*}
\textup{Hvis }a \cdot b = 0, \textup{ så er } a=0 \textup{ eller } b=0.
\end{align*}
Denne regel kaldes nulreglen og den er ekstremt nyttig når man skal løse ligninger hvor man kan trække den ubekendte variabel udenfor en parentes.

\paragraph*{Eksempler:}
\begin{enumerate}
\item Løs ligningen $2x^2+3x=0$:

Vi trækker $x$ udenfor en parentes:
\begin{align*}
2x^2+3x=0 &\Leftrightarrow x(2x+3)=0.
\end{align*}
Ved at bruge nulreglen ser vi nu at løsningerne er $x=0$ eller $2x+3=0 \Leftrightarrow x = \frac{-3}{2}$.
\end{enumerate}

\paragraph*{Uligheder}
Hvis vi erstatter lighedstegnet i en ligning med et ulighedstegn får vi en ulighed. Ligesom med ligninger vil vi gerne bestemme alle de tal vi kan sætte ind på den ubekendtes plads i vores ulighed, så den er sand. Det at løse en ulighed minder meget om at løse ligninger, men med den forskel at hvis vi ganger med et negativt tal på begge sider, så skal vi vende ulighedstegnet om. Det giver god mening da hvis vi f.eks. har uligheden $4 \leq 5$ og vi trækker først $5$ fra på begge sider og dernæst trækker $4$ fra på begge sider, så får vi
\begin{align*}
4 \leq 5 &\Leftrightarrow 4 - 5 \leq 5 - 5 \\
&\Leftrightarrow 4 - 5 - 4 \leq 5 - 5 - 4 \\
&\Leftrightarrow -5 \leq -4. 
\end{align*} 

\paragraph*{Regneregler:} 
Det giver os følgende regneregler til at løse uligheder:
\begin{enumerate}
\item Vi må lægge tal til og trække fra, så længe vi gør det samme på begge sider af ulighedstegnet, f.eks.:
\begin{align*}
a + x \leq b \Leftrightarrow a+x \pm c \leq b \pm c, 
\end{align*}
hvor $a,b,c$ kan være alle tal.
\item Vi må gange begge sider med det samme positive tal, f.eks.:
\begin{align*}
a + x \leq b \Leftrightarrow (a + x)c \leq bc \Leftrightarrow ac + xc \leq bc, 
\end{align*}
hvor $c$ er et positivt tal.
\item Vi må gange begge sider med det samme negative tal, hvis vi vender ulighedstegnet, f.eks.:
\begin{align*}
a + x \leq b \Leftrightarrow (a + x)d \geq bd \Leftrightarrow ad + xd \geq bd, 
\end{align*}
hvor $d$ er et negativt tal.
\item Vi må dividere begge sider med det samme positive tal, f.eks.:
\begin{align*}
a + x \leq b \Leftrightarrow \frac{a+x}{c} \leq \frac{b}{c} \Leftrightarrow \frac{a}{c}  + \frac{x}{c} \leq \frac{b}{c},
\end{align*}
hvor $c$ er et positivt tal.
\item Vi må dividere begge sider med det samme negative tal, hvis vi vender ulighedstegnet, f.eks.:
\begin{align*}
a + x \leq b \Leftrightarrow \frac{a+x}{d} \geq \frac{b}{d} \Leftrightarrow \frac{a}{c}  + \frac{x}{d} \geq \frac{b}{d},
\end{align*}
hvor $d$ er et negativt tal.
\end{enumerate}

\paragraph*{Eksempler:}
\begin{enumerate}
\item Løs uligheden $4+x \leq 5$:

Vi isolerer $x$ på venstresiden og får at $x \leq 1$.
\item Løs uligheden $-2x+4 \leq 3x$:

Vi isolerer $x$ på højresiden og får at $4 \leq 5x \Leftrightarrow \frac{4}{5} \leq x$.
\end{enumerate}


