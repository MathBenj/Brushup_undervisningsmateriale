\section{Begyndelsesværdiproblemer}
\noindent Sidste gang studerede vi fuldstændige løsninger til differentialligninger. Vi så at der var uendeligt mange løsninger til en differentialligning og vi kaldte hver af disse løsninger for en partikulær løsning. Denne gang vil vi studere differentialligninger, der opfylder en såkaldt begyndelsesbetingelse. 

Hvis man bliver bedt om at finde en funktion $f(x)$ og den eneste ledetråd man får, er at $f(x)$ er en partikulær løsning til differentialligningen
\begin{align*}
\frac{d}{dx}f(x)=5f(x),
\end{align*}
kan man så finde $f(x)$? Fra sidste gang, ved vi at den fuldstændige løsning til differentialligningen er på formen 
\begin{align*}
f(x)=ce^{5x},
\end{align*}
men vi har ingen mulighed for at finde ud af hvad $c$ er. Hvis man derudover får at vide at funktionen opfylder at $f(0)=10$, så kan man bestemme hvad $c$ er. Hvis vi indsætter, at $f(0)=10$ i den fuldstændige løsning får vi at
\begin{align*}
10=f(0)=ce^{5\cdot 0} = c e^0 = c,
\end{align*}
hvilket betyder, at den partikulære løsning jeg tænkte på må være
\begin{align*}
f(x)=10e^{5x}.
\end{align*}
Betingelsen $f(0)=10$ kaldes for en begyndelsesbetingelse og noteres generelt ved
\begin{align*}
f(t_0)=y_0.
\end{align*}
Det betyder, at hvis vi har en differentialligning, hvor løsningen skal opfylde at den går igennem punktet $(t_0,y_0)$, så findes der præcis én løsning. Det at løse en differentialligning med begyndelsesbetingelse kaldes ofte for et begyndelsesværdiproblem.

\paragraph*{Eksempler:}
\begin{enumerate}
\item Løs begyndelsesværdiproblemet $f'(x)=5$ med $f(2)=4$:

Vi finder først den fuldstændige løsning til differentialligningen $f'(x)=5$. Ved at benytte tabelen fra sidste gang ser vi at differentialligningen er på formen $f'(x)=k$ og har dermed den fuldstændige løsning
\begin{align*}
f(x)=5x+c.
\end{align*}
Vi indsætter nu vores begyndelsesbetingelse og bestemmer $c$
\begin{align*}
4=f(2)=5 \cdot 2 + c = 10 + c \Leftrightarrow c = -6.
\end{align*}
Det betyder at den løsning vi leder efter er $f(x)=5x-6$.

\item Vis at  $f(x)=ce^{x^2}$ er en løsning til differentialligningen $f'(x)=2xf(x)$ og bestem den partikulære løsning der opfylder at $f(0)=3$:

Vi viser, at $f(x)=ce^{x^2}$ er en løsning ved at gøre prøve. Vi har, at venstresiden i vores differentialligning giver
\begin{align*}
f'(x) = \frac{d}{dx} \big(ce^{x^2}\big) = 2xce^{x^2},
\end{align*}
hvor vi har brugt kædereglen til at differentiere. Derudover har vi, at højresiden er
\begin{align*}
2xf(x)=2xce^{x^2}.
\end{align*}
Da venstresiden er lig højresiden er $f(x)=ce^{x^2}$ en løsning. Vi bestemmer nu $c$ ved at benytte begyndelsesbetingelsen
\begin{align*}
3=f(0)=ce^{0^2}=ce^0=c.
\end{align*}
Dermed er den partikulære løsning vi leder efter $f(x)=3e^{x^2}$.
\end{enumerate}

\paragraph*{Tangentligningen:}
Vi har tidligere studeret tangentens ligning
\begin{align*}
f(x)=f'(x_0)(x-x_0)+f(x_0).
\end{align*}
Hvis vi er givet en differentialligning, f.eks.
\begin{align*}
f'(x)=x^2-f(x),
\end{align*}
og vi gerne vil bestemme tangentens ligning for en løsning $f$ i punktet $(3,7)$, så har vi at $x_0=3$ og $f(x_0)=7$. Det betyder at vi kun mangler at bestemme $f'(x_0)$. Hvis vi indsætter $x_0=3$ i vores differentialligning får vi at
\begin{align*}
f'(3)=3^2-f(3)=9-7=2.
\end{align*}
Hvilket giver at tangentens ligning i punktet $(3,7)$ er givet ved
\begin{align*}
f(x)=2(x-3)+7 = 2x -6 + 7 = 2x+1.
\end{align*}
Bemærk, at det ikke er nødvendigt at løse differentialligningen for at finde tangentens ligning til en løsning gennem et givet punkt, såfremt man kender $x_0$ og $f(x_0)$.

\paragraph*{Eksempel:}
\begin{enumerate}
\item Lad $f$ være en funktion der løser differentialligningen $\displaystyle \frac{d}{dx}f(x) = \frac{2f(x)-1}{x}$ med begyndelsesbetingelse $f(1)=3$. Bestem tangentens ligning til $f$ i punktet $(1,3)$:

Vi har at $x_0=1$ og $f(x_0)=3$. Vi finder dernæst $f'(x_0)$ ved
\begin{align*}
f'(x_0) = \frac{2f(1)-1}{1} = \frac{2 \cdot 3 - 1}{1} = \frac{5}{1} = 5.
\end{align*}
Dermed har vi, at tangentens ligning i punktet $(1,3)$ er givet ved
\begin{align*}
f(x)=f'(x_0)(x-x_0)+f(x_0)=5(x-1)+3 = 5x-5+3 = 5x-2.
\end{align*}
\end{enumerate}










