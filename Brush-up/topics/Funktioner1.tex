\section{Funktioner: Injektivitet, surjektivitet, summer og produkter}
Vi vil nu se nærmere på hvad en funktion egentlig er, for at gøre dette starter vi med at kigge kort på mængder. Lad $X$ og $Y$ være to mængder, det kan f.eks. være et interval som $[0,1]$ eller $(0,1)$, der henholdsvis består af alle tal der opfylder $0 \leq x \leq 1$ og $0 < x < 1$, eller en endelig mængde $\{a,b,c,d\}$. Hvis vi lader $X = \{1,2,3,4\}$ så siger vi f.eks. at $2$ ligger i $X$ og notere det med $2 \in X$ mens vi siger at $5$ ikke ligger i $X$ hvilket vi notere $5 \not\in X$. Hvis vi vil fjerne et element i en mængde skriver vi f.eks. $X \setminus 3 = \{1,2,4\}$. Derudover kan vi også tage en delmængde af en allerede givet mængde, hvilket vi notere f.eks. med $\{1,2\} \subset X$. For at simplificere vores notation vil vi ofte skrive intervallet $(-\infty,\infty)$ som $\mathbb{R}$ og kalde det for de reelle tal. 



Vi siger at $f$ er en funktion der går mellem $X$ og $Y$, skrevet $f \colon X \to Y$, hvis $f(x)$ giver præcis et element i $Y$ for alle $x \in X$. Vi kalder $X$ for domænet (også kaldet definitionsmængden) af $f$ og $Y$ for codomænet af $f$. Bemærk, at det betyder at hvis $f$ sender et element fra $X$ over i flere forskellige elementer i $Y$ så er $f$ ikke en funktion, men en funktion kan godt sende flere elementer fra $X$ over i det samme element i $Y$.

\paragraph*{Eksempler:}
\begin{enumerate}
\item Lad $f \colon \{1,2,3\} \to \{1,2,3,4,5,6\}$ være givet ved $f(x)=2x$, så er $f$ en funktion da ethvert element i $X$ bliver sendt over i præcis et element af $Y$. Bemærk, at vi behøver ikke ramme alle elementer i $Y$. 
\item Lad $f \colon \mathbb{R} \to \mathbb{R}$ være givet ved $f(x)=x^2$ så er $f$ en funktion.
\item Lad $X=\{a,b,c,d\}$ og $Y=\{1,2,\pi,\mathrm{abe}\}$ og bestem en funktion $f\colon X \to Y$. Det eneste der skal gælde for en funktion er, at den tager ethvert element i sit domæne og sender over i præcis et element i codomænet. Det  betyder at en mulig funktion $f$ er givet ved, $f(a)=1$, $f(b)=\pi$, $f(c)=\mathrm{abe}$  og $f(d)=2$. Dette kan også skrives som en ``gaffel funktion'' på følgende måde:
\begin{align*}
f (x) = \begin{cases} 
1 & x=a \\
\pi & x= b\\
\mathrm{abe} & x=c\\
2 & x=d
\end{cases}.
\end{align*}
Bemærk, at hvis vi f.eks. havde sat både $f(a)=1$ og $f(a)=2$ så havde $f$ ikke været en funktion!
\end{enumerate}

\paragraph*{Injektiv og surjektiv:}
Hvis $f \colon X \to Y$ er en funktion, så kalder vi alle de elementer i $Y$ som bliver ramt af $f$ for værdimængden af $f$. Bemærk, at vi på intet tidspunkt har sagt at vi skal ramme alle elementer i $Y$, derfor er værdimængden af $f$ en delmængde af codomænet for $f$. Vi siger derfor at en funktion $f$ er surjektiv hvis der gælder at værdimængden og codomænet for $f$ er den samme mængde, som det er tilfældet i Figur~\ref{fig:funktioner1etlec}.

Derudover har vi at hvis en funktion opfylder at der ikke er to forskellige elementer i $X$ der bliver sendt over i det samme element i $Y$, også skrevet
\begin{align*}
f(x_1) = y \textup{ og } f(x_2) =y \quad  \Rightarrow \quad x_1=x_2, 
\end{align*}
så siges funktionen at være injektiv, som f.eks. i Figur~\ref{fig:funktioner1tolec}. 

Hvis en funktion er både injektiv og surjektiv, så kalder vi den for en bijektiv funktion.

\begin{figure}[!htbp]
\begin{minipage}{0.49\textwidth}
  \pgfplotsset{width=0.5\textwidth,compat=1.11}
  \centering
  \begin{tikzpicture}
  \draw \boundellipse{0,0}{0.7}{1.4};
  \draw \boundellipse{3,0}{0.7}{1.4};
  \node[circle,fill,inner sep=1pt] at (0.05,1.0) [label=left:$a$]{};
  \node[circle,fill,inner sep=1pt] at (0.05,0.35) [label=left:$b$]{};
  \node[circle,fill,inner sep=1pt] at (0.05,-0.35) [label=left:$c$]{};
  \node[circle,fill,inner sep=1pt] at (0.05,-1.0) [label=left:$d$]{}; 
  \node[] at (0.45,1.7) [label=left:$X$]{}; 
  \node[circle,fill,inner sep=1pt] at (3.0,0.7) [label=right:]{};
  \node[circle,fill,inner sep=1pt] at (3.0,0.0) [label=right:]{};
  \node[circle,fill,inner sep=1pt] at (3.0,-0.7) [label=right:]{}; 
  \node[] at (3.45,1.7) [label=left:$Y$]{}; 
  \node[] at (1.9,1.3) [label=left:$g$]{}; 
  \draw[->,thick] (0.2,1.0) -- (2.8,0.07);
  \draw[->,thick] (0.2,0.35) -- (2.8,0.7);
  \draw[->,thick] (0.2,-0.35) -- (2.8,-0.07);
  \draw[->,thick] (0.2,-1.0) -- (2.8,-0.7);
 \end{tikzpicture}
  \caption{En surjektiv funktion}
  \label{fig:funktioner1etlec}
\end{minipage}
\begin{minipage}{0.49\textwidth}
\centering
  \pgfplotsset{width=0.5\textwidth,compat=1.11}
  \centering
  \begin{tikzpicture}
  \draw \boundellipse{0,0}{0.7}{1.4};
  \draw \boundellipse{3,0}{0.7}{1.4};
  \node[circle,fill,inner sep=1pt] at (0.05,0.7) [label=left:$a$]{};
  \node[circle,fill,inner sep=1pt] at (0.05,0.0) [label=left:$b$]{};
  \node[circle,fill,inner sep=1pt] at (0.05,-0.7) [label=left:$c$]{}; 
  \node[] at (0.45,1.7) [label=left:$X$]{}; 
  \node[circle,fill,inner sep=1pt] at (3.0,1.0) [label=right:]{};
  \node[circle,fill,inner sep=1pt] at (3.0,0.35) [label=right:]{};
  \node[circle,fill,inner sep=1pt] at (3.0,-0.35) [label=right:]{};
  \node[circle,fill,inner sep=1pt] at (3.0,-1.0) [label=right:]{}; 
  \node[] at (3.45,1.7) [label=left:$Y$]{}; 
  \node[] at (1.9,1.3) [label=left:$f$]{}; 
  \draw[->,thick] (0.2,0.7) -- (2.8,1);
  \draw[->,thick] (0.2,0.0) -- (2.8,-1.0);
  \draw[->,thick] (0.2,-0.7) -- (2.8,0.35);
 \end{tikzpicture}
  \caption{En injektiv funktion}
  \label{fig:funktioner1tolec}
\end{minipage}
\end{figure}

\paragraph*{Eksempler:}
\begin{enumerate}
\item Hvis $X=\{1,2,3,4\}$ og $Y= \{3,4,5,6,7\}$ så er funktionen $f \colon X \to Y$ givet ved $f(x)=x+2$ injektiv da ethvert $x \in X$ bliver sendt over i forskellige $y \in Y$ men den er ikke surjektiv, da $7$ ikke bliver ramt af noget $x \in X$.
\item Vi betragtede tidligere funktionen $f \colon \mathbb{R} \to \mathbb{R}$ givet ved $x^2$, denne funktion er hverken injektiv eller surjektiv da $-x$ og $+x$ bliver sent i det samme $y$, og vi rammer ikke hele $\mathbb{R}$ da $f(x)$ altid er positiv. Ved at ændre på domænet og codomænet kan vi gøre funktionen henholdsvis injektiv eller surjektiv. F.eks. er funktionen $f \colon [0,\infty) \to \mathbb{R}$ injektiv og $f \colon \mathbb{R} \to [0,\infty)$ er surjektiv. Vi ser endvidere at hvis vi har $f \colon [0,\infty) \to [0,\infty)$ så er $f$ endda bijektiv.
\end{enumerate}

\paragraph*{Sum og produkt af funktioner:}
Ligesom at vi kan lægge tal sammen og gange tal sammen, kan vi også gøre det samme med funktioner.

\paragraph*{Regneregler:}
Hvis $f$ og $g$ er funktioner, så har vi at
\begin{enumerate}
\item Summen af $f$ og $g$ evalueret i $x$ er det samme som at evaluere de to funktioner i $x$ og så lægge dem sammen:
\begin{align*}
(f\pm g)(x)=f(x) \pm g(x).
\end{align*}
\item Produktet af $f$ og $g$ evalueret i $x$ er det samme som at evaluere de to funktioner i $x$ og så gange dem sammen: 
\begin{align*}
(f \cdot g)(x)=f(x) \cdot g(x).
\end{align*}
\item At dividere funktionerne $f$ og $g$ og så evaluere i $x$ er det samme som at evaluere de to funktioner i $x$ og så dividere dem bagefter, såfremt $g(x)\neq 0$:
\begin{align*}
\Big(\frac{f}{g}\Big) (x) = \frac{f(x)}{g(x)},
\end{align*}
hvor $g(x) \neq 0$.
\end{enumerate}

\paragraph*{Eksempler:}
\begin{enumerate}
\item Hvis $f(x)=3x^2+1$ og $g(x)=\frac{1}{x}$ hvad er $(f+g)(2)$ så:

Vi udregner først $f(2)$ og $g(2)$:
\begin{align*}
f(2) &= 3 \cdot 2^2 + 1 = 3 \cdot 4 + 1 = 13, \\
g(2) &= \frac{1}{2},
\end{align*}
hvilket betyder at
\begin{align*}
(f+g)(2)=f(2)+g(2) = 13 + \frac{1}{2} = \frac{27}{2}.
\end{align*}
\item Hvis $f(x)=2x$ og $g(x)=\frac{1}{x}$ hvad er $\big(\frac{f}{g}\big)(3)$ så:

Vi udregner først $f(3)$ og $g(3)$:
\begin{align*}
f(3) &= 2 \cdot 3 = 6,\\
g(3) &= \frac{1}{3},
\end{align*}
hvilket medfører at 
\begin{align*}
\Big( \frac{f}{g}\Big) (3) = \frac{f(3)}{g(3)} = \frac{6}{\frac{1}{3}} = 6 \cdot 3 = 18.
\end{align*}
\end{enumerate}