\section{Brøker}
\noindent Når vi snakker om brøker tænker vi på tal på formen
\begin{align*}
\textup{brøk } = \frac{\textup{ tæller }}{ \textup{ nævner }},
\end{align*}
hvor det eneste krav er at nævneren ikke må være $0$. 

Vi vil ofte skrive en brøk som
\begin{align*}
a=\frac{b}{c},
\end{align*}
hvor $b$ kan være et hvilket som helst tal og $c$ kan være alle tal bortset fra $0$.

Det kan f.eks. være $5$ venner der vælger at dele en pizza ligeligt imellem sig, så de får $\frac{1}{5}$ hver, eller $3$ venner der har tjent 200 kr og hver især får $\frac{200}{3}$ kr.

\paragraph*{Størrelsesforhold:} 
Der gælder at hvis nævneren i en brøk er større end tælleren, så er brøken mindre end $1$, f.eks. $\frac{1}{2} < 1$. Hvis nævneren er lig med tælleren så er brøken lig med $1$, f.eks. $\frac{4}{4}=1$ og hvis nævneren er mindre end tælleren så er brøken større end $1$, f.eks. $\frac{4}{3} > 1$.

En anden måde vi kan skrive dette på er:
\begin{align*}
&\frac{a}{b} < 1, \quad \textup{ hvis }a < b,\\
&\frac{a}{b} = 1, \quad \textup{ hvis }a = b,\\
&\frac{a}{b} > 1, \quad \textup{ hvis }a > b.
\end{align*}
Derudover gælder der, at hvis man gør nævneren større så får man en mindre brøk, eksempelvis $\frac{2}{3} > \frac{2}{5}$, hvorimod hvis man gør tælleren større så får man en større brøk, som i tilfældet $\frac{3}{4} < \frac{7}{4}$.

Igen kan vi skrive dette mere præcist ved:
\begin{align*}
\frac{a}{b} < \frac{a}{c}, \quad \textup{ hvis }c < b,\\
\frac{a}{b} < \frac{c}{b}, \quad \textup{ hvis }a < c.
\end{align*}

\paragraph*{Regneregler:}
Det er ekstremt vigtigt at man lærer de følgende regneregler for brøker, da de vil dukke op igen og igen resten af kurset og i jeres videre studieforløb.
\begin{enumerate}
\item Når vi lægger to brøker sammen eller trækker dem fra hinanden, finder vi først fælles nævner og derefter lægger vi tællerne sammen eller trækker dem fra hinanden. 

En måde at finde fælles nævner, som altid virker, er at gange på kryds:
\begin{align*}
\frac{a}{b}\pm \frac{c}{d}= \frac{a \cdot d}{b \cdot d} \pm \frac{b \cdot c}{b \cdot d} = \frac{a \cdot d \pm b \cdot c}{b \cdot d}.
\end{align*}
\item Når vi ganger et tal med en brøk, ganger vi tallet op i tælleren:
\begin{align*}
a \cdot \frac{b}{c}=\frac{a \cdot b}{c}.
\end{align*}
\item Når vi dividerer en brøk med et tal, ganger vi tallet ned i nævneren:
\begin{align*}
\frac{\frac{a}{b}}{c}=\frac{a}{b\cdot c}.
\end{align*}
Bemærk, at brøkstregen i den brøk der står i tælleren er mindre end den anden brøkstreg. Grunden til det er at så kan vi se forskel på om vi dividere en brøk med et tal eller et tal med en brøk. 
\item Når vi dividerer et tal med en brøk, ganger vi tallet på den omvendte brøk:
\begin{align*}
\frac{a}{\frac{b}{c}}= a \cdot \frac{c}{b} = \frac{a \cdot c}{b}.
\end{align*}
Bemærk, at når vi siger den omvendte brøk af $\frac{a}{b}$ så mener vi brøken $\frac{b}{a}$, hvor vi har byttet om på tæller og nævner.
\item Når vi ganger to brøker sammen, ganger vi tæller med tæller og nævner med nævner:
\begin{align*}
\frac{a}{b} \cdot \frac{c}{d} = \frac{a \cdot c}{b \cdot d}
\end{align*}
\item Når vi dividerer en brøk med en brøk, ganger vi med den omvendte brøk:
\begin{align*}
\frac{\frac{a}{b}}{\frac{c}{d}} = \frac{a}{b} \cdot \frac{d}{c} = \frac{a \cdot d}{b \cdot c}.
\end{align*}
\end{enumerate}
Det er ofte nemmere at forstå regnereglerne hvis man ser nogle konkrete eksempler.
\paragraph*{Eksempler:}
\begin{enumerate}
\item Udregn $\frac{4}{5}-\frac{2}{3}$:
\begin{align*}
\frac{4}{5}-\frac{2}{3} = \frac{4\cdot 3}{5 \cdot 3} - \frac{5 \cdot 2}{5 \cdot 3} = \frac{12}{15} - \frac{10}{15} = \frac{12-10}{15} = \frac{2}{15}.
\end{align*}
\item Udregn $2 \cdot \frac{4}{5}$:
\begin{align*}
2 \cdot \frac{4}{5}= \frac{2 \cdot 4}{5} = \frac{8}{5}.
\end{align*}
\item Udregn $\frac{\frac{4}{3}}{7}$:
\begin{align*}
\frac{\frac{4}{3}}{7} = \frac{4}{3 \cdot 7} = \frac{4}{21}.
\end{align*}
\item Udregn $\frac{5}{\frac{2}{3}}$:
\begin{align*}
\frac{5}{\frac{2}{3}} = 5 \cdot \frac{3}{2} = \frac{5 \cdot 3}{2} = \frac{15}{2}.
\end{align*}
\item Udregn $\frac{3}{4} \cdot \frac{9}{4}$:
\begin{align*}
\frac{3}{4} \cdot \frac{9}{4} = \frac{3 \cdot 9}{4 \cdot 4} = \frac{27}{16}.
\end{align*}
\item  Udregn $\frac{\frac{1}{2}}{\frac{3}{5}}$:
\begin{align*}
\frac{\frac{1}{2}}{\frac{3}{5}} = \frac{1}{2} \cdot \frac{5}{3} = \frac{1 \cdot 5}{2 \cdot 3} = \frac{5}{6}.
\end{align*}
\end{enumerate}

\paragraph*{Forlænge og forkorte brøker:}
Et trick i matematikken er at tage et tal og gange det med $1$, da det ikke ændre noget ved det oprindelige tal, men hvis man skriver $1$ på en smart måde kan det ofte simplificere ens udregninger. 

Hvis vi ganger en brøk med $a$ i både tæller og nævner kalder vi det at \emph{forlænge brøken med $a$}:
\begin{align*}
\frac{b}{c} = \frac{b}{c} \cdot 1 = \frac{b}{c} \cdot \frac{a}{a} = \frac{b \cdot a}{c \cdot a}.
\end{align*}
Hvis vi dividere en brøk med $a$ i både tæller og nævner så kalder vi det at \emph{forkorte brøken med $a$}:
\begin{align*}
\frac{b}{c} \cdot 1 = \frac{b}{c} \cdot \frac{\frac{1}{a}}{\frac{1}{a}} = \frac{\frac{b}{a}}{\frac{c}{a}}. 
\end{align*}
Bemærk at ved at forlænge eller forkorte brøker, ændre vi ikke deres værdi.
\paragraph{Eksempler:}
\begin{enumerate}
\item Forlæng brøken $\frac{4}{5}$ med $2$:
\begin{align*}
\frac{4}{5} \cdot 1 = \frac{4}{5} \cdot \frac{2}{2} = \frac{4 \cdot 2}{5 \cdot 2} = \frac{8}{10}.
\end{align*}
\item Forkort brøken $\frac{8}{10}$ med $2$:
\begin{align*}
\frac{8}{10} \cdot 1 = \frac{8}{10} \cdot  \frac{\frac{1}{2}}{\frac{1}{2}} = \frac{\frac{8}{2}}{\frac{10}{2}} = \frac{4}{5}.
\end{align*}
\end{enumerate}
Bemærk at dette betyder, at når vi forkorter en brøk med $a$ dividerer vi samtlige led i både tæller og nævner med $a$. Det er en klassisk fejl som mange begår, at man tror man kan nøjes med at forkorte nogle af leddene i tælleren eller nævneren, f.eks. er
\begin{align*}
\frac{x^3+x^2+1}{x} \neq x^2+x+1,
\end{align*} 
da $x$ ikke går op i $1$. Det rigtige vil i stedet være at forkorte brøken med $x$, hvilket giver
\begin{align*}
\frac{x^3+x^2+1}{x}= \frac{\frac{x^3}{x}+\frac{x^2}{x}+ \frac{1}{x}}{\frac{x}{x}} = x^2+x+\frac{1}{x}.
\end{align*}

Vi siger at en brøk er \emph{uforkortelig} hvis der ikke eksisterer noget heltal større end $1$, som går op i både tælleren og nævneren. 

Alle facit i dette kursus, som er brøker, skal angives som uforkortelige brøker (ikke decimaltal), eksempelvis:
\begin{align*}
\frac{2}{4}+\frac{8}{4} = \frac{10}{4} = \frac{5}{2},
\end{align*}
har facit $\frac{5}{2}$ og ikke $\frac{10}{4}$, selvom tallene i princippet et ens. 













