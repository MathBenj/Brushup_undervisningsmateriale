\section{Vektorer i rummet}
\noindent Vi er nu kommet til at studere 3-dimensionelle vektorer, som vi kan tænke på som værende punkter i et 3-dimensionalt koordinatsystem. Vi noterer en sådan vektor ved
\begin{align*}
\vec{v}=\begin{bmatrix}
x \\
y \\
z
\end{bmatrix}.
\end{align*}
\paragraph*{Regneregler:}
Lad $\vec{u} = \begin{bmatrix} x_1 \\ y_1 \\ z_1 \end{bmatrix}$ og $\vec{v} = \begin{bmatrix} x_2 \\ y_2 \\ z_2 \end{bmatrix}$, så har vi har følgende regneregler for vektorer i rummet.
\begin{enumerate}
\item $\displaystyle \vec{u} \pm \vec{v} = \begin{bmatrix} x_1 \\ y_1 \\ z_1 \end{bmatrix} \pm \begin{bmatrix} x_2 \\ y_2 \\ z_2 \end{bmatrix} = \begin{bmatrix} x_1 \pm x_2 \\ y_2 \pm y_2 \\ z_1 \pm z_2 \end{bmatrix} $.
\item $\displaystyle c \vec{u} = c  \begin{bmatrix} x_1 \\y_1 \\ z_1 \end{bmatrix} = \begin{bmatrix} cx_1 \\ c y_1 \\ cz_1 \end{bmatrix}$, hvis $c \in \mathbb{R}$.
\item Længden af $\vec{u}$ noteres $\Vert \vec{u} \Vert$ og er givet ved $\displaystyle \Vert \vec{u} \Vert = \sqrt{x_1^2+y_1^2+z_1^2}$
\end{enumerate}

Hvis vi får givet to punkter i vores koordinatsystem $A=(x_1,y_1,z_1)$ og $B=(x_2,y_2,z_2)$ og vi gerne vil bestemme vektoren fra $A$ til $B$, så er den givet ved
\begin{align*}
\overrightarrow{AB}= \begin{bmatrix}
x_2-x_1 \\
y_2 - y_1\\
z_2 - z_1
\end{bmatrix}.
\end{align*}

Det betyder, at hvis $O$ er origo i vores koordinatsystem og $A=(x,y,z)$ så er vektoren fra origo til $A$ givet ved
\begin{align*}
\overrightarrow{OA}= \begin{bmatrix}x-0 \\y - 0 \\ z - 0 \end{bmatrix}= \begin{bmatrix}x \\y \\ z \end{bmatrix}.
\end{align*}
Vi kalder vektoren der går fra origo til $A$ for stedvektoren til $A$. Da vi kun er interesseret i længden og retningen for en vektor og ikke de to punkter den går mellem, så vil vi altid tænke på en vektor som værende stedvektoren. Bemærk, at $\overrightarrow{OA}$ og $A$ har de samme koordinater, og derfor kan man både tænke på en vektor som et objekt med en længde og en retning, men også som et punkt i et koordinatsystem.

\paragraph*{Eksempler:}
\begin{enumerate}
\item Lad $\vec{u}= \begin{bmatrix} 3 \\ 2 \\ 4 \end{bmatrix}$ og $\vec{v}= \begin{bmatrix} 7 \\ 1 \\ 2 \end{bmatrix}$ og udregn $\vec{u}+\vec{v}$ og $\vec{v}-\vec{u}$:

Vi benytter regneregel $1$. og får
\begin{align*}
\vec{u} + \vec{v}&= \begin{bmatrix} 3 \\ 2 \\ 4 \end{bmatrix} + \begin{bmatrix} 7 \\ 1 \\ 2 \end{bmatrix} = \begin{bmatrix} 3+7 \\ 2+1 \\ 4 + 2 \end{bmatrix}  = \begin{bmatrix} 10 \\ 3 \\ 6 \end{bmatrix}. \\
\vec{v} - \vec{u}&= \begin{bmatrix} 7 \\ 1 \\ 2 \end{bmatrix} - \begin{bmatrix} 3 \\ 2 \\ 4 \end{bmatrix} = \begin{bmatrix} 7-3 \\ 1-2 \\ 2 - 4 \end{bmatrix}  = \begin{bmatrix} 4 \\ -1 \\ -2 \end{bmatrix}.
\end{align*}
\item Lad $\vec{u}= \begin{bmatrix} 1 \\ 2 \\ 3 \end{bmatrix}$ og udregn $3\vec{u}$:

Vi benytter regneregel $2$. og får
\begin{align*}
3\vec{u} = 3\begin{bmatrix} 1 \\ 2 \\ 3 \end{bmatrix} = \begin{bmatrix} 3\cdot 1 \\ 3 \cdot 2 \\ 3 \cdot 3 \end{bmatrix}  = \begin{bmatrix} 3 \\ 6 \\ 9 \end{bmatrix}. 
\end{align*}
\item Lad $\vec{u}= \begin{bmatrix} 2 \\ 2 \\ 1 \end{bmatrix}$ og udregn $\norm{\vec{u}}$:

Vi benytter regneregel $3.$ og får
\begin{align*}
\norm{\vec{u}} = \sqrt{2^2+2^2 + 1^2} = \sqrt{4+4+1}= \sqrt{9}=3.
\end{align*}
\end{enumerate}

\paragraph*{Prikprodukt:}
Vi definerer prikproduktet mellem to vektorer $\vec{u} = \begin{bmatrix} x_1 \\ y_1 \\ z_1 \end{bmatrix}$ og $\vec{v} = \begin{bmatrix} x_2 \\ y_2 \\ z_2 \end{bmatrix}$ til at være
\begin{align}\label{eq:vec3d1prikprodukt}
\vec{u} \cdot \vec{v} =\begin{bmatrix} x_1 \\ y_1 \\ z_1 \end{bmatrix} \cdot \begin{bmatrix} x_2 \\ y_2 \\ z_2 \end{bmatrix} = x_1 x_2 + y_1  y_2 + z_1z_2.
\end{align}
Bemærk, at prikproduktet af to vektorer er et tal og ikke en vektor og at man ikke kan gange to vektorer sammen, man prikker dem sammen (så prikken mellem de to vektorer er ikke et gange tegn, men tegnet for et prikprodukt)!

Ved at benytte prikproduktet kan vi udregne vinklen $\theta$ mellem to vektorer $\vec{u}$ og $\vec{v}$ ved brug af formlen 
\begin{align}\label{eq:vec3d1vinkelvedprikprodukt}
\cos \theta = \frac{\vec{u} \cdot \vec{v}}{\norm{\vec{u}} \norm{\vec{v}}}.
\end{align}
Hvis prikproduktet $\vec{u} \cdot \vec{v}=0$, så har vi at 
\begin{align*}
\cos \theta = 0,
\end{align*}
hvilket betyder at $\theta$ er lig enten $\frac{\pi}{2}$ eller $\frac{3\pi}{2}$. Hvis vinklen mellem to vektorer er $\frac{\pi}{2}$ eller $\frac{3\pi}{2}$, så siger vi, at de to vektorer er ortogonale (vinkelrette). Dermed har vi at 
\begin{align*}
\vec{u} \perp \vec{v} \qquad \textup{ hvis og kun hvis } \qquad \vec{u} \cdot \vec{v} = 0,
\end{align*}
hvor $\perp$ betyder at $\vec{u}$ og $\vec{v}$ er ortogonale.

\paragraph*{Krydsprodukt:}
Det næste vi vil betragte er krydsproduktet af to vektorer. Hvis $\vec{u} = \begin{bmatrix} x_1 \\ y_1 \\ z_1 \end{bmatrix}$ og $\vec{v} = \begin{bmatrix} x_2 \\ y_2 \\ z_2 \end{bmatrix}$ så er krydsproduktet af $\vec{u}$ og $\vec{v}$ givet ved
\begin{align}\label{eq:vec3d1krydsprodukt}
\vec{u} \times \vec{v} = \begin{bmatrix}
\begin{vmatrix}
y_1 & y_2 \\
z_1 & z_2
\end{vmatrix}
\\
\\
\begin{vmatrix}
z_1 & z_2 \\
x_1 & x_2
\end{vmatrix}
\\
\\
\begin{vmatrix}
x_1 & x_2 \\
y_1 & y_2
\end{vmatrix}
\end{bmatrix}
=\begin{bmatrix}
\det\begin{bmatrix}
y_1 & y_2 \\
z_1 & z_2
\end{bmatrix}
\\
\\
\det\begin{bmatrix}
z_1 & z_2 \\
x_1 & x_2
\end{bmatrix}
\\
\\
\det\begin{bmatrix}
x_1 & x_2 \\
y_1 & y_2
\end{bmatrix}
\end{bmatrix}
=
\begin{bmatrix}
y_1z_2 - z_1y_2 \\
z_1x_2 - x_1z_2 \\
x_1y_2 - y_1x_2
\end{bmatrix}.
\end{align}
Bemærk, at krydsproduktet af to vektorer er en vektor, som står vinkelret på både $u$ og $v$.

Ligesom ved prikproduktet kan vi også benytte krydsproduktet af to vektorer $\vec{u}$ og $\vec{v}$ til at bestemme vinklen i mellem dem ud fra formlen
\begin{align}\label{eq:vec3d1vinkelvedkrydsprodukt}
\sin \theta = \frac{\norm{\vec{u} \times \vec{v}}}{\norm{\vec{u}}\norm{\vec{v}}},
\end{align}
hvor $0\leq\theta\leq\pi$. Hvis krydsproduktet har norm $\norm{\vec{u} \times \vec{v}}=0$, så har vi at 
\begin{align*}
\sin \theta = 0,
\end{align*}
hvilket betyder at $\theta$ er lig enten $0$ eller $\pi$. Hvis vinklen mellem to vektorer er $0$ eller $\pi$, så siger vi, at de to vektorer er parallelle. Dermed har vi at 
\begin{align*}
\vec{u} \parallel \vec{v} \qquad \textup{ hvis og kun hvis } \qquad \norm{\vec{u} \times \vec{v}} = 0,
\end{align*}
hvor $\parallel$ betyder at $\vec{u}$ og $\vec{v}$ er parallelle. Da den eneste vektor der har norm lig med $0$ er nulvektoren, har vi at
\begin{align*}
\vec{u} \parallel \vec{v} \qquad \textup{ hvis og kun hvis } \qquad \vec{u} \times \vec{v} = \vec{0}.
\end{align*}
Derudover har vi at normen af $\vec{u} \times \vec{v}$ er lig med arealet af det parallellogram der er udspændt af $\vec{u}$ og $\vec{v}$, altså
\begin{align*}
A = \norm{\vec{u} \times \vec{v}},
\end{align*}
hvor $A$ er arealet af det udspændte parallellogram.

\paragraph*{Eksempler:}
\begin{enumerate}
\item Lad $\vec{u}= \begin{bmatrix} 3 \\ 2 \\ 4\end{bmatrix}$ og $\vec{v}= \begin{bmatrix} 7 \\ 1 \\ 2 \end{bmatrix}$ og udregn $\vec{u}\cdot \vec{v}$ og $\vec{u} \times \vec{v}$:

Vi benytter \eqref{eq:vec3d1prikprodukt} og \eqref{eq:vec3d1krydsprodukt} til at finde henholdsvis prikproduktet og krydsproduktet
\begin{align*}
\vec{u}\cdot \vec{v} &=\begin{bmatrix} 3 \\ 2 \\ 4 \end{bmatrix}\cdot  \begin{bmatrix} 7 \\ 1 \\ 2 \end{bmatrix} = 3 \cdot 7 + 2 \cdot 1 + 4 \cdot 2 = 21+2 + 8 = 31.\\
\vec{u} \times \vec{v}&=\begin{bmatrix}
2 \cdot 2 - 4 \cdot 1 \\
4 \cdot 7 - 3 \cdot 2 \\
3 \cdot 1 - 2 \cdot 7
\end{bmatrix}
=\begin{bmatrix}
0 \\
22 \\
-11
\end{bmatrix}.
\end{align*}
\item Lad $\vec{u}= \begin{bmatrix} 0 \\ 0 \\ 1 \end{bmatrix}$ og $\vec{v}= \begin{bmatrix} 0 \\ -1 \\ 0 \end{bmatrix}$ og udregn vinklen mellem $\vec{u}$ og $\vec{v}$:

Vi kan udregne vinklen enten ved at benytte \eqref{eq:vec3d1vinkelvedprikprodukt} eller \eqref{eq:vec3d1vinkelvedkrydsprodukt}. Vi vælger at benytte \eqref{eq:vec3d1vinkelvedprikprodukt}, så vi finder først prikproduktet og normen af de to vektorer
\begin{align*}
\vec{u} \cdot \vec{v} &= \begin{bmatrix} 0 \\ 0 \\1 \end{bmatrix} \cdot \begin{bmatrix} 0 \\ -1 \\ 0 \end{bmatrix} = 0 \cdot 0 + 0 \cdot -1 + 1 \cdot 0 = 0 \\
\norm{\vec{u}} &= \sqrt{0^2+0^2+1^2} = \sqrt{1}=1 \\
\norm{\vec{v}} &= \sqrt{0^2 + (-1)^2 + 0^2}=\sqrt{1}=1.
\end{align*}
Dermed har vi, at
\begin{align*}
\cos \theta = \frac{\vec{u} \cdot \vec{v}}{\norm{\vec{u}}\norm{\vec{v}}} = \frac{0}{1 \cdot 1} = 0.
\end{align*}
Dvs. at vinklen mellem $\vec{u}$ og $\vec{v}$ er den vinkel der opfylder at $\cos \theta =0$ og vi husker, at det gør $\theta = \frac{\pi}{2}$ eller $\theta=\frac{3\pi}{2}$.
\item Lad $\vec{u}= \begin{bmatrix} 3 \\ 0 \\ 3 \end{bmatrix}$ og $\vec{v}= \begin{bmatrix} 1 \\ 0 \\ 1 \end{bmatrix}$ og bestem om $\vec{u}\perp \vec{v}$ eller $\vec{u} \parallel\vec{v}$:

Vi udregner prikproduktet og krydsproduktet af de to vektorer
\begin{align*}
\vec{u} \cdot \vec{v} &= \begin{bmatrix} 3 \\ 0 \\ 3 \end{bmatrix}\cdot  \begin{bmatrix} 1 \\ 0 \\ 1 \end{bmatrix} = 3 \cdot 1 + 0 \cdot 0 + 3 \cdot 1 = 6 \\
\vec{u} \times \vec{v} &= \begin{bmatrix}
0 \cdot 1 - 3 \cdot 0 \\
3 \cdot 1 - 1 \cdot 3 \\
3 \cdot 0 - 0 \cdot 1
\end{bmatrix}
= \begin{bmatrix}
0 \\
0 \\
0
\end{bmatrix}.
\end{align*}
Da $\vec{u} \cdot \vec{v} \neq 0$ men $\vec{u} \times \vec{v}= \vec{0}$, har vi at $\vec{u}$ og $\vec{v}$ er parallelle men ikke ortogonale. Bemærk, at det eneste tidspunkt to vektorer kan være både ortogonale of paralelle, er hvis den ene af dem er nulvektoren.
\end{enumerate}