\section{Regneregler for bestemte integraler}
\noindent De sidste par gange har vi studeret ubestemte integraler. Det næste vi vil betragte er bestemte integraler. 

Hvis $f$ er en kontinuert funktion, så er det bestemte integral af $f$ i intervallet $[a,b]$ givet ved
\begin{align*}
\int_a^b f(x) \d x = [F(x)]^b_a = F(b)-F(a),
\end{align*}
hvor $F$ er en stamfunktion til $f$. Et ubestemt integrale er en funktion, hvorimod et bestemt integrale er et tal, som beskriver arealet mellem en funktion og $x$-aksen i intervallet $[a,b]$ (se Figur~\ref{fig:bestemtint1et}).
\begin{figure}[!htbp]
  \centering
  \begin{tikzpicture}
\begin{axis}[xmin=-0.1,xmax=3.5,ymin=-0.1,ymax=1,ticks=none,restrict y to domain =-0.2:3.5]
  	
	\addplot[thick,samples=300,name path=A] {{sin(deg(x))}} node[above right,pos=0.8] {$f(x)$};
	\draw[gray] (axis cs:pi/4,0) -- (axis cs:pi/4,{sqrt(2)/2}){};
	\draw[gray] (axis cs:11*pi/12,0) -- (axis cs:11*pi/12,{(sqrt(6)-sqrt(2))/4}){};
    \addplot[draw=none,name path=B] {0};     % “fictional” curve
    \addplot[colorgray] fill between[of=A and B,soft clip={domain=pi/4:11*pi/12}];
    \node[below] at (axis cs:pi/4,0){$a$};
    \node[below] at (axis cs:11*pi/12,0){$b$};
    \node[] at (axis cs:1.7,0.45){$\displaystyle\int_a^b f(x) \d x$};
\end{axis}
 \end{tikzpicture}
  \caption{Arealet under $f$ og over $x$-aksen mellem $a$ og $b$.}
  \label{fig:bestemtint1et}
\end{figure}

\paragraph*{Regneregler:}
Hvis $f$ og $g$ begge er kontinuerte funktioner, så har vi følgende regneregler for bestemte integraler (bemærk, at de minder meget om dem for ubestemte integraler).
\begin{enumerate}
\item $\displaystyle \int_a^b cf(x) \d x = c \int_a^b f(x) \d x$, hvor $c \in \mathbb{R}$.
\item $\displaystyle \int_a^b f(x) \pm g(x) \d x = \int_a^b f(x) \d x \pm \int_a^b g(x) \d x$.
\end{enumerate}

\paragraph*{Eksempler:}
\begin{enumerate}
\item Bestem arealet under $f(x)=\frac{1}{x}$ og over $x$-aksen i intervallet $[1,2]$:

Vi udregner det bestemte integrale
\begin{align*}
\int_1^2 f(x) \d x &= \int_1^2 \frac{1}{x} \d x \\
&=[\ln x]_1^2 \\
&=\ln 2 - \ln 1 \\
&=\ln 2.
\end{align*}

\item Bestem arealet under $f(x)=3x^2+3e^x$ og over $x$-aksen i intervallet $[0,4]$:

Vi udregner det bestemte integrale ved at benytte regnereglerne
\begin{align*}
\int_0^4 f(x) \d x &= \int_0^4 (3x^2+3e^x) \d x \\
&= 3\int_0^4 x^2 \d x+ 3 \int_0^4 e^x \d x \\
&= 3\Big[ \frac{1}{3}x^3 \Big]_0^4 + 3[e^x]_0^4 \\
&=3 \Big(\frac{1}{3}\cdot 4^3 - \Big(\frac{1}{3} \cdot 0^4 \Big) \Big) + 3 ( e^4 - (e^0)) \\
&=3 \cdot \frac{64}{3} + 3e^4-3\\
&=61+3e^4.
\end{align*}
\end{enumerate}

\paragraph*{Arealet mellem to funktioner:}
Vi har indtil nu kun betragtet arealet mellem en funktion og $x$-aksen, men det er også muligt at finde arealet mellem to funktioner. Hvis $f$ og $g$ er to funktioner hvor $f(x) \geq g(x)$ for alle $x \in [a,b]$, så er arealet mellem de to funktioner givet ved
\begin{align*}
\int_a^b f(x) \d x - \int_a^b g(x) \d x = \int_a^b f(x) - g(x) \d x.
\end{align*}
Det betyder, at for at finde arealet mellem $f$ og $g$ findet vi arealet mellem $f$ og $x$-aksen, og trækker så arealet mellem $g$ og $x$-aksen fra (se Figur~\ref{fig:bestemtint1et},~\ref{fig:bestemtint1to} og~\ref{fig:bestemtint1tre}).


\begin{figure}[!htbp]
\begin{minipage}{0.49\textwidth}
\centering
\begin{tikzpicture}
\begin{axis}[xmin=-0.1,xmax=3.5,ymin=-0.1,ymax=1,ticks=none,restrict y to domain =-0.2:3.5]
  	
	\addplot[thick,samples=300,name path=A] {0.031*x^2} node[above=5pt,pos=0.82] {$g(x)$};
	\draw[gray] (axis cs:pi/4,0) -- (axis cs:pi/4,0.01912){};
	\draw[gray] (axis cs:11*pi/12,0) -- (axis cs:11*pi/12,0.257){};
    \addplot[draw=none,name path=B] {0};     % “fictional” curve
    \addplot[colorgray] fill between[of=A and B,soft clip={domain=pi/4:11*pi/12}];
    \node[below] at (axis cs:pi/4,0){$a$};
    \node[below] at (axis cs:11*pi/12,0){$b$};
     \node[] at (axis cs:02.4,0.07){$\int_a^b g(x) \d x$};
\end{axis}
 \end{tikzpicture}
  \caption{Arealet under $g$.}
  \label{fig:bestemtint1to}
\end{minipage}
\begin{minipage}{0.49\textwidth}
 \centering
\begin{tikzpicture}
\begin{axis}[xmin=-0.1,xmax=3.5,ymin=-0.1,ymax=1,ticks=none,restrict y to domain =-0.2:3.5]
  	\addplot[thick,samples=300,name path=A] {{sin(deg(x))}} node[above right,pos=0.8] {$f(x)$};
	\addplot[thick,samples=300,name path=B] {0.031*x^2} node[above=5pt,pos=0.82] {$g(x)$};
	\draw[gray] (axis cs:pi/4,0.01912) -- (axis cs:pi/4,{sqrt(2)/2}){};
    \addplot[colorgray] fill between[of=A and B,soft clip={domain=pi/4:11*pi/12}];
    \node[below] at (axis cs:pi/4,0){$a$};
    \node[below] at (axis cs:11*pi/12,0){$b$};
    \node[] at (axis cs:1.8,0.4){$\displaystyle \int_a^b f(x) - g(x) \d x$};
\end{axis}
 \end{tikzpicture}
  \caption{Arealet mellem $f$ og $g$.}
  \label{fig:bestemtint1tre}
\end{minipage}
\end{figure}

\paragraph*{Eksempel:}
\begin{enumerate}
\item Find arealet under $f(x)=12-2x^2$ og over $g(x)=x^2$ i intervallet $[-2,2]$:

Vi udregner det bestemte integralet af de to funktioner trukket fra hinanden
\begin{align*}
\int_{-2}^2 f(x) - g(x) \d x &= \int_{-2}^2 \big( 12 -2x^2 - x^2\big) \d x \\
&= \int_{-2}^2 12 - 3x^2 \d x \\
&=\Big[ 12x - \frac{3}{3}x^3 \Big]_{-2}^2 \\
&=\Big[ 12x - x^3 \Big]_{-2}^2 \\
&= 24 -  8 - \Big( -24 - (-8) \Big) \\
&= 24-8+24-8 \\
&= 32.
\end{align*}
\end{enumerate}
