\subsection{Opgaver}
\begin{enumerate}
	
	
	
	\item Bestem den fuldstændige løsning til differentialligningen 
	\begin{align*}
	\frac{dy}{dx}+y=x.
	\end{align*}
	(Hint: Brug Opgave~\ref{it:int21}).
	
	\item Betragt ligningen 
	\begin{align}\label{eq:diffeq41}
	y'+y=\sin(x).
	\end{align}
	\begin{enumerate}
		\item Bestem den fuldstændige løsning $y_h$ til den homogene ligning
		\begin{align*}
		y'+y=0.
		\end{align*}
		\item Bestem $a$ og $b$ så $y_p(x)=a\cos(x)+b\sin(x)$ bliver en partikulær løsning til~\eqref{eq:diffeq41}. 
		\item Brug løsningsformlen for inhomogene førsteordens differentialligninger til at løse ligningen. Hvad er forskellen på denne løsning og $y_h+y_p$. (Hint: Brug Opgave~\ref{it:int22}.)
	\end{enumerate}

	\item Bestem den fuldstændige løsning til differentialligningen 
	\begin{align*}
	y'-2y=x^2.
	\end{align*}
	(Hint: Opgave~\ref{it:int21}.)
	
	\item Den generelle løsningsformel for en inhomogen førsteordens differentialligning med konstante koefficienter
	\begin{align*}
	y'+ky=q(x),
	\end{align*}
	er givet ved
	\begin{align*}
	y(x)=e^{-kx}\int q(x)e^{kx}\dd x+ ce^{-kx}.
	\end{align*}
	Brug dette til at bestemme en formel for løsningen af en inhomogen førsteordens differentialligning som ikke indeholder et $y$ afhængigt led.
	
	
	
	\item Antag at det radioaktive materiale fra Opgave~\ref{it:diffeq31}, med mængde beskrevet ved $A(t)$, henfalder til et andet radioaktivt materiale, med mængde beskrevet ved $B(t)$. Dette radioaktive materiale henfalder også med en hastighed der er proportionel med mængden $B(t)$. Det betyder at den afledede af $B$ både afhænger af $A$ og $B$. Vi har dermed ligningen
	\begin{align*}
	B'=k_1 A-k_2 B.
	\end{align*} 
	Løs ligningen når $k_1=4$, $k_2=2$, $A(0)=3$ og $B(0)=0$. (Hint: Brug formlen for $A(t)$ fundet i Opgave~\ref{it:diffeq31}). Hvad er den største værdi af $B(t)$ og hvor stor er $A(t)$ på det tidspunkt?
	
	\item Bestem den fuldstændige løsning til differentialligningen
	\begin{align*}
	y'+y=\frac{1}{x}+\ln x.
	\end{align*}
	(Hint: Opgave~\ref{it:int23})
	
	
	\item Vi vil gerne finde en generel formel til at løse differentialligninger på formen
	\begin{align}\label{eq:diffeq2}
	y'+p(x)y=q(x),
	\end{align}
	hvor $p$ og $q$ er kontinuerte funktioner. En måde at gøre dette på er at anvende produktreglen for differentiation:
	\begin{align}\label{eq:diffeq4}
	(yf)'(x)=y'(x)f(x)+y(x)f'(x).
	\end{align}
	\begin{enumerate}
		\item Ved at gange~\eqref{eq:diffeq2} igennem med $f(x)$ får vi at
		\begin{align}\label{eq:diffeq3}
		y'(x)f(x)+p(x)f(x)y(x)=f(x)q(x).
		\end{align}
		Hvilken differentialligning skal $f$ opfylde for at vi kan anvende produktreglen? (Hint: sammenlign højresiden af~\eqref{eq:diffeq4} med venstresiden af~\eqref{eq:diffeq3})
		
		\item Brug Opgave~\ref{it:diffeq12} til at finde en løsning til differentialligning vi bestemte i del (a). 
		
		\item Redegør for at vi kan omskrive~\eqref{eq:diffeq2} til
		\begin{align*}
		\frac{d}{dx} \Big(y(x) e^{P(x)} \Big)=e^{P(x)}q(x),
		\end{align*}
		hvor $P$ er en stamfunktion til $p$. Brug dette til at bestemme den generelle løsning til~\eqref{eq:diffeq2}. (Hint: Ved at integrere begge sider af ligningen ovenfor får vi at
		\begin{align*}
		y(x)e^{P(x)}=\int q(x)e^{P(x)}\dd x+c,
		\end{align*}
		hvorefter vi kan isolere for $y$.
	\end{enumerate}
	
	\item Bestem den fuldstændige løsning til
	\begin{align*}
	xy'=2(y-4).
	\end{align*}
	
	\item Bestem den fuldstændige løsning til 
	\begin{align*}
	y'+\frac{y}{x}=\ln(x).
	\end{align*}
	(Hint: Brug første delopgave i Opgave~\ref{it:int23}.)
	
	
\end{enumerate}