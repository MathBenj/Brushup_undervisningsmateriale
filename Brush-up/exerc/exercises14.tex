\subsection{Opgaver}

\begin{enumerate}
	\item \label{it:diff23}Bestem den afledede af funktionen $f(x)=x\ln(x)-x$.
	
	\item\label{it:diff21} Vis at 
	\begin{align*}
	\frac{d}{dx} \tan x= 1+\tan^2x.
	\end{align*}
	(Hint: Brug $\tan x=\frac{\sin x}{\cos x}$)
	Hvordan passer dette med Tabel 1 i oplægget fra sidste gang?

	\item \label{it:diff24} Bestem den afledede af funktionen $f(x)=(x-1)e^x$.
	
	\item Differentier funktionerne
	\begin{align*}
	f(x)=3xe^x,&& f(x)=2x^2\sin x,&& f(x)=\frac{3x^2+2x-1}{x-1}.
	\end{align*}
	
	
	\item Differentier funktionerne
	\begin{align*}
	f(x)=\frac{2x^5-2x^3+1}{x^4-2x},&& f(x)=\frac{\frac{2}{x}-\frac{3}{x^2}}{\frac{1}{x^3}+\frac{2}{x^4}},&& f(x)=\frac{x\sin x}{-\cos x}
	\end{align*} 
	
	
	\item Lad $f,g,h$ være differentiable funktioner. Brug produktreglen til at vise at 
	\begin{align*}
	\frac{d}{dx} (fgh)(x)=f'(x)g(x)h(x)+f(x)g'(x)h(x)+f(x)g(x)h'(x).
	\end{align*}
	
	
	
	
	\item Lad $f$ og $g$ være differentiable funktioner. Brug produktreglen til at vise at
	\begin{align*}
	(fg)''(x)=f''(x)g(x)+2f'(x)g'(x)+f(x)g''(x).
	\end{align*}
	
	\item Differentier følgende funktioner
	\begin{align*}
	f(x)=\frac{x-e^x}{1+x},&&f(x)=\frac{x}{e^x+1},&&f(x)=\frac{\ln(x)}{x},&& f(x)=\frac{\cos x}{\sin x}
	\end{align*}
	
	\item Udregn følgende
	\begin{align*}
	\frac{d^2}{dx^2} e^{-x}x^2,&& \frac{d^2}{dx^2} e^x\ln(x),&& \frac{d^2}{dx^2} (x^2+1)\sin(x).
	\end{align*}
	
	
	
	
	
	\item Differentier funktionerne
	\begin{align*}
	f(x)=\frac{x^2e^x}{-\ln(x^{x})},&& g(x)=x^2e^x\ln x,&& h(x)=\tan(x)e^{-2x}\ln(x)x^2
	\end{align*}
	
	
	
	
	\item\label{it:diff22} Vis at den afledede af funktionen
	\begin{align*}
	f(x)=\begin{cases}
	\frac{\sin x}{x},&\textup{hvis }x\neq 0\\
	1,&\textup{ellers}. 
	\end{cases}
	\end{align*}
	Er givet ved
	\begin{align*}
	f'(x)=\begin{cases}
	\frac{x\cos x-\sin x}{x^2},&\textup{hvis }x\neq 0\\
	0,&\textup{ellers}. 
	\end{cases}
	\end{align*}
	(Hint: Fra Opgave~\ref{it:diff14} kender vi $f'(0)$, så vi behøver kun betragte tilfældet $ x\neq 0 $.)
	
	\item Vis at $f'(x)$ fra Opgave~\ref{it:diff22} er kontinuert. (Hint: Brug Opgave~\ref{it:lim4} og Opgave~\ref{it:lim5} til at bestemme $ \lim_{x\to 0}f'(x) $)
	

	\item Lad $f$ være givet som i Opgave~\ref{it:diff22} og lad $g(x)=1/x$. Vis at $g'(\frac{\pi}{2}+2\pi k)=f'(\frac{\pi}{2}+2\pi k)$ for alle heltal $k$.
	
\end{enumerate}