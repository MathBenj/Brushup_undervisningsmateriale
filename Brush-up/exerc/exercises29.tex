\subsection{Opgaver}
\begin{enumerate}
	\item  Lad 
	\begin{align*}
	\vec{u}=\begin{bmatrix}
	7\\1\\4
	\end{bmatrix},\quad \textup{og}\quad \vec{v}=\begin{bmatrix}
	2\\-1\\8.
	\end{bmatrix}.
	\end{align*}
	Udregn
	\begin{align*}
	3\vec{u},&&-\vec{v},&&\vec{u}+\vec{v},&&\vec{u}-\vec{v},&&\norm{\vec{u}},&&\norm{\vec{v}},&&\vec{u}\c \vec{v},&& \vec{u}\times \vec{v}.
	\end{align*}
	
	\item Er vektorerne 
	\begin{align*}
	\vec{u}=\begin{bmatrix}
	1\\5\\0
	\end{bmatrix},\quad \textup{og}\quad \vec{v}=\begin{bmatrix}
	3\\6\\-1.
	\end{bmatrix}
	\end{align*}
	ortogonale?
	
	\item Lad
	\begin{align*}
	\vec{u}=\begin{bmatrix}
	2\\-2\\4
	\end{bmatrix},\quad \textup{og}\quad \vec{v}=\begin{bmatrix}
	1\\2\\-1.
	\end{bmatrix}.
	\end{align*}
	Bestem de værdier af $t$ hvor $\vec{u}+t\vec{v}$ står vinkelret på $\vec{u}-t\vec{v}$.

	\item Bestem arealet af parallelogrammet udspændt af vektorerne
	\begin{align*}
	\vec{u}=\begin{bmatrix}
	1\\-2\\3
	\end{bmatrix},\quad \textup{og}\quad \vec{v}=\begin{bmatrix}
	2\\2\\1.
	\end{bmatrix}.
	\end{align*}

	\item Vis at $\vec{u}\times \vec{u}=\vec{0}$.
	
	\item Lad
		\begin{align*}
	\vec{u}=\begin{bmatrix}
	0\\3\\2
	\end{bmatrix},\quad \textup{og}\quad \vec{v}=\begin{bmatrix}
	-1\\3\\1
	\end{bmatrix}
	\end{align*}
	og bestem $t$ så parallelogrammet udspændt af $\vec{u}$ og $t\vec{v}$ har areal $3$.

	\item Lad
	\begin{align*}
	\vec{u}=\begin{bmatrix}
	1\\1\\-2
	\end{bmatrix},\quad \textup{og}\quad \vec{v}=\begin{bmatrix}
	3\\-1\\0
	\end{bmatrix}
	\quad \textup{og}\quad \vec{w}=\begin{bmatrix}
	2\\3\\-1
	\end{bmatrix}.
	\end{align*}
	Udregn
	\begin{align*}
	\vec{u}\times \vec{v},&& \vec{u}\times(\vec{v}+\vec{w}),&& \vec{u}\cdot (\vec{v}+\vec{w}),&& \vec{v}\times \vec{u},&& \vec{u}\times \vec{v}+\vec{u}\times \vec{w},&& \vec{w}\cdot (\vec{u}\times \vec{v}).
	\end{align*}
	
	\item Vis at der også i rummet gælder at $\vec{u}\c \vec{v}=\vec{v}\c \vec{u}$. 
	
	\item Vis at der også i rummet gælder at $\norm{k\vec{u}}=\abs{k}\norm{\vec{u}}$.
	
	\item Vis at der også i rummet gælder at $\norm{\vec{u}+\vec{v}}^2=\norm{\vec{u}}^2+\norm{\vec{v}}^2+2(\vec{u}\c \vec{v})$.
	
	\item Vis at $\vec{u}$ og $\vec{v}$ står vinkelret på $\vec{u}\times \vec{v}$ for alle vektorer $u$ og $v$.
	
\end{enumerate}