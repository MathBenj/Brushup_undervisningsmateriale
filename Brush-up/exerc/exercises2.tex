\subsection{Opgaver}

\begin{enumerate}
\item Reducer følgende udtryk:
\begin{align*}
(x+1)^2,&& (2x-3)^2,&& (x-2)(x+2)+4,&& (3a-2b)^2+6ab.
\end{align*}
\item Forkort følgende børker
\begin{align*}
\frac{(x+3)^2}{2x^2+6x},&& \frac{4x^2-9}{4x^2+9-12x},&& \frac{2x^2+18+12x}{x^2+3x},&& \frac{(x-y)^2-y^2}{2x}.
\end{align*}
\item Følgende ligninger beskriver cirkler i planen. Angiv deres centrumskoordinater og radius.
\begin{align*}
x^2+y^2=1,&& x^2-2x+y^2+2y-23=0,&& x^2+4x+y^2=0.
\end{align*}
\item Udregn følgende tal.
\begin{align*}
99^2-101^2,&& 999^2,&& 499^2-501^2,&& 99998^2-100002^2.
\end{align*}
\item Reducer følgende udtryk
\begin{align*}
(a-2)^2-(a-2)(a+2),&& \frac{x^2-y^2}{x-y}+\frac{x^2-y^2}{x+y},&&\frac{ 4x^2+9+12x}{2x-3}-\frac{24}{2- \frac{3}{x}}.
\end{align*}
\item Følgende ligninger beskriver cirkler i planen. Angiv deres centrumskoordinater og radius.
\begin{align*}
2x^2-12x+2y^2-16y=0,&& x^2-x+y^2+y=\frac{1}{2}.
\end{align*}
\item \label{it:1} Gør rede for hvordan formlen $(a+b)^2=a^2+b^2+2ab$ kan illustreres med Figur~\ref{fig:1}.
\item \label{it:2} Gør rede for hvordan formlen $(a-b)(a+b)=a^2-b^2$ kan illustreres med Figur~\ref{fig:2}.
\item \label{it:3} Vis Pythagoras' Sætning $a^2+b^2=c^2$ ved hjælp af Figur~\ref{fig:3}.

\item \label{it:ex13} Reducer følgende udtryk:
\begin{align*}
(-a-6b)^2,&& (-4-a)(-4+a),&& \Big(x+\frac{1}{x}\Big)^2.
\end{align*}
\item Vis, at
\begin{align*}
\frac{7a +b}{4a^2-4b^2}-\frac{3}{4a+4b}-\frac{3}{4a-4b}=\frac{1}{4a-4b}.
\end{align*}

\item \label{it:4} Vis at
\begin{align*}
(a+b+c)^2= a^2+b^2+c^2+2ab+2ac+2bc.
\end{align*}
(Hint: lad $d=b+c$ og start med at betragte $(a+d)^2$.)
\item\label{it:eks21} Lad $a,b,c$ være reelle med $a\neq 0$. Bestem konstanter $d,k$ således at ligningen
\begin{align*}
ax^2+bx+c=0
\end{align*}
kan omskrives til 
\begin{align*}
(x+k)^2=\frac{d}{4a^2}.
\end{align*}
(Hint: Divider med $a$ og brug en kvadratsætning.)

\begin{figure}
\centering
\begin{tikzpicture}
\draw (0,0)--(5,0)--(5,5)--(0,5)-- cycle;
\draw[dashed] (0,4)--(5,4);
\draw[dashed] (4,0)--(4,5);
\node at (2,0) [label=below: $a$] {};
\node at (4.5,0) [label=below:$b$] {};
\node at (0,2) [label=left: $a$] {};
\node at (0,4.5) [label=left:$b$] {};
\node at (2,5) [label=above: $a$] {};
\node at (4.5,5) [label=above:$b$] {};
\node at (5,2) [label=right: $a$] {};
\node at (5,4.5) [label=right:$b$] {};
\end{tikzpicture}
\caption{Opgave~\ref{it:1}}
\label{fig:1}
\end{figure}
%
\begin{figure}
\centering
\begin{tikzpicture}
\draw (0,0)--(0,4)--(4,4)--(4,0)--cycle;
\draw[pattern=north east lines] (0,0)--(3,0)--(3,1)--(0,4)--cycle;
\draw[fill=gray] (3,1)--(0,4)--(4,4)--(4,1)--cycle;
\node at (1.5,0) [label=below: $a-b$] {};
\node at (3.5,0) [label=below:$b$] {};
\node at (0,2) [label=left: $a$] {};
\node at (2,4) [label=above: $a$] {};
\node at (4,2.5) [label=right: $a-b$] {};
\node at (4,0.5) [label=right:$b$] {};
%%%Newfig
\draw (8,0)--(11,0)--(11,5)--(8,5)--cycle;
\draw[pattern=north east lines] (8,0)--(8,4)--(11,1)--(11,0)--cycle;
\draw[fill=gray] (8,4)--(8,5)--(11,5)--(11,1)-- cycle;
\node at (9.5,0) [label=below: $a-b$] {};
\node at (8,2) [label= left: $a$] {};
\node at (8,4.5) [label= left: $b$] {};
\node at (9.5,5) [label= above: $a-b$] {};
\node at (11,3) [label= right: $a$] {};
\node at (11,0.5) [label= right: $b$] {};
\end{tikzpicture}
\caption{Opgave~\ref{it:2}}
\label{fig:2}
\end{figure}
\begin{figure}
\centering
\begin{tikzpicture}[auto]
\draw (0,0)--(0,5)--(5,5)--(5,0)--cycle;
\draw (0,3)-- node {$c$} (2,0) ;
\draw (0,3)--node {$c$} (3,5);
\draw (3,5)--node {$c$} (5,2);
\draw (2,0)-- node {$c$} (5,2);
\node at (1,0) [label=below: $a$] {};
\node at (3.5,0) [label=below: $b$] {};
\node at (0,1.5) [label=left: $b$] {};
\node at (0, 4) [label=left: $a$] {};
\node at (1.5,5) [label=above: $b$] {};
\node at (4,5) [label=above: $a$] {};
\node at (5,1)  [label= right: $a$] {};
\node at (5,3.5) [label=right: $b$] {};
\end{tikzpicture}
\caption{Opgave~\ref{it:3}}
\label{fig:3}
\end{figure}
\end{enumerate}