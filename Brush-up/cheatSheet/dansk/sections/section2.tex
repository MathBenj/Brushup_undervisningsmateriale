\section{Ligninger}
Ligninger kan reduceres med følgende regler:
\begin{enumerate}
	\item Man må lægge til/trække fra med det samme tal på begge sider af et lighedstegn.
	\item Man må gange/dividere med det samme tal (undtagen 0) på begge sider af et lighedstegn.
\end{enumerate}
\subsection{Andengradsligninger}
Andengradsligninger er på formen
\begin{align}\label{eq:lig1}
ax^2+bx+c=0,
\end{align}
Løsningerne til~\eqref{eq:lig1} er
\begin{align*}
x=\frac{-b\pm \sqrt{b^2-4ac}}{2a}.
\end{align*}
\subsection{Faktorisering}
Hvis $ax^2+bx+c=0$ har rødder $r_1$ og $r_2$ så gælder.
\begin{align*}
ax^2+bx+c=a(x-r_1)(x-r_2).
\end{align*}