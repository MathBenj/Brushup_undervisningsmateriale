\newpage
\section{Math101 opgaver til 2. gang}
\begin{enumerate}
\item Løs ligningerne
\begin{align*}
4x+2=26,&& -3x-5=0,&&-5x+7=-28,&& 8x+13=5.
\end{align*}

\item Løs ligningerne
\begin{align*}
x^2=9,&&(x-3)(x+7)=0,&& 2x^2-6x+4=0,&& x^2+4x-5=0.
\end{align*}

\item Løs ligningerne
\begin{align*}
3x+7=-(2x+3),&& 3(x-4)=2(x+1),&& -3x-2=-x+3.
\end{align*}

\item Løs ligningerne
\begin{align*}
2x^2-6x=0,&& 3x^2-2x=0,&& x^2=\frac{1}{2}x,&& 25\Big(\frac{x}{2}\Big)^2= 1.
\end{align*}

\item Bestem $a$ så ligningen 
\begin{align*}
ax-\frac{1}{2}=7x+\frac{3}{2}
\end{align*}
ikke har nogen løsninger. 

\item Brug faktorisering af polynomier til at forkorte følgende brøker:
\begin{align*}
\frac{x^2-25}{x^2+4x-5},&&\frac{x^2-3x+2}{x^2-5x+6},&& \frac{(x+2)(x^2-3x-10)}{x^2+4x+4}.
\end{align*}

\item Bestem $a$ så $x=2$ bliver en løsning til ligningen
\begin{align*}
\frac{a}{4}+ax=1.
\end{align*}

\item Løs ligningerne
\begin{align*}
-(x+3)+2x=2(x-1)-x-1 ,&& 3(x-3)+2=3x-5.
\end{align*}


\item Løs ligningerne
\begin{align*}
\frac{2}{3}\Big(x- \frac{4}{5}\Big)=\frac{2}{3},&& \frac{1}{3}(x-2)=-\frac{2}{5}\Big(x-\frac{3}{4}\Big)
\end{align*}

\item Løs ligningerne
\begin{align*}
-\frac{1}{4}x^2-2x=-5,&& \frac{1}{2}x^2+3x=-\frac{5}{2},&& x^2-\frac{5}{6}x+\frac{1}{6}=0,&& 2x^2=1000.
\end{align*}



\item Løs ligningerne
\begin{align*}
\sqrt{2}x+1=\sqrt{2}+5,&& \pi(2x-6)=\sqrt{8}x+12,&& \sqrt{2}(2\sqrt{2}x+\sqrt{8})=x-1.
\end{align*}

	\item 	Opskriv en andengradsligning på formen $x^2+bx+c=0$ med rødderne
	\begin{align*}
	1 \textup{ og } 1,&& \frac{1}{3} \textup{ og } -1,&&  -\sqrt{2} \textup{ og } \sqrt{8}.
	\end{align*}
	
	
	\item For hvilke $b$ har ligningen
	\begin{align*}
	\frac{7}{6}x^2+bx+\frac{21}{2}=0,
	\end{align*}
	præcis en løsning?
	
	
	\item Bestem løsningerne til ligningerne
	\begin{align*}
	x^4-3x^2+2=0,&& x^4=\frac{17}{4}x^2-1.
	\end{align*}
	(Hint: Lad $y=x^2$.)
	

	
\end{enumerate}