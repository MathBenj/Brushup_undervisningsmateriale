\section{Ubestemte integraler}
\begin{frame}{Ubestemte integraler}{Stamfunktioner}
\begin{itemize}
	\setlength\itemsep{1em}
	\item<1-> En funktion $f$ har stamfunktion $F$ hvis
	\begin{align*}
	F'(x)=f(x)
	\end{align*}
	\item<2-> Hvis $F$ er stamfunktion til $f$ så er $F(x)+c$ også, for alle $c\in \R$.
	\item<3-> Det ubestemte integral af $f$ defineres til
	\begin{align*}
	\int f(x)\, dx =F(x)+c,
	\end{align*}
	hvor $F$ er en stamfunktion til $f$ og $c\in \R$.
	\item<4-> Eksempler: Er $e^{x^2}$ stamfunktion til $2xe^{x^2}$? \setbeamercovered{} \onslide<5->{$\frac{d}{dx} e^{x^2}=2xe^{x^2}$.}
\end{itemize}
\end{frame}

\section{Regneregler for ubestemte integraler}
\begin{frame}{Regneregler for ubestemte integraler}
	\vspace{-0.5cm}
\begin{itemize}
	\setlength\itemsep{1em}
	\item<1-> Vi har følgende regneregler:
\end{itemize}
\begin{minipage}{0.49\textwidth}
	\centering
	\begin{tabular}{@{}l c@{}}
\onslide<1->{$f(x)$      & $\int f(x)\, dx$}  		\\ \toprule
\onslide<2->{$c$			& $cx+k$} 				\\ \midrule
\onslide<3->{$x$			& $\frac{1}{2}x^2+k$}	\\ \midrule
\onslide<4->{$x^n$  		& $\frac{1}{n+1}x^{n+1}+k$, $(n\neq -1)$}	\\ \midrule
\onslide<5->{$e^x$  		& $e^x+k$}					\\ \midrule
\onslide<6->{$e^{cx}$  	& $\frac{1}{c}e^{cx}+k$}				\\ \bottomrule
	\end{tabular}
\end{minipage}
\begin{minipage}{0.49\textwidth}
	\centering
	\begin{tabular}{@{}l c@{}}
\onslide<1->{$f(x)$      & $\int f(x)\, dx$}  				\\ \toprule
\onslide<7->{$\frac{1}{x}$ & $\ln(\vert x\vert)+k $}			\\ \midrule
\onslide<8->{$\ln x$ 	& $x\ln(x)-x+k$}			\\ \midrule
\onslide<9->{$\cos x$  	& $\sin x+k$}				\\ \midrule
\onslide<10->{$\sin x$  	& $-\cos x+k$}				\\ \midrule
\onslide<11->{$\tan x$ 	& $-\ln(\vert \cos(x)\vert)+k$}		\\ \bottomrule  
	\end{tabular}
\end{minipage}
\vspace{1em}
\begin{itemize}
	\item<12-> Udregn:   
\end{itemize}
\vspace{-0.5cm}
\begin{align*}
\onslide<12->{\int \sqrt{x}\,d x\setbeamercovered{}\onslide<13->{=\int x^{\frac{1}{2}}\,d x \onslide<14->{=\frac{2}{3}x^\frac{3}{2}+k}},}&&\onslide<15->{\int x^3\, d x\setbeamercovered{}\onslide<16->{=\frac{1}{4}x^4+k}}
\end{align*}
\end{frame}

\begin{frame}{Regneregler for ubestemte integraler}
\begin{itemize}
		\setlength\itemsep{1em}
	\item<1-> Vi har følgende generelle regneregler
	\begin{align*}
	\onslide<1->{\int cf(x) \, d x&=c\int f(x)\, dx}\\
	\onslide<2->{\int f(x)\pm g(x) \, d x&=\int f(x)\, dx\pm \int g(x) \, dx.}
	\end{align*}
	\item<3-> Eksempler: Udregn
\end{itemize}
	\begin{align*}
\onslide<3->{\int e^{3x}+\sqrt[3]{x}+1\, dx,}\setbeamercovered{}\onslide<4->{=\int e^{3x} \, dx \onslide<5->{+ \int x^\frac{1}{3}\, dx}\onslide<6->{+\int 1 \, dx}\onslide<7->{=\frac{1}{3}e^{3x}\onslide<8->{+\frac{3}{4}x^\frac{4}{3}}\onslide<9->{+x+k}}}\\
\onslide<10->{\int \frac{1}{2x}-cos(x)\, dx\setbeamercovered{}\onslide<11->{=\frac{1}{2}\int\frac{1}{x}\, dx \onslide<12->{-\int \cos(x)\, dx}\onslide<13->{=\frac{1}{2}\ln(\vert x\vert )\onslide<14->{-\sin(x)+k} }}}
\end{align*}
\end{frame}

\section{Bestemte integraler}
\begin{frame}{Bestemte integraler}
\begin{itemize}
		\setlength\itemsep{1em}
	\item<1-> Vi vil bestemme arealer under grafer for funktioner.
	\item<2-> Arealet mellem grafen for $f$ og $x$-asksen i intervallet $[a,b]$ er givet ved
	\begin{align*}
	F(b)-F(a),
	\end{align*} 
	hvor $F$ er en stamfunktion til $f$.
	\item<3-> Derfor defineres det bestemte integral af $f$ i intervallet $[a,b]$ til
	\begin{align*}
	\int_a^b f(x)\, dx =[F(x)]_a^b=F(b)-F(a).
	\end{align*}
	\item<4-> Eksempel: Bestem $\int_0^1 x^2 \, d x\setbeamercovered{}\onslide<5->{=[\frac{1}{3}x^3]_0^1\onslide<6->{=\frac{1}{3}}}$.
\end{itemize}
\end{frame}

\section{Regneregler for bestemte integraler}
\begin{frame}{Regneregler for bestemte integraler}

\begin{itemize}
\setlength\itemsep{1em}
 \item<1-> Vi har følgende generelle regneregler for bestemte integraler
 	\begin{align*}
 \onslide<1->{\int_a^b cf(x) \, d x&=c\int_a^b f(x)\, dx}\\
 \onslide<2->{\int_a^b f(x)\pm g(x) \, d x&=\int_a^b f(x)\, dx\pm \int_a^b g(x) \, dx.}
 %\int_a^b f(x)\, dx&=\int_a^c f(x) \, dx+\int_c^b f(x) \, dx\\
 %\int_a^b f(x)\, dx&=-\int_b^a f(x) \, dx.
 \end{align*}
 \item<3-> Eksempler: Udregn
\end{itemize}
 \begin{align*}
\onslide<3->{\int_1^2 \frac{1}{2x}-1 \, dx\setbeamercovered{}\onslide<4->{=\frac{1}{2}\int_{1}^2 \frac{1}{x}\, dx}\onslide<5->{-\int_1^2 1 \, dx}\onslide<6->{=\frac{1}{2}[\ln(x)]_1^2}\onslide<7->{-[x]_1^2}\onslide<8->{=\frac{1}{2}\ln(2)}\onslide<9->{-1}}\\
\onslide<10->{\int_0^4 3x^2+3e^x \, dx\setbeamercovered{}\onslide<11->{=\int_0^4 3x^2\, dx}\onslide<12->{+3\int_0^4e^x \, dx}\onslide<13->{=[x^3]_0^4}\onslide<14->{+3[e^x]_0^4}\onslide<15->{=64}\onslide<16->{+3e^4-3}\onslide<17->{=61+3e^4}}
\end{align*}
\end{frame}


