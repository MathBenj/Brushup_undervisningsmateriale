\section{Førstegradsligninger}
\begin{frame}{Førstegradslignigner}
\begin{itemize}
		\setlength\itemsep{1em}
	\item<1-> En ligning består af to udtryk adskilt af et lighedstegn, hvor mindst et af udtrykkene indeholder en ubekendt variabel.
	\item<2-> En ligning løses ved at bestemme alle tal der kan indsættes på variablens plads så ligningen er sand.
	\item<3-> Eksempler:
\end{itemize}
	\begin{align*}
\onslide<3->{x+2=7,}&& \onslide<4->{2(x-1)=2x+3,}\\ \onslide<5->{x^2=9,}&&\onslide<6->{ x+1=\frac{1}{2}(2x+2).}
\end{align*}
\end{frame}

\begin{frame}{Førstegradsligninger}
\begin{itemize}
		\setlength\itemsep{1em}
	%\item For komplicerede ligninger kan vi ikke bare aflæse løsningerne.
	\item<1-> Ligninger kan reduceres med følgende regler:
	\begin{itemize}
			\setlength\itemsep{1em}
		\item<2-> Man må lægge til og trække fra med det samme tal på begge sider af et lighedstegn.
		\item<3-> Man må gange og dividere med det samme tal (undtagen 0) på begge sider af et lighedstegn.
	\end{itemize}
%	\item Strategi: Saml alle led som indeholder den ubekendte på den ene side af lighedstegnet og alle andre led på den modsatte side af lighedstegnet.
	\item<4-> Eksempler: Løs ligningerne

\end{itemize}
	\begin{align*}
\onslide<4->{4x+7=3(x+8),}&& \onslide<5->{\frac{2x+1}{4x}=3,}&& \onslide<6->{\pi x=3-2x.}\\
\end{align*}
\end{frame}
\setbeamercovered{} 
\begin{frame}{Førstegradsligninger}
	\begin{itemize}
		\item Svar:
	\end{itemize}
	\begin{align*}
\onslide<1->{4x+7=3(x+8),}&& \onslide<1->{\frac{2x+1}{4x}=3,}&& \onslide<1->{\pi x=3-2x.}\\
\onslide<2->{4x+7=3x+24,}&& \onslide<4->{2x+1=12x,}&& \onslide<7->{\pi x+2x=3.}\\
\onslide<3->{x=17,}&& \onslide<5->{1=10x,}&& \onslide<8->{x(\pi+2)=3}\\
				&& \onslide<6->{x=\frac{1}{10},}&& \onslide<9->{x=\frac{3}{(\pi+2)}}\\
\end{align*}
\end{frame}


\setbeamercovered{transparent} 
\section{Andengradsligninger}
\begin{frame}{Andengradsligninger}
\begin{itemize}
		\setlength\itemsep{1em}
	\item<1-> Vi betragter andengradsligninger på formen
	\begin{align}\label{eq:lig1}
	ax^2+bx+c=0,
	\end{align}
%	hvor $a\neq 0$.
	\item<2-> Løsningerne til~\eqref{eq:lig1} er
	\begin{align*}
	x=\frac{-b\pm \sqrt{b^2-4ac}}{2a}.
	\end{align*}
	\item<3-> Vi har nu tre tilfælde
	\begin{itemize}
			\setlength\itemsep{1em}
		\item<4-> Hvis $b^2-4ac>0$ har \eqref{eq:lig1} to reelle løsninger.
		\item<5-> Hvis $b^2-4ac=0$ har \eqref{eq:lig1} én reel løsning.
		\item<6-> Hvis $b^2-4ac<0$ har \eqref{eq:lig1} to komplekst konjugerede rødder.
	\end{itemize}
	\item<7-> Eksempler: Løs ligningerne
\end{itemize}
	\begin{align*}
\onslide<7->{x^2+5x+4=0,}&&\onslide<8->{x^2-3x+10=8.}
\end{align*}
\end{frame}
\setbeamercovered{} 
\begin{frame}{Andengradsligninger}
	\begin{itemize}
		\item Svar:
		\begin{align*}
	\onslide<1->{x^2+5x+4=0,}&&\onslide<1->{x^2-3x+10=8.}\\
	\onslide<2->{x=\frac{-5\pm \sqrt{25-4\cdot 1\cdot 4}}{2\cdot 1},}&&\onslide<6->{x^2-3x+2=0.}\\
	\onslide<3->{x=\frac{-5\pm \sqrt{9}}{2},}&&\onslide<7->{x=\frac{-(-3)\pm \sqrt{9-4\cdot 1\cdot 2}}{2\cdot 1}.}\\
	\onslide<4->{x=\frac{-5\pm 3}{2},}&&\onslide<8->{x=\frac{3\pm 1}{2}.}\\
	\onslide<5->{x=-1,\ x=-4,}&&\onslide<9->{x=1,\ x=2.}\\
		\end{align*}
	\end{itemize}
\end{frame}

\setbeamercovered{transparent} 
\begin{frame}{Andengradsligninger}{Særtilfælde}
\begin{itemize}
		\setlength\itemsep{1em}
	\item<1-> Hvis $b=0$ reducerer \eqref{eq:lig1} til
	\begin{align*}
	ax^2+c=0.
	\end{align*}
	\begin{itemize}
		\item<2-> Vi har dermed løsningen $x=\pm \sqrt{-c/a}$.
	\end{itemize}
	\item<3-> Hvis $c=0$ reducerer \eqref{eq:lig1} til
	\begin{align*}
	ax^2+bx=0.
	\end{align*}
	\begin{itemize}
			\setlength\itemsep{1em}
		\item<4-> Sætter vi $x$ udenfor en parentes får vi, at $x(ax+b)=0$.
		\item<5-> Nulreglen giver så at løsningerne er $x=0$ og $x=-b/a$.
	\end{itemize}
	\item<6-> Eksempler: Løs ligningerne

\end{itemize}
	\begin{align*}
\onslide<6->{2x^2-72=0,}&& \onslide<7->{-x^2+2x=0.}
\end{align*}
\end{frame}

\setbeamercovered{}
\begin{frame}{Andengradsligninger}{Særtilfælde}
	\begin{itemize}
		\item Svar:
	\end{itemize}
\begin{align*}
\onslide<1->{2x^2-72=0,}&& \onslide<5->{-x^2+2x=0.}\\
\onslide<2->{2x^2=72,}&& \onslide<6->{x(-x+2)=0.}\\
\onslide<3->{x^2=36,}&& \onslide<7->{x=0,\ -x+2=0.}\\
\onslide<4->{x=\pm 6,}&& \onslide<8->{x=0,\ x=2.}\\
\end{align*}
\setbeamercovered{transparent}
\end{frame}
\begin{frame}{Andengradsligninger}{Faktorisering}
\begin{itemize}
		\setlength\itemsep{1em}
	\item<1-> Hvis $ax^2+bx+c=0$ har to reelle løsninger $r_1$ og $r_2$ så gælder
	\begin{align*}
	ax^2+bx+c=a(x-r_1)(x-r_2).
	\end{align*}
	\item<2-> Hvis $ax^2+bx+c=0$ har én reel løsning $r$ så gælder
	\begin{align*}
	ax^2+bx+c=a(x-r)^2.
	\end{align*}
	\item<3-> Eksempler: Reducer udtrykket
	\begin{align*}
	\frac{2x^2+2x-4}{x-1}.
	\end{align*}
\end{itemize}
\end{frame}

\begin{frame}{Andengradsligninger}{Faktorisering}
	\begin{itemize}
		\item Svar: Reducer udtrykket
			\begin{align*}
		\frac{2x^2+2x-4}{x-1}.
		\end{align*}
		\begin{itemize}
			\item<2-> Først løser vi andengradsligningen
			\begin{align*}
			2x^2+2x-4=0:
			\end{align*}
			\item<3-> 
			\begin{align*}
			x=\frac{-2\pm \sqrt{4-4\cdot 2\cdot (-4)}}{2\cdot 2}\onslide<4->{=\frac{-2\pm 6}{4}}\onslide<5->{=\begin{cases}
			1\\-2
			\end{cases}}
			\end{align*}
			\item<6-> Ved at faktorisere får vi
			\begin{align*}
		\frac{2x^2+2x-4}{x-1}=\frac{2(x-1)(x-(-2))}{x-1}\onslide<7->{=2x+4}
			\end{align*}
		\end{itemize}
	\end{itemize}
	
\end{frame}
