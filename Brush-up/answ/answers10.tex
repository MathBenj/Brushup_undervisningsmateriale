\section{Eksponentiel og logaritme funktioner}
\begin{enumerate}
	\item Svarene er:
	\begin{align*}
	16,&& \frac{1}{4},&& e^3, &&\frac{1}{8},&&27,&& 1.
	\end{align*}
	\item Svarene er:
	\begin{align*}
	8,&& 3,&& -2,&& 6,&&0.
	\end{align*}
	\item Svarene er:
	\begin{align*}
	1,&&\ln 2,&&\ln 2.
	\end{align*}
	\item Svarene er:
	\begin{align*}
	3,&& 1,&& 2.
	\end{align*}
	\item Svarene er:
	\begin{align*}
	\frac{1}{2}\ln(2),&& 2,&&\frac{3}{2}
	\end{align*}
	\item Svarene er:
	\begin{align*}
	1,&&40,&& 343,&& \frac{1}{9},&& 9.
	\end{align*}
	\item Svaret er $a=\sqrt{2}$ og $b=\sqrt{2}$.
	
	\item Svarene er:
	\begin{align*}
	x=\ln 5,&& x=0,x=2,&& x=\log_2(3)-1,&& x=9,&&x=10.
	\end{align*}
	
	\item Svarene er:
	\begin{align*}
	x=e^4,&& x=3,&&x=2.
	\end{align*}
	
	\item Svarene er:
	\begin{enumerate}
		\item $f(-11)=48$.
		\item $f(10)=6$.
		\item $x=24$.
	\end{enumerate}
	
	\item Da $e^x$ altid er positiv og $\ln x$ er den inverse til $e^x$ må $\ln x$ nødvendigvis kun være defineret på de positive tal. 
	
	\item Svaret er $a=e^{\frac{1}{5}}$.
	
	
	\item Svaret er $a=e^2$ og $b=e$. 
	
	\item Eksponentialfunktionen med den grønne graf har den største fordoblingskonstant, da den vokser langsomst.
	
%	\begin{figure}
%		\centering
%		\begin{tikzpicture}
%		\begin{axis}[xmin=-2,xmax=2,ymin=-0.2,ymax=4,axis x line=center,
%		axis y line=center,ticks=none,restrict y to domain =0:20]
%		\addplot[thick,blue, samples = 200] {exp(x)};
%		\addplot[thick, red, samples = 200] {10^x};
%		\addplot[thick, green, samples = 200] {2^x};
%		\end{axis}
%		\end{tikzpicture}
%		\caption{Opgave~\ref{it:eks1}}
%		\label{fig:eks1}
%	\end{figure}


	\item Svaret er $T_{\frac{1}{2}}= \frac{1}{2}\ln 2$.
	
	
	\item Bruger vi hintet få vi at $g(x)=ba^{-x}$ er spejlingen, så $c=b$ og $d=a^{-1}$.
	
	
	\item\label{it:eks2ans} Vi har at
	\begin{align*}
	\frac{\ln(a^y)}{\ln(a)}=\frac{y\ln(a)}{\ln(a)}=y.
	\end{align*}
	Hvis vi bruger at $x=a^{\log_a(x)}$ så får vi
	\begin{align*}
	a^{\frac{\ln(x)}{\ln(a)}}=a^{\frac{\ln(a^{\log_a(x)})}{\ln(a)}}=a^{\frac{\log_a(x)\ln(a)}{\ln(a)}}=a^{\log_a(x)}=x.
	\end{align*}
	Disse to udregninger viser, at $\frac{\ln x}{\ln a}$ er invers til funktionen $a^x$ og vi har derfor at
	\begin{align*}
	\log_a(x)=\frac{\ln x}{\ln a}.
	\end{align*}
	
	
\end{enumerate}