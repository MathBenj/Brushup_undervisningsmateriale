\newpage
\section{Ligninger og uligheder}
\begin{enumerate}
\item Svarene er:
\begin{align*}
x=3,&& x=-3,&&x=6,&& x=\frac{-3}{4}.
\end{align*}

\item Svarene er:
\begin{align*}
x=-1,&& x=12,&& x=\frac{-7}{2}.
\end{align*}

\item Svarene er: 
\begin{align*}
x<\frac{4}{7},&& x>\frac{5}{2},&& \textup{ingen løsning},&& \frac{7}{2}\leq x.
\end{align*}

\item Svarene er:
\begin{enumerate}
\item $a\neq -1, b\in \mathbb{R}$
\item $a=-1,b\neq 4$
\item $a=-1,b=4$.
\end{enumerate}

\item Svarene er:
\begin{align*}
 \textup{ingen løsning},&& x\in \mathbb{R}.
\end{align*}

\item Svarene er:
\begin{align*}
x=\frac{11}{5},&& x=1,&& x=\frac{7}{2},&& x=-3.
\end{align*}

\item Svarene er:
\begin{align*}
x=\frac{13}{4},&& x=\frac{6}{11}.
\end{align*}

\item Svarene er:
\begin{align*}
x=2\sqrt{2},&& x=\frac{\pi+3}{\pi-\sqrt{2}},&&x=\frac{5}{2}.
\end{align*}

\item Svarene er:
\begin{enumerate}
	\item Kvadrat med hjørner $(0,0),(2,0),(0,2),(2,2)$.
	\item Cirkelskive med centrum $(0,0)$ og $r=2$.
	\item Trapez med hjørner $(0,0),(0,2),(2,6),(2,0)$.
\end{enumerate}


\item Svarene er:
\begin{enumerate}
	\item Rektangel: $ -4\leq x\leq -2 $, $ 0\leq y\leq 4 $.
	\item Cirkelskive: $ (x+2)^2+(y+3)^2\leq 4 $.
	\item Trekant: $ 0\leq x\leq 4 $, $ \frac{1}{2}x\leq y\leq 2 $, eller $0\leq y\leq 2$, $ 0\leq x\leq 2y $.
\end{enumerate}
%\begin{figure}
%\centering
%\begin{tikzpicture}
%\draw[help lines, color=gray!30, dashed] (-4.9,-4.9) grid (4.9,4.9);
%\draw[->,ultra thick] (-5,0)--(5,0) node[right]{$x$};
%\draw[->,ultra thick] (0,-5)--(0,5) node[above]{$y$};
%
%\draw[fill=gray] (-2,-3) circle (2);
%\draw[fill=gray] (0,0)--(0,2)--(4,2)--cycle;
%\draw[fill=gray] (-4,0)--(-2,0)--(-2,4)--(-4,4)--cycle;
%\end{tikzpicture}
%\caption{Find uligheder der beskriver de grå områder.}
%\label{fig:ligninger1}
%\end{figure}

\item Det er klart at $0\leq (a-b)^2$. Bruger vi kvadratsætningerne har vi
\begin{align*}
0 \leq (a-b)^2=a^2+b^2-2ab.
\end{align*}
Ved at flytte lidt rundt på ovenstående ulighed får vi
\begin{align*}
ab\leq \frac{a^2}{2}+\frac{b^2}{2}.
\end{align*}
Bemærk der findes uendeligt mange valg af $a,b,c,d$. Et muligt kunne være $a=1$, $b=1$, $c=0$, $d=1$.


\item Bruger vi kvadratsætningerne får vi
\begin{align*}
(\sqrt{a}+\sqrt{b})^2=a+b+2\sqrt{ab}.
\end{align*}
Da $a,b\geq 0$ er det sidste led positivt. Derfor kan vi fjerne det og opnå følgende ulighed
\begin{align*}
(\sqrt{a}+\sqrt{b})^2\geq a+b.
\end{align*}
Tager vi kvadratroden på begge sider får vi den ønskede ulighed
\begin{align*}
\sqrt{a}+\sqrt{b}\geq \sqrt{a+b}.
\end{align*}
Bemærk at der findes uendeligt mange valg af $a,b,c,d$ som løser opgaven. Et valg kunne være $a=0$, $b=0$, $c=1$, $ d=1 $.
\end{enumerate}