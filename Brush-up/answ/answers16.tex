\section{Tangentligningen og monotoniforhold}
\begin{enumerate}
	
	\item\label{it:mon1} Det ses let at $f'(x)=4x-2$ så $f'(x)=0$ har løsningen $x=\frac{1}{2}$. Vælger vi punkterne $x_1=0$ og $x_2=1$ ser vi at $f'(x_1)=-2$ og at $f(x_2)=2$. Dette giver monotonilinjen som ses i Tabel~\ref{fig:mon1}. Vi ser dermed at $f$ er aftagende i intervallet $ ]-\infty,\frac{1}{2}] $ og voksende i intervallet $[\frac{1}{2},\infty[$.
	
	\begin{table}[h!]
		\centering
		\begin{tabular}{@{}l  c c c@{}}
			$x$      & $0$ 		 & $\frac{1}{2}$	& $1$			\\ \toprule
			$f'(x)$  & $-2$		 &     $0$ 		 	& $2$			\\ \midrule
			$f(x)$   & $\searrow$&					& $\nearrow$	\\ \bottomrule  
		\end{tabular}
		\caption{Opgave~\ref{it:mon1}.}
		\label{fig:mon1}
	\end{table}
	
	
	\item Tangentlignignen er $y=9(x-1)+5=9x-4$.
	
	\item Svarene er: $f'(0)<0$, $f'(2)>0$ og $f'(1)=0$.
	
	
	\item Tangentligningen er $y=x-1$.
	
	\item \label{it:mon2} Ved at differentiere får vi at $f'(x) = 9x^2+6x+1$ og løser vi ligningen $f'(x)=0$ får vi at $x=-\frac{1}{3}$. Vi ser også at $f'(-1))=4$ og at $f'(0)=1$. Dermed har vi at $f$ er voksende på hele den reelle akse med en vendetangent i punktet $x=-\frac{1}{3}$. I Tabel~\ref{fig:mon2}.
	\begin{table}[h!]
	\centering
	\begin{tabular}{@{}l  c c c@{}}
		$x$      & $-1$ 	 & $-\frac{1}{3}$	& $0$			\\ \toprule
		$f'(x)$  & $4$		 &     $0$ 		 	& $1$			\\ \midrule
		$f(x)$   & $\nearrow$&					& $\nearrow$	\\ \bottomrule  
	\end{tabular}
	\caption{Opgave~\ref{it:mon2}.}
	\label{fig:mon2}
\end{table}


	
	\item Tangentligningen er givet ved $y=\frac{1}{2}(x-2)+4$. Sætter vi $y=0$ og løser ligningen får vi at $x=-6$.
	
	\item Ligningen $f'(x)=2$ har løsningen $x=-\frac{1}{3}$ og tangentligningen i $ (-\frac{1}{3},f(-\frac{1}{3})) $ er $ y=2(x+\frac{1}{3}) +\frac{4}{9}=2x+\frac{10}{9}$.
	
	
	\item For at bestemme tangentligningerne for tangenterne til $f$ som går gennem punktet $(\frac{5}{4},0)$ skal vi bestemme hvilke punkter på grafen tangenterne går gennem. Hvis vi vælger et vilkårligt punkt $ (a,f(a)) $ på grafen så er tangentliningen gennem punktet givet ved
	\begin{align*}
	y=f'(a)(x-a)+f(a)=(2a-4)(x-a)+(2-a)^2+1=2ax-4x-a^2+5.
	\end{align*}
	Indsætter vi punktet $(\frac{5}{4},0)$ i denne ligning får vi at
	\begin{align*}
	0=\frac{5}{2}a-a^2.
	\end{align*}
	Løsningerne er $a=0$ og $a=\frac{5}{2}$. Indsættes dette i ligningen for tangenten får vi at
	\begin{align*}
	y_1&=-4(x-0)+5=-4x+5,\\
	y_2&=1(x-\frac{5}{2}) +\frac{5}{4}=x-\frac{5}{4}.
	\end{align*}
	
	
	\item Da $f$ er symmetrisk giver kædereglen at
	\begin{align*}
	f'(-x)=-f'(x).
	\end{align*}
	Derfor bliver tangentligningen $y= -2(x+1)-1=-2x-3$
	
	\item Tangentligningen er $y=f'(g(-3))g'(-3)(x+3)+f(g(-3))=-2x-4$.
	
	\item Funktionen er konstant.
	
	\item Svarene er:
	\begin{enumerate}
		\item $y=x$
		\item $-x+\frac{\pi}{2}$.
		\item $\dfrac{\pi^2}{16}$.
	\end{enumerate}


\end{enumerate}