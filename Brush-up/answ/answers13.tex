\newpage
\section{Differentiabilitet 1}
\begin{enumerate}
	\item Svaret er $f'(1)=10$.

	\item Svarene er:
	\begin{align*}
	f'(x)=3x^2,&& f'(x)=1,&& f'(x)=\frac{1}{2}x^{-\frac{1}{2}},&& f'(x)=-x^{-2},&& f'(x)=12x^5.
	\end{align*}
	
	\item Svarene er: 
	\begin{align*}
	f'(x)=6e^{2x}-\frac{1}{2x},&& f'(x)=\frac{1}{2}\cos x,&& f'(x)=\frac{2}{x}-\frac{1}{4}e^{-\frac{1}{12}x}.
	\end{align*}
	
	\item Svarene er:
	\begin{align*}
	f'(x)=21x^6+8x^3-6x,&&f(x)=10x^4+\frac{9}{2}x^{\frac{1}{2}}+4x^{-3},&&f(x)=\frac{3}{2}x^{-\frac{1}{2}}-x^{-2}.
	\end{align*}
	
		
	\item \label{it:diff12ans} Svarene er:
	\begin{enumerate}
		\item Den første blå og den anden røde hører sammen.
		\item Den anden blå og den tredje røde hører sammen.
		\item Den tredje blå og den første røde hører sammen.
	\end{enumerate}
%	\begin{figure}
%		\centering
%		
%		\begin{minipage}{0.3\linewidth}
%			\begin{tikzpicture}[scale=0.5]
%			\begin{axis}[xmin=-2,xmax=2,ymin=-2,ymax=2,axis x line=center,
%			axis y line=center,ticks=none]
%			\addplot[thick,blue, samples = 600] {3/4*x^3+7/8*x^2-7/8*x-3/4};
%			\end{axis}
%			\end{tikzpicture}
%		\end{minipage}
%		\begin{minipage}{0.3\linewidth}
%			\begin{tikzpicture}[scale=0.5]
%			\begin{axis}[xmin=-2,xmax=2,ymin=-2,ymax=2,axis x line=center,
%			axis y line=center,ticks=none]
%			\addplot[thick,blue, samples = 200] {-0.5*x-0.5};
%			\end{axis}
%			\end{tikzpicture}
%		\end{minipage}
%		\begin{minipage}{0.3\linewidth}
%			\begin{tikzpicture}[scale=0.5]
%			\begin{axis}[xmin=-2,xmax=2,ymin=-2,ymax=2,axis x line=center,
%			axis y line=center,ticks=none, restrict y to domain=-2:2]
%				\addplot[thick,blue, samples = 600] {sqrt(x+2)-1};
%			\end{axis}
%			\end{tikzpicture}
%		\end{minipage}
%		
%		\begin{minipage}{0.3\linewidth}
%			\begin{tikzpicture}[scale=0.5]
%			\begin{axis}[xmin=-2,xmax=2,ymin=-2,ymax=2,axis x line=center,
%			axis y line=center,ticks=none]
%				\addplot[thick,red, samples = 600] {1/(2*sqrt(x+2))};
%			\end{axis}
%			\end{tikzpicture}
%		\end{minipage}
%		\begin{minipage}{0.3\linewidth}
%			\begin{tikzpicture}[scale=0.5]
%			\begin{axis}[xmin=-2,xmax=2,ymin=-2,ymax=2,axis x line=center,
%			axis y line=center,ticks=none]
%			\addplot[domain=-2:2,thick,red, samples = 600] {9/4*x^2+7/4*x-7/8};
%			\end{axis}
%			\end{tikzpicture}
%		\end{minipage}
%		\begin{minipage}{0.3\linewidth}
%			\begin{tikzpicture}[scale=0.5]
%			\begin{axis}[xmin=-2,xmax=2,ymin=-2,ymax=2,axis x line=center,
%			axis y line=center,ticks=none]
%			\addplot[thick,red, samples = 200] {-0.5};					
%			\end{axis}
%			\end{tikzpicture}
%		\end{minipage}
%		\caption{Opgave~\ref{it:diff12}}
%		\label{fig:diff12}
%	\end{figure}	
	
	\item Ved brug af \href{https://www.geogebra.org/m/eTmzBFEq}{Geogebra} ses at
	\begin{align*}
	\lim_{h\to 0^+} \frac{f(1+h)-f(1)}{h}= \infty.
	\end{align*}
	Funktionen $f$ er ikke differentiabel i $1$ da det ville kræve at differentialkvotienten er endelig.
	
	\item Svarene kan være:
	\begin{enumerate}
		\item Vi indsætter $f$ i differentialkvotienten og reducerer indtil det er let at tage grænsen $h\to 0$.
		\begin{align*}
		\lim_{h\to 0}\frac{k-k}{h}=0.
		\end{align*}
		\item Vi indsætter $f$ i differentialkvotienten og reducerer indtil det er let at tage grænsen $h\to 0$.
		\begin{align*}
		\lim_{h\to 0}\frac{x+h-x}{h}=\lim_{h\to 0}\frac{h}{h}=1.
		\end{align*}
		\item Vi indsætter $f$ i differentialkvotienten og reducerer indtil det er let at tage grænsen $h\to 0$.
		\begin{align*}
		\lim_{h\to 0}\frac{k(x+h)-kx}{h}=\lim_{h\to 0}\frac{kh}{h}=k.
		\end{align*}
	\end{enumerate}
	
	\item Svarene er:
	\begin{align*}
	x=1,&& x=0,x=4.
	\end{align*}
	
	\item Svarene er:
	\begin{align*}
	f'(x)=x^{-\frac{2}{3}},&& f'(x)=6x^2-2x,&& f'(x)=40x^3+228x^2+64x-76.
	\end{align*}
	
	\item Vi har $(f+g)'(x)=3x^2-2x$ og $(f-g)'(x)=-3x^2+10x-6$.
	
	\item Svarene er: 
	\begin{align*}
	\lim_{h\to 0^+} \frac{f(h)-f(0)}{h}=-\infty,\quad \textup{og}\quad \lim_{h\to 0^-} \frac{f(h)-f(0)}{h}=\infty.
	\end{align*}
	Funktionen $f$ er ikke differentiabel i $x=0$ da det ville kræve at grænserne ovenfor er endelige.
	
	\item Svarene er:
	\begin{align*}
	f'(x)=\frac{-1}{2}x^{-\frac{3}{2}}-x^{-2},&& f'(x)=\frac{15}{4}x^{\frac{11}{4}},&& f'(x)=\cos(x),&&f(x)=-\frac{2}{x}.
	\end{align*}
	
	

	
	\item \label{it:diff11} Svarene er:
		\begin{enumerate}
		\item Den første blå og den tredje røde hører sammen.
		\item Den anden blå og den første røde hører sammen.
		\item Den tredje blå og den anden røde hører sammen.
	\end{enumerate}
%		\begin{figure}
%		\centering
%		
%		\begin{minipage}{0.3\linewidth}
%			\begin{tikzpicture}[scale=0.5]
%			\begin{axis}[xmin=-2,xmax=2,ymin=-2,ymax=2,axis x line=center,
%				axis y line=center,ticks=none]
%				\addplot[domain=0:2,thick,blue, samples = 600] {(x^2)^(1/3)};
%				\addplot[domain=-2:0,thick,blue,samples=600] {((-x)^2)^(1/3)};
%				\end{axis}
%			\end{tikzpicture}
%		\end{minipage}
%		\begin{minipage}{0.3\linewidth}
%			\begin{tikzpicture}[scale=0.5]
%			\begin{axis}[xmin=-2,xmax=2,ymin=-2,ymax=2,axis x line=center,
%			axis y line=center,ticks=none]
%			\addplot[thick,blue, samples = 200] {x*sqrt(2-x)};
%			\end{axis}
%			\end{tikzpicture}
%		\end{minipage}
%			\begin{minipage}{0.3\linewidth}
%		\begin{tikzpicture}[scale=0.5]
%		\begin{axis}[xmin=-2,xmax=2,ymin=-2,ymax=2,axis x line=center,
%		axis y line=center,ticks=none]
%		\addplot[domain=0:2,thick,blue, samples = 200] {x*sqrt(4-x^2)};
%		\addplot[domain=-2:0,thick,blue, samples = 200] {x*sqrt(4-(x)^2)};		
%		\end{axis}
%		\end{tikzpicture}
%	\end{minipage}
%
%		\begin{minipage}{0.3\linewidth}
%	\begin{tikzpicture}[scale=0.5]
%	\begin{axis}[xmin=-2,xmax=2,ymin=-2,ymax=2,axis x line=center,
%	axis y line=center,ticks=none]
%	\addplot[thick,red, samples = 200] {sqrt(2-x)-x/(2*sqrt(2-x))};
%	\end{axis}
%	\end{tikzpicture}
%\end{minipage}
%\begin{minipage}{0.3\linewidth}
%	\begin{tikzpicture}[scale=0.5]
%	\begin{axis}[xmin=-2,xmax=2,ymin=-2,ymax=2,axis x line=center,
%	axis y line=center,ticks=none]
%	\addplot[domain=0:2,thick,red, samples = 200] {sqrt(4-x^2)-x^2/sqrt(4-x^2)};
%\addplot[domain=-2:0,thick,red, samples = 200] {sqrt(4-x^2)-x^2/sqrt(4-x^2)};
%	%\addplot[domain=-2:0,thick,blue,samples=600] {((-x)^2)^(1/3)};
%	\end{axis}
%	\end{tikzpicture}
%\end{minipage}
%\begin{minipage}{0.3\linewidth}
%	\begin{tikzpicture}[scale=0.5]
%	\begin{axis}[xmin=-2,xmax=2,ymin=-2,ymax=2,axis x line=center,
%	axis y line=center,ticks=none,restrict y to domain=-4:4]
%		\addplot[domain=0:2,thick,red, samples = 600] {2/3*(x)^(-1/3)};
%		\addplot[domain=-2:0,thick,red, samples = 600] {-2/3*(-x)^(-1/3)};						
%	\end{axis}
%	\end{tikzpicture}
%	\end{minipage}
%	\caption{Opgave~\ref{it:diff11}}
%	\label{fig:diff11}
%	\end{figure}	
	
	\item\label{it:diff13} Svarene er: 
	\begin{enumerate}
		\item $x=\pi k$ hvor $k$ er et heltal.
		\item Vi har at 
	\begin{align*}
	f(\pi k)=\pi k+2\cos(\pi k)=\pi k-(-1)^k 2,
	\end{align*}
	hvilket viser, at når $k$ er lige ligger $(x,f(x)) $ på linjen $y=x-2$ og når $k$ er ulige ligger punktet på $y=x+2$.
	\end{enumerate}
	
%	\begin{figure}
%		\centering
%		\begin{tikzpicture}
%		\begin{axis}[axis x line=center,axis y line=center,ticks=none]
%		\addplot[thick,blue, samples = 600] {x-2*cos(deg(x))};
%		\addplot[thick,red, samples = 600] {x-2};
%		\addplot[thick,red, samples = 600] {x+2};												
%		\end{axis}
%		\end{tikzpicture}
%		\caption{Opgave~\ref{it:diff13}}
%		\label{fig:diff13}
%	\end{figure}
%	
	\item Svarene er:
	\begin{enumerate}
	\item Vi indsætter $f$ i differentialkvotienten og reducerer indtil det er let at tage grænsen $h\to 0$.
	\begin{align*}
	\lim_{h\to 0}\frac{(x+h)^2-x^2}{h}=\lim_{h\to 0}\frac{h^2+2xh}{h}=\lim_{h\to 0}h+2x=2x.
	\end{align*}
	\item Vi indsætter $f$ i differentialkvotienten og reducerer indtil det er let at tage grænsen $h\to 0$.
	\begin{align*}
	\lim_{h\to 0}\frac{\frac{1}{x+h}-\frac{1}{x}}{h}=\lim_{h\to 0}\frac{\frac{x-(x+h)}{x(x+h)}}{h}=\lim_{h\to 0}\frac{\frac{-h}{x(x+h)}}{h}=\lim_{h\to 0}\frac{-1}{x(x+h)}=-\frac{1}{x^2}.
	\end{align*}
	\item Vi indsætter $f$ i differentialkvotienten og reducerer indtil det er let at tage grænsen $h\to 0$.
	\begin{align*}
	\lim_{h\to 0}\frac{\sin(x+h)-\sin(x)}{h}&=\lim_{h\to 0}\frac{\sin(x)\cos(h)+\sin(h)\cos(x)-\sin(x)}{h}\\
	&=\sin(x)\lim_{h\to 0}\frac{\cos(h)-1}{h}+\cos(x)\lim_{h\to 0}\frac{\sin(h)}{h}\\
	&=0\sin(x)+1\cos(x)=\cos(x)
	\end{align*}
	\item Vi indsætter $f$ i differentialkvotienten og reducerer indtil det er let at tage grænsen $h\to 0$.
	\begin{align*}
	\lim_{h\to 0}\frac{\sqrt{x+h}-\sqrt{x}}{h}&=\lim_{h\to 0}\frac{(\sqrt{x+h}-\sqrt{x})(\sqrt{x+h}+\sqrt{x})}{h(\sqrt{x+h}+\sqrt{x})}\\
	&=\lim_{h\to 0}\frac{x+h-x}{h(\sqrt{x+h}+\sqrt{x})}\\
	&=\lim_{h\to 0}\frac{1}{\sqrt{x+h}+\sqrt{x}}=\frac{1}{2\sqrt{x}}.
	\end{align*}
\end{enumerate}
	
	\item Svarene er:
	\begin{align*}
	f'(x)=\frac{3}{x},&& f'(x)=3e^{3x}.
	\end{align*}
	
	\item \label{it:diff14}	Bruger vi hintet får vi at
	\begin{align*}
	\lim_{h\to 0} \frac{\frac{\sin(h)}{h}-1}{h}=\lim_{h\to 0} \frac{\sin(h)-h}{h^2}=0.
	\end{align*}
	
	\item\label{it:diff15} Svarene er:	
	\begin{enumerate}
		\item Da en tangent til en cirkel står vinkelret på radius vil den stiplede linje tegnet i forlængelse af radius på Figur~\ref{fig:diff15} dele $\theta$ i en del på $\frac{\pi}{2}$ og en som er præcis $\phi$. Dermed er $\theta=\frac{\pi}{2}+\phi$.
		\item Hvis vi isolerer for $\phi$ får vi at 
		\begin{align*}
		\phi=\frac{s}{r},
		\end{align*}
		og dermed bliver
		\begin{align*}
		\theta=\frac{\pi}{2}+\frac{s}{r}.
		\end{align*}
		
		\item Differentierer vi $ \theta $ i forhold til $s$ i formlen ovenfor følger det det direkte at
		\begin{align*}
		\frac{d \theta}{d s}= \frac{1}{r}.
		\end{align*}
	\end{enumerate}
	
% 	\begin{figure}
% 		\centering
% \begin{tikzpicture}
% \begin{axis}[xmin=-0.2,xmax=1,ymin=-0.2,ymax=1.2,axis x line=center,
% axis y line=center, axis equal,ticks =  none]
% \pgfmathsetmacro\foo{sqrt(2)/2}
% %\draw[fill=gray!40] (axis cs:0,0)--(axis cs:1,0)--(axis cs:\foo,\foo);		
% %\draw[fill=gray!40] (axis cs:1,0)  arc[start angle=0, end angle=45,radius={transformdirectionx(1)}];

% \addplot[blue,domain=0:2*pi,thick, samples=100] ({cos(deg(x))},{sin(deg(x))}) node[black,below,left,pos=0.0625] {$s$};
% \addplot[domain=0:\foo,thick] {x} node[above,pos=0.5] {$r$};
% \addplot[domain=\foo:1,thick,dotted] {x};
% \addplot[thick] {-x+2*\foo};
% \addplot[domain=\foo:1,dotted,thick] {\foo};
% \addplot[domain=0:pi/4,thick,samples=100] ({0.2*cos(deg(x))},{0.2*sin(deg(x))}) node[label={[label distance=2pt]0:\small$\phi$},pos=0.5] {};

% \addplot[domain=0:3*pi/4,thick,samples=100] ({\foo+0.2*cos(deg(x))},{\foo+0.2*sin(deg(x))}) node[label={[label distance=2pt]0:\small$\theta$},pos=0.5] {};

% \node[fill, circle, inner sep=1pt] at (axis cs:\foo,\foo) [label=below: $P$]{};
% \end{axis}
% \end{tikzpicture}
% \caption{Opgave~\ref{it:diff15}}
% \label{fig:diff15}
% 	\end{figure}
	\end{enumerate}