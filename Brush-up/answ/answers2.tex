\section{Kvadratsætninger}
\begin{enumerate}
\item Svarene er:
\begin{align*}
x^2+1+2x,&& 4x^2+9-12x,&& x^2,&& 9a^2+4b^2-6ab.
\end{align*}
\item Svarene er:
\begin{align*}
\frac{x+3}{2x},&& \frac{2x+3}{2x-3},&& \frac{2x+6}{x},&& \frac{x-2y}{2}.
\end{align*}
\item Centrumskoordinaterne og radierne er:
\begin{align*}
(0,0), r=1,&& (1,-1), r=5,&& (-2,0), r=2.
\end{align*}
\item Vi har at
\begin{align*}
99^2-101^2&=(99-101)(99+101)=-2(200)=-400,\\
999^2&=(1000-1)^2=1000^2+1^2-2000= 998001,\\
499^2-501^2&=(499-501)(499+501)=-2(1000)= -2000,\\ 
99998^2-100002^2&=(99998-100002)(99998+100002)=-4(200000)=-800000.
\end{align*}
\item Ved at reducere fås
\begin{align*}
8-4a,&& 2x,&&2x-3.
\end{align*}
\item Centrumskoordinaterne og radierne er:
\begin{align*}
(3,4),r=5,&& \Big( \frac{1}{2},-\frac{1}{2}\Big), r=1.
\end{align*}
\item \label{it:1ans} Arealet af figuren i Figur~\ref{fig:1} er $(a+b)^2$ hvilket figuren viser kan beskrives som $a^2+b^2+2ab$.
\item \label{it:2ans} Den højre del af Figur~\ref{fig:2} viser, at summen af det grå areal og det skraverede areal er $(a+b)(a-b)$. Den venstre figur viser, at dette areal er det samme som $a^2-b^2$.
\item \label{it:3ans} Det totale areal som ses i Figur~\ref{fig:3} kan beskrives både som $(a+b)^2$ og som $c^2+4( 1/2) ab$. Dette giver ligningen
\begin{align*}
c^2+4 \frac{1}{2}ab=(a+b)^2,
\end{align*}
som kan reduceres til $c^2=a^2+b^2$.

\item Svarene er:
\begin{align*}
a^2+36b^2+12ab,&& 16-a^2,&& x^2+\frac{1}{x^2}+2.
\end{align*}
\item Ved at sætte brøkerne på fælles nævner fås
\begin{align*}
\frac{7a +b}{4a^2-4b^2}-\frac{3}{4a+4b}-\frac{3}{4a-4b}&=\frac{7a+b-3(a-b)-3(a+b)}{4(a-b)(a+b)}\\
&=\frac{a+b}{4(a-b)(a+b)}\\
&=\frac{1}{4a-4b}.
\end{align*}
\item \label{it:4ans} Lader vi $d=b+c$ får vi
\begin{align*}
(a+b+c)^2= a^2+d^2+2ad&= a^2+(b+c)^2+2a(b+c)\\&=a^2+b^2+c^2+2bc+2ab+2ac.
\end{align*}
\item \label{it:ex13ans} Dividerer vi med $a$ får vi
\begin{align*}
x^2+\frac{b}{a}x+\frac{c}{a}=0.
\end{align*}
Hvis vi skal omskrive dette så vi får en parentes $(x+k)^2$ på venstresiden må $\frac{b}{a}x$ være det dobbelte produkt hvilket betyder, at
\begin{align*}
k=\frac{b}{2a}.
\end{align*}
For at samle parentesen skal vi så lægge $k^2$ til på begge sider, hvilket giver
\begin{align*}
x^2+2\frac{b}{2a} x+\frac{b^2}{4a^2}+\frac{c}{a}=\frac{b^2}{4a^2}.
\end{align*}
Samler vi parentesen og reducerer får vi
\begin{align*}
\Big( x+\frac{b}{2a}\Big)^2=\frac{b^2-4ac}{4a^2},
\end{align*}
hvilket medfører at $d=b^2-4ac$.

% \begin{figure}
% \centering
% \begin{tikzpicture}
% \draw (0,0)--(5,0)--(5,5)--(0,5)-- cycle;
% \draw[dashed] (0,4)--(5,4);
% \draw[dashed] (4,0)--(4,5);
% \node at (2,0) [label=below: $a$] {};
% \node at (4.5,0) [label=below:$b$] {};
% \node at (0,2) [label=left: $a$] {};
% \node at (0,4.5) [label=left:$b$] {};
% \node at (2,5) [label=above: $a$] {};
% \node at (4.5,5) [label=above:$b$] {};
% \node at (5,2) [label=right: $a$] {};
% \node at (5,4.5) [label=right:$b$] {};
% \end{tikzpicture}
% \caption{Opgave~\ref{it:1}}
% \label{fig:1ans}
% \end{figure}
% %
% \begin{figure}
% \centering
% \begin{tikzpicture}
% \draw (0,0)--(0,4)--(4,4)--(4,0)--cycle;
% \draw[pattern=north east lines] (0,0)--(3,0)--(3,1)--(0,4)--cycle;
% \draw[fill=gray] (3,1)--(0,4)--(4,4)--(4,1)--cycle;
% \node at (1.5,0) [label=below: $a-b$] {};
% \node at (3.5,0) [label=below:$b$] {};
% \node at (0,2) [label=left: $a$] {};
% \node at (2,4) [label=above: $a$] {};
% \node at (4,2.5) [label=right: $a-b$] {};
% \node at (4,0.5) [label=right:$b$] {};
% %%%Newfig
% \draw (8,0)--(11,0)--(11,5)--(8,5)--cycle;
% \draw[pattern=north east lines] (8,0)--(8,4)--(11,1)--(11,0)--cycle;
% \draw[fill=gray] (8,4)--(8,5)--(11,5)--(11,1)-- cycle;
% \node at (9.5,0) [label=below: $a-b$] {};
% \node at (8,2) [label= left: $a$] {};
% \node at (8,4.5) [label= left: $b$] {};
% \node at (9.5,5) [label= above: $a-b$] {};
% \node at (11,3) [label= right: $a$] {};
% \node at (11,0.5) [label= right: $b$] {};
% \end{tikzpicture}
% \caption{Opgave~\ref{it:2}}
% \label{fig:2ans}
% \end{figure}
% \begin{figure}
% \centering
% \begin{tikzpicture}[auto]
% \draw (0,0)--(0,5)--(5,5)--(5,0)--cycle;
% \draw (0,3)-- node {$c$} (2,0) ;
% \draw (0,3)--node {$c$} (3,5);
% \draw (3,5)--node {$c$} (5,2);
% \draw (2,0)-- node {$c$} (5,2);
% \node at (1,0) [label=below: $a$] {};
% \node at (3.5,0) [label=below: $b$] {};
% \node at (0,1.5) [label=left: $b$] {};
% \node at (0, 4) [label=left: $a$] {};
% \node at (1.5,5) [label=above: $b$] {};
% \node at (4,5) [label=above: $a$] {};
% \node at (5,1)  [label= right: $a$] {};
% \node at (5,3.5) [label=right: $b$] {};
% \end{tikzpicture}
% \caption{Opgave~\ref{it:3}}
% \label{fig:3ans}
% \end{figure}
\end{enumerate}