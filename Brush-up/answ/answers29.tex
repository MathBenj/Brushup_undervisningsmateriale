\section{Vektorer i rummet}
\begin{enumerate}
	\item Svarene er:
	\begin{align*}
	\begin{bmatrix}
	21\\3\\12
	\end{bmatrix},&&\begin{bmatrix}
	-2\\1\\-8
	\end{bmatrix},&&\begin{bmatrix}
	9\\0\\12
	\end{bmatrix},&&\begin{bmatrix}
	5\\2\\-4
	\end{bmatrix},&&\sqrt{66},&&\sqrt{69},&&45,&& \begin{bmatrix}
	12\\-48\\-9
	\end{bmatrix}
	\end{align*}
	
	\item Nej
	
	\item Prikker vi vektorerne $\vec{u}+t\vec{v}$ og $\vec{u}-t\vec{v}$ får vi ligningen
	\begin{align*}
	0=4-t^2+4-4t^2+16-t^2=24-6t^2.
	\end{align*}
	Denne har løsningerne $t=\pm 2$.
	
	\item Vi har at
	\begin{align*}
	\vec{u}\times \vec{v}=\begin{bmatrix}
	-8\\5\\6
	\end{bmatrix},\quad \textup{og}\quad \norm{\vec{u}\times \vec{v}}=5\sqrt{5}.
	\end{align*}

	\item Lad 
	\begin{align*}
	\vec{u}=\begin{bmatrix}
	u_1\\u_2\\u_3.
	\end{bmatrix}.
	\end{align*}
	Så er
	\begin{align*}
	\vec{u}\times \vec{u}=\begin{bmatrix}
	u_2u_3-u_2u_3\\-(u_1u_3-u_1u_3)\\u_1u_2-u_1u_2
	\end{bmatrix}=\begin{bmatrix}
	0\\0\\0
	\end{bmatrix}.
	\end{align*}
	\item Vi har at
		\begin{align*}
	\vec{u}\times t\c\vec{v}=t\begin{bmatrix}
	-3\\-2\\3
	\end{bmatrix},\quad \textup{og}\quad \norm{\vec{u}\times t \vec{v}}=\abs{t}\sqrt{22}.
	\end{align*}
	Dermed har parallelogrammet areal $3$ for $t=\pm \frac{3}{\sqrt{22}}$.

	\item Svarene er
	\begin{align*}
	\begin{bmatrix}
	-2\\-6\\-4
	\end{bmatrix},&&3\begin{bmatrix}
	1\\-3\\-1
	\end{bmatrix},&& 9,&& \begin{bmatrix}
	2\\6\\4
	\end{bmatrix},&& 3\begin{bmatrix}
	1\\-3\\-1
	\end{bmatrix},&&-18.
	\end{align*}
	
	\item Vi har at
	\begin{align*}
	\vec{u}\c \vec{v}=u_1v_1+u_2v_2+u_3v_3=v_1u_1+v_2u_2+v_3u_3=\vec{v}\c \vec{u}.
	\end{align*}
	
	\item Vi har at
	\begin{align*}
	\norm{k \vec{u}}=\sqrt{k^2u_1^2+k^2u_2^2+k^2u_3^2}=\sqrt{k^2}\sqrt{u_1^2+u_2^2+u_3^2}= \abs{k}\norm{\vec{u}}.
	\end{align*} 	
	\item Bemærk først at $\vec{u} \c \vec{u}=u_1^2+u_2^2+u_3^2=\norm{\vec{u}}^2$. Bruger vi dette får vi
	\begin{align*}
	\norm{\vec{u}+\vec{v}}^2&=(\vec{u}+\vec{v})\c (\vec{u}+\vec{v})=(u_1+v_1)^2+(u_2+v_2)^2+(u_3+v_3)^2\\&=u_1^2+v_1^2+2u_1v_1+u_2^2+v_2^2+2u_2v_2+u_3^2+v_3^2+2u_3v_3\\
	&=\norm{\vec{u}}^2+\norm{\vec{v}}^2+2(\vec{u}\c \vec{v}).
	\end{align*}
	
	\item Lad 
	\begin{align*}
	\vec{u}=\begin{bmatrix}
	u_1\\u_2\\u_3
	\end{bmatrix}
	\quad \textup{og} \quad \vec{v}\begin{bmatrix}
	v_1\\v_2\\v_3
	\end{bmatrix}.
	\end{align*}
	Så er
	\begin{align*}
	\vec{u} \times \vec{v}=\begin{bmatrix}
	u_2v_3-u_3v_2\\ u_3v_1-u_1v_3\\ u_1v_2-u_2v_1
	\end{bmatrix}
	\end{align*}
	og vi får
	\begin{align*}
	\vec{u}\c (\vec{u}\times \vec{v})= u_1(u_2v_3-u_3v_2)+u_2( u_3v_1-u_1v_3)+u_3(u_1v_2-u_2v_1)=0,
	\end{align*}
	samt 
	\begin{align*}
	\vec{v}\c (\vec{u}\times \vec{v})= v_1(u_2v_3-u_3v_2)+v_2( u_3v_1-u_1v_3)+v_3(u_1v_2-u_2v_1)=0.
	\end{align*}
\end{enumerate}