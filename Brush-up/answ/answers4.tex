\section{Rødder}
\begin{enumerate}
\item Svarene er:
\begin{align*}
2,&&5,&& \ \frac{1}{9},&& 3,&& 4, &&100.
\end{align*}
\item Svarene er:.
\begin{align*}
\frac{\sqrt{2}}{2},&& \sqrt{2},&& \sqrt{2},&& 3\sqrt{3},&& \frac{1}{2},&& 7^{2/3}.
\end{align*}

\item Svarene er:.
\begin{align*}
\sqrt{2}+2,&& 9\sqrt{2}-4,&& 1.
\end{align*}

\item Svarene er:.
\begin{align*}
x^{4/3},&& x^{-1/6} ,&& 2^{-3/4} x^{1/8}y^{3/4},&& x^{-1/2},&& x^{5/2}.
\end{align*}


\item Svarene er:
\begin{align*}
2\sqrt{x},&&\sqrt{3},&&\sqrt{7},&& \frac{x+y}{x-y}.
\end{align*}


\item Svarene er:
\begin{align*}
4,&& 9,&& 2\sqrt{2},&& 100000,&& -5.
\end{align*}

\item\label{it:root1ans} Ved at anvende kvadratsætningerne får vi at
\begin{align*}
(1\pm\sqrt{3})^2=1+3\pm2\sqrt{3}=4\pm 2\sqrt{3},
\end{align*}
og fra definitionen af kvadratroden følger det at 
\begin{align*}
1\pm\sqrt{3}=\sqrt{4+2\sqrt{3}}
\end{align*}

\item Svarene er:
\begin{align*}
\frac{\sqrt{7}(\sqrt{2}+6)}{3},&& \frac{(1-\sqrt{3})^2}{\sqrt{2}(1-\sqrt{3})}=\frac{\sqrt{2}-\sqrt{6}}{2},&& 24.
\end{align*}

\item Svarene er:
\begin{align*}
a^{11/12}b^{4/3},&&a^{-1/4},&& a^{-3}.
\end{align*}
\end{enumerate}