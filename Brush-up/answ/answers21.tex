\section{Regneregler for bestemte integraler}
\begin{enumerate}
	\item Svarene er:
	\begin{align*}
	\frac{7}{3},&&\frac{1}{2}, && \ln(2),&&0.
	\end{align*}
	
	\item Svarene er:
	\begin{align*}
	\frac{\sqrt{2}}{2}.
	\end{align*}
	
	\item Arealet er $e^3-e$.
	
	\item Vi har at 
	\begin{align*}
	\int_a^b 1\dd x=[x]_a^b=b-a.
	\end{align*}
	
	\item Da $\sqrt{x}>x^2$ i det givne interval har vi
	\begin{align*}
	\int_0^1 \sqrt{x}-x^2\dd x=[\frac{2}{3}x^{\frac{3}{2}}-\frac{1}{3}x^3]_0^1=\frac{1}{3}.
	\end{align*}
	
	\item Arealet af rektanglet kan beskrives som
	\begin{align*}
	\int_0^a b\dd x=[bx]_0^a=ba.
	\end{align*}
	
	\item \label{it:best2ans} Vi har at
	\begin{align*}
	\int_a^c f(x)\dd x+\int_c^b f(x)\dd x=F(c)-F(a)+F(b)-F(c)=F(b)-F(a)=\int_a^b f(x)\dd x.
	\end{align*}
	
	\item Husk at 
	\begin{align*}
	f(x)=\begin{cases}
	-x,& \textup{hvis }x<0\\
	x,& \textup{hvis }x\geq 0.
	\end{cases}
	\end{align*}
	Anvender vi hintet får fi at
\begin{align*}
	\int_{-2}^1 \abs{x}\dd x=\int_{-2}^0-x\dd x+\int_0^1 x\dd x=[-\frac{1}{2}x^2]_{-2}^0+[\frac{1}{2}x^2]_0^1=\frac{5}{2}.
\end{align*}
	

	
	\item \label{it:bes3ans} Vi har at
	\begin{align*}
	-\int_b^a f(x)\dd x=-(F(a)-F(b))=F(b)-F(a)=\int_a^b f(x)\dd x.
	\end{align*}
	
	\item Et muligt svar er $a=\pi$.
	
	\item Arealet er $2$.
	

	
	
	
	\item Svarene er:
	\begin{enumerate}
		\item Vi har at
		\begin{align*}
		\int_0^1 x^{\frac{1}{k}}-x^k\dd x=[\frac{k}{k+1}x^{\frac{k+1}{k}}-\frac{1}{k+1}x^{k+1}  ]_0^1=\frac{k-1}{k+1}
		\end{align*}
		\item $k=399$.	
		\item Vi får at
		\begin{align*}
		\frac{k-1}{k+1}=1\quad\Leftrightarrow\quad k-1=k+1,
		\end{align*}
		hvilket viser at denne ligning ikke har nogen løsninger.
		\item Det korrekte areal når $k=399$ er
		\begin{align*}
		\frac{398}{400}=\frac{199}{200}.
		\end{align*}
		
		\end{enumerate}
	
	\item Vi har at
	\begin{align*}
	\int_{-1}^{1}f(x)\dd x=\int_{-1}^{0}x^{\frac{2}{3}}\dd x+\int_{0}^{^1} x^{\frac{1}{3}}\dd x=[\frac{3}{5}x^{\frac{5}{3}}]_{-1}^0+[\frac{3}{4}x^{\frac{4}{3}}]_0^1=\frac{27}{20}.
	\end{align*}
	
	


	\item De to funktioner skærer hinanden når $x=\pm \frac{\sqrt{2}}{2}$. Vi har dermed at 
	\begin{align*}
	\int_{-\frac{\sqrt{2}}{2}}^\frac{\sqrt{2}}{2} 1-x^2-x^2\dd x=[x-\frac{2}{3}x^{3}]_{-\frac{\sqrt{2}}{2}}^\frac{\sqrt{2}}{2}=\frac{2\sqrt{2}}{3}.
	\end{align*}
	
	\item\label{it:bes1ans} Først ser vi at de to tangenter skærer hinanden i $(\frac{5}{4},0)$ samt at $y=-4x+5$ skærer parablen i $(0,5)$ og $y=x+\frac{5}{4}$ skærer parablen i $(\frac{5}{2},\frac{5}{2})$.  Ved at integrere får vi at
	\begin{align*}
	\int_0^{\frac{5}{4}} x^2-4x+5-(-4x+5)\dd x=\int_0^{\frac{5}{4}} x^2\dd x=[\frac{1}{3}x^3]_0^{\frac{5}{4}}=\frac{125}{192}
	\end{align*}
	og at
	\begin{align*}
	\int_{\frac{5}{4}}^{\frac{5}{2}} x^2-4x+5-(x-\frac{5}{4})\dd x&=\int_{\frac{5}{4}}^{\frac{5}{2}} x^2-5x+\frac{25}{4}\dd x\\&
	=[\frac{1}{3}x^3-\frac{5}{2}x^2+\frac{25}{4}x]_{\frac{5}{4}}^{\frac{5}{2}}\\
	&=\frac{125}{24}-\frac{125}{8}+\frac{125}{8}-(\frac{125}{192}-\frac{125}{32}+\frac{125}{16})\\
	&=\frac{125}{24}-\frac{125}{192}-\frac{125}{32}=\frac{1000-125-750}{192}=\frac{125}{192}.
	\end{align*}
	Dermed ses at de to arealer er lige store. Det totale areal er $ \frac{125}{96}$.
	
%			\begin{figure}
%		\centering
%		\begin{tikzpicture}
%		\begin{axis}[axis x line=center,xmin=-0.1,xmax=2.6,ymin=-0.1,ymax=5.1,axis y line=center]
%		\addplot[domain=-1:3,name path=A,thick,blue, samples = 600] {(x-2)^2+1} node[above right=5pt,pos=0.75] {\small$x^2-4x+5$};
%		\addplot[domain=-1:1.5,name path=B,thick,red, samples = 600] {-4*x+5} node[left,pos=0.75] {\small$-4x+5$};
%		\addplot[domain=1:3,name path=C,thick,red, samples = 600] {x-5/4} node[right,below,pos=0.6] {\small$x+\frac{5}{4}$};
%		\addplot[gray!50] fill between[of=A and B,soft clip={domain=0:5/4}];
%		\addplot[gray!25] fill between[of=A and C,soft clip={domain=5/4:5/2}];
%		\end{axis}
%		\end{tikzpicture}
%		\caption{Opgave~\ref{it:bes1}}
%		\label{fig:bes1}
%	\end{figure}
%	
	
\end{enumerate}