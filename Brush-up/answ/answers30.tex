\section{Planer i rummet}
\begin{enumerate}
	\item Parameterfremstillingen er:
	\begin{align*}
	\begin{bmatrix}
	x\\y\\z
	\end{bmatrix}=\begin{bmatrix}
	1\\-2\\6
	\end{bmatrix}+t \begin{bmatrix}
	2\\1\\-2
	\end{bmatrix}.
	\end{align*}
	Punktet $P$ ligger ikke på linjen.
	
	\item Ligningen for planen er
	\begin{align*}
	3(x-1)+2(y+5)+6(z-4)=0.
	\end{align*}
	
	\item Afstanden er $\sqrt{30}$
	
	\item Indsætter vi koordinaterne for linjen i planens ligning får vi
	\begin{align*}
	0&=3(1+t)-2(-2-2t)+(2+4t)-20\\&=3+3t+4+4t+2+4t-20=11t-11.
	\end{align*}
	Denne ligning har løsningen $t=1$ og dermed bliver skæringspunktet $(2,-4,6)$.
	
	\item Ved skiftevis at holde to af variablerne $x,y,z$ lig $0$ får vi skæringspunkterne $(0,0,12)$, $(0,-6,0)$ og $(3,0,0)$.

	
	\item Da $\vec{u}\times \vec{v}=0$ udspænder vektorerne ikke en plan.
	
	\item Hvis vi lægger de to ligninger sammen får vi at $3x=2$ hvilket giver at $x=\frac{2}{3}$. Ganger vi den første ligning igennem med $2$ og trækker den fra den anden får vi ligningen
	\begin{align*}
	3y+3z=2.
	\end{align*}
	Isolerer vi for $y$ får vi at $y=\frac{2}{3}-z$ og hvis vi skriver $t=z$ får vi parameterfremstillingen
	\begin{align*}
	\begin{bmatrix}
	x\\y\\z
	\end{bmatrix}=\frac{1}{3}\begin{bmatrix}
	2\\2\\0
	\end{bmatrix}+t \begin{bmatrix}
	0\\-1\\1
	\end{bmatrix}.
	\end{align*}
	
	
	
	\item Svarene er
	\begin{enumerate}
		\item Vi har at 
		\begin{align*}
		\vec{v}\times \vec{v}=4\begin{bmatrix}
		1\\-2\\2
		\end{bmatrix},
		\end{align*}
		hvorfor planens ligning bliver
		\begin{align*}
		x-2y+2z=0.
		\end{align*}
		\item Arealet er $12$.
		\item Krydsproduktet for normalvektorerne til $P$ og $Q$ er
		\begin{align*}
		\begin{bmatrix}
		-10\\2\\7
		\end{bmatrix}.
		\end{align*}
		og for at planerne kan være parallelle eller sammenfaldende skal dette krydsprodukt være $\vec{0}$.
		\item Indsætter vi linjens koordinater i venstresiden af ligningen for $P$ får vi
		\begin{align*}
		\frac{2}{7}-10t-2(\frac{1}{7}+2t)+2(7t)=0
		\end{align*}
		 hvilket viser at linjen er indehold i $P$. Gør vi det samme for $Q$ får vi
		 \begin{align*}
		 2(\frac{2}{7}-10t)+3(\frac{1}{7}+2t)+2(7t)=1.
		 \end{align*}
		Altså ligger linjen både i $P$ og $Q$.
	\end{enumerate}

	\item Afstanden er $2$.


	\item Vi har at
	\begin{align*}
	\vec{w}=\frac{1}{5}\begin{bmatrix}
	6\\15\\7
	\end{bmatrix}.
	\end{align*}
	
	
\end{enumerate}