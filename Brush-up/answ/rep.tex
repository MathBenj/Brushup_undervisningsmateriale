\section{Repetition}
\begin{enumerate}
	\item Svarene er:
	\begin{align*}
	-\frac{1}{12},&& \frac{1}{2},&& \frac{11}{6}.
	\end{align*}
	\item Svarene er:
	\begin{align*}
	x=25,&& x=2,x=-\frac{1}{2},&& x=1,x=-3,&& x=\pm 2,x=\pm \sqrt{2}.
	\end{align*}
	\item Svarene er:
	\begin{align*}
	\frac{x-2}{x+3},&& -\frac{x+3}{x+1}.
	\end{align*}
	\item Svarene er:
	\begin{align*}
	27x^6,&& y^-1,&& \sqrt{x}.
	\end{align*}
	\item Svarene er: 
\begin{align*}
x=2,y=\frac{-1}{2}.
\end{align*}
	\item $g$ er surjektiv men ikke injektiv og $f$ er hverken injektiv eller surjektiv.
	
	\item Svarene  $h(g(-1))=-1$ og $ g(f(3))=2 $.
	
	\item Svarene er
	\begin{align*}
	f(g(x))=\frac{1}{1+\cos^4(x)},&& g(f(x))=\cos^2(\frac{1}{1+x^2}).
	\end{align*}
	
	\item Svarene er:
	\begin{align*}
	6,&& 27,&& \frac{1}{\sqrt[3]{e}}.
	\end{align*}
	
	\item Svarene er:
	\begin{align*}
	x=1, && x=\frac{1}{2}.
	\end{align*}
	
	\item Svarene er:
	\begin{align*}
	\frac{3}{2},&& 3.
	\end{align*}
	
	\item Funkionen $f$ er kontinert på mængden $\R\setminus{0,1,2}$, altså i alle punkter undtagen 1,2 og 3.
	
	\item Svaret er: $\lim_{x\to 2} xe^{x^2-4}-x=0$.
	
	\item Svarene er:
	\begin{align*}
	f'(x)=4x+\frac{1}{2x^{\frac{3}{2}}},&&g'(x)=\frac{2}{3}x^{-\frac{1}{3}}+\sin(x),&& h'(x)=\frac{3}{2x}+4xe^{2x^2}.
	\end{align*}
	
	\item Svarene er:
	\begin{align*}
	f'(x)=2x(1+\tan^2(x^2)),&& g(x)=e^{2\sin(x)}(\sin(2x)+\cos(x)),&& h(x)=-\frac{1}{2}x^{-\frac{3}{2}}.
	\end{align*}
	
	
	\item\label{it:rep1} 	\begin{enumerate}
		\item Den første blå og den første røde hører sammen.
		\item Den anden blå og den tredje røde hører sammen.
		\item Den tredje blå og den anden røde hører sammen.
	\end{enumerate}
	
	\item Svaret er: $(f\circ g)'(3)=\frac{1}{4}-\frac{\pi}{6}$. 
	
	
 	\item\label{it:rep3} Det ses let at $f'(x)=2x-4$ så $f'(x)=0$ har løsningen $x=2$. Vælger vi punkterne $x_1=0$ og $x_2=3$ ser vi at $f'(x_1)=-4$ og at $f(x_2)=2$. Dette giver monotonilinjen som ses i Tabel~\ref{fig:mon1}. Vi ser dermed at $f$ er aftagende i intervallet $ ]-\infty,2] $ og voksende i intervallet $[2,\infty[$. Tangenten gennem punktet $(1,f(1))$ har forskriften
 	\begin{align*}
 	y=-2(x-1)+1=-2x+3.
 	\end{align*}
	
	\begin{table}[h!]
		\centering
		\begin{tabular}{@{}l  c c c@{}}
			$x$      & $0$ 		 & $2$				& $3$			\\ \toprule
			$f'(x)$  & $-4$		 &     $0$ 		 	& $2$			\\ \midrule
			$f(x)$   & $\searrow$&					& $\nearrow$	\\ \bottomrule  
		\end{tabular}
		\caption{Opgave~\ref{it:rep3}.}
		\label{fig:rep3}
	\end{table}
	
	
	
	\item Ved at undersøge kritiske punkter og interval endepunkter ses det at maksimumsværdien antages i $x=1$ samt at $f(1)=9$. Yderligere ses det at minimumsværdien tages i $x=-\frac{1}{3}$ samt at $f(-\frac{1}{3})=\frac{11}{3}$.
	
	\item\label{it:rep4} Lad $a$ og $b$ være givet som i Figur~\ref{fig:rep4} i.e.\ $b$ er højden af rektanglet og $a$ er halvdelen af længden. Ved at anvende trekantsberegninger får vi at
	\begin{align*}
	\tan(\frac{\pi}{3})=\frac{b}{\frac{1}{2}-a}.
	\end{align*}
	Isolerer vi $b$ får vi at
	\begin{align*}
	b=\frac{\sqrt{3}}{2}-\sqrt{3}a.
	\end{align*}
	Arealet af rektanglet kan dermed beskrives ved følgende funktion af $a$:
	\begin{align*}
	A(a)=2a(\frac{\sqrt{3}}{2}-\sqrt{3}a)=\sqrt{3}a-2\sqrt{3}a^2.
	\end{align*}
	Ved at anvende den sædvanlige optimeringsmetode finder vi at $A$ tager sit maksimum når $a=\frac{1}{4}$. Dette giver et maksimalt areal på $A(\frac{1}{4})=\frac{\sqrt{3}}{8}$. 
	
	
	
		\begin{figure}
			\centering
			\begin{tikzpicture}
			\begin{axis}[xmin=-0.2,xmax=1.1,ymin=-0.2,ymax=1.2,axis x line=center,
			axis y line=center, axis equal,xtick={0,1/2,1}]
			\addplot[blue,thick,domain=0:1/2] {sqrt(3)*x};
			\addplot[blue,thick,domain=1/2:1] {sqrt(3)-sqrt(3)*x};
			\addplot[blue,thick,domain=0:1] {0};
			
			\addplot[thick,red,domain=1/4:3/4] {0} node [pos=0.25,label=above: $a$] {}  node [pos=0.75,label=above: $a$] {} ;
			\addplot[thick,red,domain=1/4:3/4] {sqrt(3)/4};
			\addplot[thick,red] coordinates {({0.25},0) ({0.25},{sqrt(3)*0.25}) } node [pos=0.5,label=left: $b$] {};
			\addplot[thick,red] coordinates {({0.75},0) ({0.75},{sqrt(3)*0.25}) } node [pos=0.5,label=right: $b$] {};
			\end{axis}
			\end{tikzpicture}
			\caption{Opgave~\ref{it:rep4}}
			\label{fig:rep4}
		\end{figure}
	
	
	
	
	\item Nej.
	

	\item Svarene er:
	\begin{align*}
	\frac{1}{3}x^3+x+c,&& \frac{2}{3}x^{\frac{3}{2}}-\cos(x)+c,&& e^{\frac{x}{2}}+c.
	\end{align*}

	\item Svarene er:
	\begin{align*}
	x\sin(x)+\cos(x)+c,&& \frac{1}{3}x^3\ln(x)-\frac{1}{9}x^3+c.
	\end{align*}
	
	\item Svarene er:
	\begin{align*}
	e^{x^3+3x-1}+c,&& \frac{1}{2}(x^2\ln(x^2)-x^2)+c, && -e^{-(x^4+x^2-x)}+c.
	\end{align*}
	
	\item Svarene er:
	\begin{align*}
	\frac{8}{3},&& 0,&& 2e^{-1}-1.
	\end{align*}
	
	\item Udregn følgende bestemte integraler:
	\begin{align*}
	1-e,&& 0.
	\end{align*}
	
	\item\label{it:rep2} Vi har at
	\begin{align*}
	\int_0^{2\pi} f(x)\dd x&=\int_0^{\frac{\pi}{4}} \cos(x)-\sin(x) \dd x+\int_{\frac{\pi}{4}}^{\frac{5\pi}{4}} \sin(x)-\cos(x)\dd x+\int_{\frac{5\pi}{4}}^{2\pi} \cos(x)-\sin(x)\dd x\\
	&=[\sin(x)+\cos(x)]_{0}^{\frac{\pi}{4}}+[-\cos(x)-\sin(x)]_{\frac{\pi}{4}}^{\frac{5\pi}{4}}+[\sin(x)+\cos(x)]_{\frac{5\pi}{4}}^{2\pi}\\
	&=(\sqrt{2}-1)+(\sqrt{2}+\sqrt{2})+(1+\sqrt{2})\\
	&=4\sqrt{2}.
	\end{align*}
	
	\item Den fuldstændige løsning er $y(x)=ce^{-3x}$ hvor $c\in \R$. Da man blot bliver bedt om at finde en løsning kunne man være doven og vælge $y(x)=0$.
	
	\item Ved at anvende kædereglen ses at $ f'(x)=-e^x e^{-e^x} $. Indsætter vi $f$ og $f'$ i differentialligningen får vi
	\begin{align*}
	-\frac{f'(x)}{f(x)}=-\frac{-e^xe^{-e^x}}{e^{-e^x}}=-(-e^x)=e^x.
	\end{align*}
	
	\item Funktionerne $y_1$ og $y_2$ er løsninger.
	
	\item Tangentligningen har forskriften $y=\ln(2)$.
	
	\item Vi har at
	\begin{align*}
	y_1'(x)&=\frac{d}{dx}\cos(x)=-sin(x)=-y_2(x)\\
	y_2'(x)&=\frac{d}{dx}\sin(x)=\cos(x)=y_1(x).
	\end{align*}
	Derudover er det velkendt at  $\cos(0)=1$ og $\sin(0)=0$. 
	
	\item Bruger vi Panzerformlen får vi at
	\begin{align*}
	y(x)=e^{-\frac{1}{2}x^2}\int xe^{\frac{1}{2}x^2}\dd x +ce^{-\frac{1}{2}x^2}=e^{-\frac{1}{2}x^2}e^{\frac{1}{2}x^2}+ce^{-\frac{1}{2}x^2}=1+ce^{-\frac{1}{2}x^2}.
	\end{align*}

	
	\item Bruger vi Panzerformlen får vi at
	\begin{align*}
	y(x)=e^{-\cos(x)}\int \sin(x)e^{\cos(x)}\dd x+ce^{-\cos(x)}=e^{-\cos(x)}(-e^{\cos(x)})+ce^{-\cos(x)}=ce^{-\cos(x)}-1.
	\end{align*}
	
	\item Funktionerne $y_1$, $y_3$ og $y_4$ løser differentialligningen.
	
	\item Svarene er:
	\begin{align*}
	\begin{bmatrix}
	-4\\2
	\end{bmatrix},&&\begin{bmatrix}
	-2\\-4
	\end{bmatrix},&&\begin{bmatrix}
	3\\-4
	\end{bmatrix},&&\begin{bmatrix}
	7\\4
	\end{bmatrix},&&2\sqrt{5},&&\sqrt{5},&&0,&& 0.
	\end{align*}
	
	\item Arealet er $5$.
	
		
	\item Vi har at
	\begin{align*}
	\vec{u}\cdot \vec{v}=-\Big(\frac{\sqrt{2}}{2}\Big)^2+\Big(\frac{\sqrt{2}}{2}\Big)^2=0,
	\end{align*}
	samt at
	\begin{align*}
	\norm{\vec{u}}&=\sqrt{\Big(\frac{\sqrt{2}}{2}\Big)^2+\Big(-\frac{\sqrt{2}}{2}\Big)^2}=\sqrt{\frac{1}{2}+\frac{1}{2}}=1,\\
	\norm{\vec{v}}&=\sqrt{\Big(-\frac{\sqrt{2}}{2}\Big)^2+\Big(-\frac{\sqrt{2}}{2}\Big)^2}=\sqrt{\frac{1}{2}+\frac{1}{2}}=1,
	\end{align*}
	
	\item Nej.
	
	\item  	\item Svarene er:
	\begin{align*}
	\begin{bmatrix}
	-1\\1\\-2
	\end{bmatrix},&&\begin{bmatrix}
	0\\-2\\6
	\end{bmatrix},&&\begin{bmatrix}
	1\\0\\-1
	\end{bmatrix},&&\begin{bmatrix}
	1\\-2\\5
	\end{bmatrix},&&\sqrt{6},&&\sqrt{10},&&-7,&& \begin{bmatrix}
	1\\3\\1
	\end{bmatrix}.
	\end{align*}
	
	\item Arealet er $ 6\sqrt{3}$.
	
	\item Bestem en parameterfremstilling for linjen $m$ gennem punkterne $P_1 =
	(2, 3,-1)$ og $P_2 = (2,-2, 0)$. Ligger $P = (2, 8, -2)$ på $m$?
	
	\item En mulig parameterfremstilling er
	\begin{align*}
	\begin{bmatrix}
	x\\y\\z
	\end{bmatrix}=\begin{bmatrix}
	2\\-2\\0
	\end{bmatrix}+t \begin{bmatrix}
	0\\5\\-1
	\end{bmatrix}.
	\end{align*}
	$P$ ligger på linjen.
	
	
	En mulig ligning for planen er
	\begin{align*}
	2x-y+3z-4=0
	\end{align*}
	Punktet $P_1$ ligger i planen.
	
\end{enumerate}