\section{Andengradsligninger og flere ligninger med flere ubekendte}
 \begin{enumerate}
\item Svarene er:
\begin{align*}
x^2-x-2=0,&& x^2-\frac{7}{2}x+\frac{3}{2}=0,&& x^2-2 ,&& x^2-x-1=0.
\end{align*}

\item Svarene er:
\begin{align*}
x=\pm 6,&&x=1,x=-2,&& x=-1,x=-\frac{1}{2},&& x=-1,x=-3.
\end{align*}

\item Svarene er:
\begin{align*}
x=0,y=4
\end{align*}
\item Svarene er:
\begin{align*}
x=0,x=\frac{3}{2},&& x=0, x=7,&& x=0,x=\frac{2}{5},&& x=\pm \frac{7}{11}.
\end{align*}

\item Svarene er:
\begin{align*}
x=3,y=3
\end{align*}

\item Svarene er:
\begin{align*}
\frac{x+6}{x+1},&& x-1&&\frac{x+1}{x-2}.
\end{align*}


\item Svarene er:
\begin{align*}
x=1,x=-4,&&x=-1,x=4,&&  x=\frac{7\pm \sqrt{39}}{5},&& x=\pm 5.
\end{align*}



\item Svarene er:
\begin{enumerate}
\item I en afstand af $\frac{3}{10}m$ fra den ene ende af stangen.
\item I en afstand af $\frac{7-\sqrt{23}}{20}m$ fra den ene ende af stangen.
\end{enumerate}

\item Svarene er:
\begin{alignat*}{5}
x=6,y=6,z=6.
\end{alignat*}


\item Svarene er: 
\begin{align*}
x=\pm 3, && x=-2,x=-\frac{2}{3},&& x=\pm \frac{1}{6}.
\end{align*}

\item Svarene er:
\begin{align*}
 x=-1,x=\frac{3}{4},&& x=-7,x=10,&&x=3.
\end{align*}

\item Svaret er $b=\pm 4$.

\item Svaret er $a>4$.

\item Svarene er:
\begin{align*}
x=\pm 2,&& x=\pm 3,&& x=2,x=-1.
\end{align*}



\item \label{it:2polyans} Polynomierne skærer hinanden i $(-1,2)$ og $(3,1))$.
%\begin{figure}
%\centering
%\begin{tikzpicture}
%\begin{axis}[xmin=-2,xmax=4,ymin=0,ymax=5,axis x line=center,
%  axis y line=center, ticks=none]
%\addplot[thick,blue,samples =100] {1/8*x*x-1/2*x+11/8};
%\addplot[thick,red,samples=100] {-3/4*x*x+5/4*x+4};
%\node[fill, circle, inner sep=1pt] at (axis cs:-1,2) {};
%\node[fill, circle, inner sep=1pt] at (axis cs:3,1) {};
%\end{axis}
%\end{tikzpicture}
%\caption{Opgave~\ref{it:2poly}}
%\label{fig:2poly}
%\end{figure}

\item Svarene er:
\begin{align*}
x=7,y=8,z=9,w=10.
\end{align*}

\item \label{it:phians} Svaret er
\begin{align*}
\phi=\frac{1+\sqrt{5}}{2}.
\end{align*}

%
%\begin{figure}
%	\centering
%	\begin{tikzpicture}
%	\draw (0,0)-- node[above] {$b$} (3,0) -- node[above] {$a$} ({3+3*(1+sqrt(5))/2},0);
%	\node[fill, inner sep =1pt,circle] at (3,0) {};
%	\end{tikzpicture}
%	\caption{Opgave~\ref{it:phi}}
%	\label{fig:phi}
%\end{figure}

\item Svarene er:
\begin{enumerate}
	\item $A(4)=\frac{1}{2^4}=\frac{1}{16}$
	\item Ved at løse de to ligninger med to ubekendte
	\begin{align*}
	ab&=\frac{1}{16}\\
	\frac{b}{a}&=\frac{\frac{a}{2}}{b},
	\end{align*}
	fås $ a=\frac{\sqrt[4]{2}}{4} $ og $ b= \frac{\sqrt[4]{2^3}}{8}$
	\item 	Ved at isolere $b$ i den anden af de to ligninger med to ubekendte
	\begin{align*}
	ab&=\frac{1}{2^n}\\
	\frac{b}{a}&=\frac{\frac{a}{2}}{b},
	\end{align*}
	får vi at $b=\frac{a}{\sqrt{2}}$. Indsættes dette i den første ligning får vi at 
	\begin{align*}
	\frac{a^2}{\sqrt{2}}=2^{-n}.
	\end{align*}
	Ganger vi igennem med $ \sqrt{2} $ og anvender regneregler for rødder og potenser får vi at
	\begin{align*}
	a=2^{\frac{-2n+1}{4}}
	\end{align*}
	og indsætter vi dette i formlen $b=\frac{a}{\sqrt{2}}$ får vi
	\begin{align*}
	b=2^{\frac{-2n-1}{4}}.
	\end{align*}
\end{enumerate}

%\begin{figure}
%	\centering
%	\begin{tikzpicture}
%	\draw (0,0)-- node[left] {$b$} (0,3)-- node[above] {$a$} ({sqrt(18)},3)--({sqrt(18)},0)--cycle;
%	\draw[dashed] ({sqrt(9/2)},3)--({sqrt(9/2)},0);
%	\node at (({sqrt(9/2)}/2,0) [label= below: $\frac{a}{2}$] {};
%	\end{tikzpicture}
%	\caption{A papirformat}
%	\label{fig:2deglig1}
%\end{figure}

\item Fra Opgave~\ref{it:ex13} har vi allerede at
\begin{align*}
\Big( x+\frac{b}{2a}\Big)^2=\frac{b^2-4ac}{4a^2},
\end{align*}
og tager vi kvadratroden på begge sider fås
\begin{align*}
x+\frac{b}{2a}=\pm\frac{\sqrt{b^2-4ac}}{2a}.
\end{align*}
Hvis vi trækker $ \frac{b}{2a} $ fra på begge sider får vi den velkendte formel
\begin{align*}
x=\frac{-b\pm \sqrt{b^2-4ac}}{2a}.
\end{align*}

\end{enumerate}
