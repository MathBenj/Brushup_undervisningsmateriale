\section{Regneregler for ubestemte integraler}

\begin{enumerate}
	\item Svaret kan være $F(x)=\frac{1}{2}x^2+2x$.
	
	\item Svaret er $f(x)= x^{-\frac{1}{2}}$.
	
	\item Vi har at
	\begin{align*}
	F'(x)=\frac{3}{5}\frac{10}{3}x^{\frac{10}{3}-1}=2x^{\frac{7}{3}}=f(x).
	\end{align*}
	
	
	\item Svarene er:
	\begin{align*}
	\frac{1}{3}x^3+x^2+c,&& 3(e^x+\cos x)+c,&& e^{2x}-\ln x+c.
	\end{align*}
	
	\item $F(x)=\frac{1}{3}x^3+\frac{1}{2}x^2+\frac{1}{6}$.
	
	\item Svarene er:
	\begin{align*}
	2\ln(x)+\frac{2}{3}x^{\frac{3}{2}}+\frac{1}{2}x^2+c,&& x^{\frac{5}{2}}+\frac{1}{2}\frac{1}{x^2} +c,&& c
	\end{align*}
	
	\item\label{it:int11ans} Svarene er:
	\begin{enumerate}
		\item Da $f+g=e^x$ er $f+g$ en stamfunktion til sig selv.
		\item Vi har at
		\begin{align*}
		f'(x)=\frac{1}{2}(e^x-(-1)e^{-x})=\frac{1}{2}(e^x+e^{-x})=g(x).
		\end{align*}
		\item Ja.
		\item Vi har at
		\begin{align*}
		\frac{d}{dx}\ln(g(x))=\frac{1}{g(x)}g'(x)=\frac{f(x)}{g(x)}.
		\end{align*}
	\end{enumerate}

	\item $F(x)=\frac{2}{7}x^{\frac{7}{2}}+\frac{3}{2}x^{\frac{4}{3}}+\sqrt{2}$.
	
	\item Vi har at 
	\begin{align*}
	F'(x)&=-(1-x)^{-2}(-1)=(1-x)^{-2}=\frac{1}{(1-x)^2}\\
	G'(x)&=(1-x)^{-1}+x(1-x)^{-2}=\frac{1}{1-x}+\frac{x}{(1-x)^2}=\frac{1}{(1-x)^2}.
	\end{align*}
	Yderligere er $F(x)-G(x)=1$.

	\item Fra Opgave~\ref{it:diff21} har vi at
	\begin{align*}
	\frac{d}{dx} \tan x=1+\tan^2 x
	\end{align*}
	Skriver vi $1=\frac{d}{dx}x$ og isolerer for $\tan^2x$ får vi at
	\begin{align*}
	\tan^2x=\frac{d}{dx} (\tan(x)-x).
	\end{align*}
	
	
	\item Da
	\begin{align*}
	\frac{d}{dx} x\ln x=\ln(x)+1
	\end{align*}
	er $f$ ikke en stamfunktion til $\ln x$. I Opgave~\ref{it:diff23} så vi at $x\ln(x)-x$ er en stamfunktion til $\ln x$.
	
	
	
\end{enumerate}